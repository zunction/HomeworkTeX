\documentclass[a4paper,10pt]{article}
\setlength{\parindent}{0cm}
\usepackage{amsmath, amssymb, amsthm, mathtools,pgfplots}
\usepackage{graphicx,caption}
\usepackage{verbatim}
\usepackage{venndiagram}
\usepackage[cm]{fullpage}
\usepackage{fancyhdr}
\usepackage{tikz}
\usepackage{listings}
\usepackage{multirow,array}
\usepackage{color,enumerate,framed}
\usepackage{color,hyperref}
\definecolor{darkblue}{rgb}{0.0,0.0,0.5}
\hypersetup{colorlinks,breaklinks,
            linkcolor=darkblue,urlcolor=darkblue,
            anchorcolor=darkblue,citecolor=darkblue}
\usetikzlibrary{calc,matrix,positioning}
\pgfplotsset{compat=1.12}
%\usepackage{tgadventor}
%\usepackage[nohug]{diagrams}
\usepackage[T1]{fontenc}
%\usepackage{helvet}
%\renewcommand{\familydefault}{\sfdefault}
%\usepackage{parskip}
%\usepackage{picins} %for \parpic.
%\newtheorem*{notation}{Notation}
%\newtheorem{example}{Example}[section]
%\newtheorem*{problem}{Problem}
\theoremstyle{definition}
%\newtheorem{theorem}{Theorem}
%\newtheorem*{solution}{Solution}
%\newtheorem*{definition}{Definition}
%\newtheorem{lemma}[theorem]{Lemma}
%\newtheorem{corollary}[theorem]{Corollary}
%\newtheorem{proposition}[theorem]{Proposition}
%\newtheorem*{remark}{Remark}
%\setcounter{section}{1}

\newtheorem{thm}{Theorem}[section]
\newtheorem{lemma}[thm]{Lemma}
\newtheorem{prop}[thm]{Proposition}
\newtheorem{cor}[thm]{Corollary}
\newtheorem{defn}[thm]{Definition}
\newtheorem*{examp}{Example}
\newtheorem{conj}[thm]{Conjecture}
\newtheorem{rmk}[thm]{Remark}
\newtheorem*{nte}{Note}
\newtheorem*{notat}{Notation}

%\diagramstyle[labelstyle=\scriptstyle]

\lstset{frame=tb,
  language=Oz,
  aboveskip=3mm,
  belowskip=3mm,
  showstringspaces=false,
  columns=flexible,
  basicstyle={\small\ttfamily},
  breaklines=true,
  breakatwhitespace=true,
  tabsize=3
}

\DeclareMathOperator*{\argmin}{argmin}
\DeclareMathOperator*{\argmax}{argmax}
\newcommand*{\argminl}{\argmin\limits}
\newcommand*{\argmaxl}{\argmax\limits}

\pagestyle{fancy}




\fancyhead{}
\renewcommand{\headrulewidth}{0pt}

\lfoot{\color{black!60}{\sffamily Zhangsheng Lai}}
\cfoot{\color{black!60}{\sffamily Last modified: \today}}
\rfoot{\textsc{\thepage}}



\begin{document}

\begin{lemma}
For $x\in (0,1)$
\begin{align*}
(1-x)^{1/x}\leq (e^{-x})^{1/x}=e^{-1}
\end{align*}
\label{lemma:exp}
\end{lemma}


\begin{thm}
Consider a $(\lambda,\mu)-$cost minimization game with a positive potential function $\Phi$ such that $\Phi(\mathbf{s}) \leq cost(\mathbf{s})$ for every outcome $\mathbf{s}$. Let $\mathbf{s^0},\mathbf{s^1},\ldots,\mathbf{s^T}$ be a sequence generated by MaxGain best response dynamics, $\mathbf{s^\ast}$ a minimum cost outcome and $1>\gamma>0$ is a parameter, Then for all but 
\begin{align}
%O\left(
\frac{k}{\gamma(1-\mu)}\log\frac{\Phi(\mathbf{s^0})}{\Phi_{min}}\label{eq:outcomes}
%\right)
\end{align}
outcomes $\mathbf{s}^t$ satisfy
%\begin{align*}
%cost(\mathbf{s^t})\leq \left(\frac{\lambda}{1-\mu}+\frac{1}{1-\gamma}\right)\cdot cost(\mathbf{s^\ast})
%\end{align*}
\begin{align}
%cost(\mathbf{s^t})\leq \left(\frac{\lambda}{1-\mu}+\gamma\right)\cdot cost(\mathbf{s^\ast})
cost(\mathbf{s^t})\leq \left(\frac{\lambda}{(1-\mu)(1-\gamma)}\right)\cdot cost(\mathbf{s^\ast})\, \label{eq:good}
\end{align}
\begin{proof}
\begin{align}
cost(\mathbf{s^t})&\leq \sum_i c_i(\mathbf{s^t})\notag\\
&=\sum_i\left[c_i(s_i^\ast,s_{-i}^t)+\delta_i(\mathbf{s^t})\right],\quad \delta_i(\mathbf{s^t})=c_i(\mathbf{s^t})-c_i(s_i^\ast,s^t_{-i})\notag\\
&\leq \lambda \cdot cost(\mathbf{s^\ast})+\mu \cdot cost(\mathbf{s^t})+\sum_i\delta_i(\mathbf{s^t})\notag\\
cost(\mathbf{s^t})&\leq \frac{\lambda}{1-\mu}\cdot cost(\mathbf{s^\ast}) + \frac{1}{1-\mu}\cdot \sum_i\delta_i(\mathbf{s^t}) \label{eq:2}
\end{align}
we shall let $\Delta(\mathbf{s^t})=\sum_i\delta_i(\mathbf{s^t})$ in the remaining parts of the proof. We shall now define a state $\mathbf{s^t}$ to be bad if it does not satisfy (\ref{eq:good}) and 
%\begin{align}
%cost(\mathbf{s^t})>\left(\frac{\lambda}{1-\mu}+\frac{1}{1-\gamma}\right)\cdot cost(\mathbf{s^t})
%\end{align} 
%\begin{align*}
%cost(\mathbf{s^t})> \left(\frac{\lambda}{1-\mu}+\gamma\right)\cdot cost(\mathbf{s^\ast})
%\end{align*}
by (\ref{eq:2}), when $\mathbf{s^t}$ is bad we get
\begin{align*}
\Delta(\mathbf{s^t})&\geq \gamma(1-\mu)\cdot cost(\mathbf{s^t})\\
\end{align*}
By the MaxGain definition and the inequality relating the potential function and cost,
\begin{align*}
\max_{i}\delta_i(\mathbf{s^t})\geq \frac{\Delta(\mathbf{s^t})}{k}\geq \frac{\gamma(1-\mu)}{k}\cdot cost(\mathbf{s^t})\geq \frac{\gamma(1-\mu)}{k}\cdot \Phi(\mathbf{s^t})
\end{align*}
and we get what we desire as
\begin{align*}
\Phi(\mathbf{s^t})-\Phi(s_i^\ast,s^t_{-i})
=c_i(\mathbf{s^t})-c_i(s_i^\ast,s^t_{-i})&=\delta_i(\mathbf{s^t})
\end{align*}
and hence
\begin{align}
%\left(1-\frac{\gamma(1-\mu)}{k}\right)\Phi(\mathbf{s^t})\geq \Phi(s_i^\ast,s^t_{-i})
\left(1-\frac{\gamma(1-\mu)}{k}\right)\Phi(\mathbf{s^t})\geq \Phi(\mathbf{s^{t+1}})\label{eq:result}
\end{align}
whenever $\mathbf{s^t}$ is a bad state. The equation in (\ref{eq:result}) says that for every MaxGain best response dynamics, if the state is bad, the new state $\mathbf{s^{t+1}}$ is smaller than the previous state $\mathbf{s^t}$ by a factor of $1-\frac{\gamma(1-\mu)}{k}$. By Lemma \ref{lemma:exp}, the potential decreases by a factor of $e$ for every $\frac{k}{\gamma(1-\mu)}$ bad states encountered. Thus solving 
\begin{align*}
e^{-n}\Phi(\mathbf{s^0}) \geq \Phi_{min}
\end{align*}
shows (\ref{eq:outcomes}).
\end{proof}
\end{thm}
\end{document}