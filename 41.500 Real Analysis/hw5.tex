\documentclass[a4paper,12pt]{article}
\setlength{\parindent}{0cm}
\usepackage{amsmath, amssymb, amsthm, mathtools,pgfplots,bbm}
\usepackage{graphicx,caption}
\usepackage{verbatim}
\usepackage{venndiagram}
\usepackage[cm]{fullpage}
\usepackage{fancyhdr}
\usepackage{tikz}
\usepackage{listings}
\usepackage{color,enumerate,framed}
\usepackage{color,hyperref}
\definecolor{darkblue}{rgb}{0.0,0.0,0.5}
\hypersetup{colorlinks,breaklinks,
            linkcolor=darkblue,urlcolor=darkblue,
            anchorcolor=darkblue,citecolor=darkblue}

%\usepackage{tgadventor}
%\usepackage[nohug]{diagrams}
\usepackage[T1]{fontenc}
%\usepackage{helvet}
%\renewcommand{\familydefault}{\sfdefault}
%\usepackage{parskip}
%\usepackage{picins} %for \parpic.
%\newtheorem*{notation}{Notation}
%\newtheorem{example}{Example}[section]
%\newtheorem*{problem}{Problem}
\theoremstyle{definition}
%\newtheorem{theorem}{Theorem}
%\newtheorem*{solution}{Solution}
%\newtheorem*{definition}{Definition}
%\newtheorem{lemma}[theorem]{Lemma}
%\newtheorem{corollary}[theorem]{Corollary}
%\newtheorem{proposition}[theorem]{Proposition}
%\newtheorem*{remark}{Remark}
%\setcounter{section}{1}

\newtheorem{thm}{Theorem}[section]
\newtheorem{lemma}[thm]{Lemma}
\newtheorem{prop}[thm]{Proposition}
\newtheorem{cor}[thm]{Corollary}
\newtheorem{defn}[thm]{Definition}
\newtheorem*{examp}{Example}
\newtheorem{conj}[thm]{Conjecture}
\newtheorem{rmk}[thm]{Remark}
\newtheorem*{nte}{Note}
\newtheorem*{notat}{Notation}

%\diagramstyle[labelstyle=\scriptstyle]

\lstset{frame=tb,
  language=Oz,
  aboveskip=3mm,
  belowskip=3mm,
  showstringspaces=false,
  columns=flexible,
  basicstyle={\small\ttfamily},
  breaklines=true,
  breakatwhitespace=true,
  tabsize=3
}


\pagestyle{fancy}




\fancyhead{}
\renewcommand{\headrulewidth}{0pt}

\lfoot{\color{black!60}{\sffamily Zhangsheng Lai}}
\cfoot{\color{black!60}{\sffamily Last modified: \today}}
\rfoot{\textsc{\thepage}}



\begin{document}
\flushright{Zhangsheng Lai\\1002554}
\section*{Real Analysis: Homework 5}

\begin{enumerate}
\item 
\begin{enumerate}[(a)]
\item
\begin{align*}
\frac{\partial}{\partial \epsilon}F(\epsilon,t) &= \int_{0}^{\infty}\frac{\partial}{\partial \epsilon}e^{-\epsilon x}\frac{\sin xt}{x}\,dx\\\
&=-\int_{0}^{\infty}e^{-\epsilon x} \sin xt\, dx\\
&=-\left\{\left[\sin xt\cdot -\frac{1}{\epsilon}e^{-\epsilon x}\right]_{0}^{\infty}-\int_{0}^{\infty}-\frac{1}{\epsilon}e^{-\epsilon x} t \cos xt \, dx\right\}\\
&=-\frac{t}{\epsilon}\int_{0}^{\infty}e^{-\epsilon x} \cos xt \, dx\\
&=-\frac{t}{\epsilon}\left\{\left[\cos xt \cdot -\frac{1}{\epsilon}e^{-\epsilon x}\right]_{0}^{\infty}-\int_{0}^{\infty}-\frac{1}{\epsilon}e^{-\epsilon x}\cdot -t \sin xt \, dx\right\}\\
&=\frac{t}{\epsilon^2}+\frac{t^2}{\epsilon^2}\int_{0}^{\infty}e^{-\epsilon x} \sin xt \, dx\\
\end{align*}
with some algebraic manipulation we obtain
\begin{align*}
\int_{0}^{\infty}e^{-\epsilon x} \sin xt \, dx = -\frac{t}{t^2+\epsilon^2}
\end{align*}

\item
We observe that
\begin{align*}
\left|e^{-\epsilon x}\frac{\sin xt}{x}\right| \leq \frac{e^{-\epsilon x}}{x}
\end{align*}
thus
\begin{align*}
\sup_{x,t \in \mathbb{R}}\left|e^{-\epsilon x}\frac{\sin xt}{x} - 0\right| \to 0 \text{ as } \epsilon \to \infty 
\end{align*}
Hence,
\begin{align*}
\lim_{\epsilon \to \infty}F(\epsilon,t) = \int_{0}^{\infty}\lim_{\epsilon \to \infty}e^{-\epsilon x}\frac{\sin xt}{x}\,dx=0
\end{align*}



\item 
As $e^{-\epsilon x}\frac{\sin xt}{x}$ is nonnegative, converges uniformly to $\frac{\sin xt}{x}$ as $\epsilon \to 0$, by Monotone convergence theorem, we have 
\begin{align}
\lim_{\epsilon \to 0}F(\epsilon,t) &= \int_{0}^{\infty}\frac{\sin xt}{x}\,dx \nonumber \\
&= \int_{0}^{\infty}\int_{0}^{\infty}e^{-xy}\sin xt\,dy\,dx \nonumber \\
&= \int_{0}^{\infty}\left(\int_{0}^{\infty}e^{-xy}\sin xt\,dx\right)\,dy \label{eq1}
\end{align}
we work on the inner integral first,
\begin{align*}
\int_{0}^{\infty}e^{-xy}\sin xt\,dx&=\left[\sin xt -\frac{1}{y}e^{-xy}\right]_{0}^{\infty}-\int_{0}^{\infty}-\frac{1}{y}e^{-xy} t\cos xt\,dx\\
&=\left[-\frac{1}{y}e^{-xy}\sin xt -\frac{t}{y^2}e^{-xy}\cos xt \right]_{0}^{\infty}-\frac{t^2}{y^2}\int_{0}^{\infty}e^{-xy} \sin xt\,dx\\
&=\left[\frac{-ye^{-xy}\sin xt-te^{-xy}\cos xt}{y^2} \right]_{0}^{\infty}-\frac{t^2}{y^2}\int_{0}^{\infty}e^{-xy} \sin xt\,dx\\
&=\frac{t}{y^2}-\frac{t^2}{y^2}\int_{0}^{\infty}e^{-xy} \sin xt\,dx
\end{align*}
thus
\begin{align*}
\int_{0}^{\infty}e^{-xy}\sin xt\,dx = \frac{t}{t^2+y^2}
\end{align*}
which then we apply it to (\ref{eq1}) to get
\begin{align*}
\lim_{\epsilon \to 0}F(\epsilon,t) &= \int_{0}^{\infty}\frac{t}{t^2+y^2}\,dy\\
&= \left[\tan^{-1}\frac{y}{t}\right]_{0}^{\infty}=\frac{\pi}{2}\text{sgn}(t)
\end{align*}


\end{enumerate}
\item
\begin{align*}
|x^3-a-bx-cx^2|^2& = x^6-2cx^5+(c^2-2b)x^4\\
&+(2bc-2a)x^3+(2ac+b^2)x^2+2abx+a^2\\
\frac{1}{2}\int_{-1}^{1}|x^3-a-bx-cx^2|^2\,dx &= \frac{1}{2}\Bigg[\frac{1}{7}x^7-\frac{1}{3}cx^6+\frac{1}{5}(c^2-2b)x^5+\frac{1}{4}(2bc-2a)x^4\\
&+\frac{1}{3}(2ac+b^2)x^3+abx^2+a^2x\Bigg]_{-1}^{1}\\
&=\frac{1}{5}c^2+\frac{2}{3}2ac+2a^2+\frac{1}{7}+\frac{1}{3}b^2-\frac{2}{5}b\\
\end{align*}
we see that it is minimum when $a=0=c$ and when $b = 3/5$. Thus
\begin{align*}
\min_{a,b,c} \frac{1}{2} \int_{-1}^{1}|x^3-a-bx-cx^2|^2\,dx = 4/175
\end{align*}


\item
\begin{enumerate}[(a)]
\item Let $n \neq m$,
\begin{align*}
\frac{1}{L}\int_{0}^{L}\sin \frac{n \pi x}{L}\sin \frac{m \pi x}{L}\,dx & = \frac{1}{2L}\int_{0}^{L}\cos \frac{(n-m) \pi x}{L}-\cos \frac{(n+m) \pi x}{L}\,dx\\
&=\frac{1}{2L}\left[-\frac{L}{(n-m)\pi}\sin \frac{(n-m) \pi x}{L}+\frac{L}{(n+m)\pi}\sin \frac{(n+m) \pi x}{L}\right]_{0}^{L}=0\\
\frac{1}{L}\int_{0}^{L}\sin^2 \frac{n \pi x}{L}\,dx &=\frac{1}{L}\int_{0}^{L}1 - \frac{1}{2}\cos \frac{2n \pi x}{L}\,dx\\
&=\frac{1}{L}\left[x + \frac{L}{4n\pi}\sin \frac{2n \pi x}{L}\right]_{0}^{L}=1\\
\end{align*}
Thus $\left\{\sin \frac{n\pi x}{L}\right\}, n \in \mathbb{N}$ is orthonormal. It is a basis as they are linearly independent. Suppose not, then there exists $\sin \frac{a\pi x}{L} a \in \mathbb{N}$ such that it can be expressed as a linear combination of $\left\{\sin \frac{n\pi x}{L}\right\}, n \in \mathbb{N}, a \neq n$,
\begin{align*}
\sin \frac{a\pi x}{L} = \sum_{n \in \mathbb{N}, n\neq a} c_n \sin \frac{n\pi x}{L}
\end{align*}
but then by the result proven above we have 
\begin{align*}
1 = \frac{1}{L}\int_{0}^{L}\sin^2 \frac{a\pi x}{L} \,dx= \sum_{n \in \mathbb{N}, n\neq a} c_n \frac{1}{L}\int_{0}^{L}\sin \frac{n\pi x}{L}\sin \frac{a\pi x}{L} \, dx = 0
\end{align*}
which is a contradiction, thus $\left\{\sin \frac{n\pi x}{L}\right\}, n \in \mathbb{N}$ is orthonormal. (still need to show it is a basis?)

\item The heat equation with Dirichlet boundary conditions is given by
\begin{align}
\frac{\partial u}{\partial t} &= k \frac{\partial u}{\partial x^2} \label{h1}\\
u(0,t) &= u(L,t)=0 \label{h2}\\
u(x,0) &= f(x) \label{h3}
\end{align}
we will show (\ref{h2}) and (\ref{h3}) first,
\begin{align*}
u(0,t) &= \sum_{n=1}^{\infty}A_ne^{-(\frac{n\pi}{L})^2kt}\sin \frac{n \pi 0}{L}=0\\
u(L,t) &= \sum_{n=1}^{\infty}A_ne^{-(\frac{n\pi}{L})^2kt}\sin n \pi=0\\
u(x,0)&=\sum_{n=1}^{\infty}A_ne^{-(\frac{n\pi}{L})^2k0}\sin \frac{n \pi x}{L}=\sum_{n=1}^{\infty}A_n\sin \frac{n \pi x}{L}=f(x)
\end{align*}
to show (\ref{h1}), we assume that term-by-term differentiation of the infinite series exists, then
\begin{align*}
\frac{\partial u}{\partial x} &= \sum_{n=1}^{\infty}A_n \frac{n\pi}{L}e^{-(\frac{n\pi}{L})^2kt}\cos \frac{n \pi x}{L}\\
\frac{\partial u}{\partial x^2}&= \sum_{n=1}^{\infty}A_n \cdot -\left(\frac{n\pi}{L}\right)^2e^{-(\frac{n\pi}{L})^2kt}\sin \frac{n \pi x}{L}=\frac{\partial u}{\partial t}
\end{align*}

\item


\item
\end{enumerate}



\end{enumerate}


\end{document}