\documentclass[a4paper,12pt]{article}
\setlength{\parindent}{0cm}
\usepackage{amsmath, amssymb, amsthm, mathtools,pgfplots,bbm}
\usepackage{graphicx,caption}
\usepackage{verbatim}
\usepackage{venndiagram}
\usepackage[cm]{fullpage}
\usepackage{fancyhdr}
\usepackage{tikz}
\usepackage{listings}
\usepackage{color,enumerate,framed}
\usepackage{color,hyperref}
\definecolor{darkblue}{rgb}{0.0,0.0,0.5}
\hypersetup{colorlinks,breaklinks,
            linkcolor=darkblue,urlcolor=darkblue,
            anchorcolor=darkblue,citecolor=darkblue}

%\usepackage{tgadventor}
%\usepackage[nohug]{diagrams}
\usepackage[T1]{fontenc}
%\usepackage{helvet}
%\renewcommand{\familydefault}{\sfdefault}
%\usepackage{parskip}
%\usepackage{picins} %for \parpic.
%\newtheorem*{notation}{Notation}
%\newtheorem{example}{Example}[section]
%\newtheorem*{problem}{Problem}
\theoremstyle{definition}
%\newtheorem{theorem}{Theorem}
%\newtheorem*{solution}{Solution}
%\newtheorem*{definition}{Definition}
%\newtheorem{lemma}[theorem]{Lemma}
%\newtheorem{corollary}[theorem]{Corollary}
%\newtheorem{proposition}[theorem]{Proposition}
%\newtheorem*{remark}{Remark}
%\setcounter{section}{1}

\newtheorem{thm}{Theorem}[section]
\newtheorem{lemma}[thm]{Lemma}
\newtheorem{prop}[thm]{Proposition}
\newtheorem{cor}[thm]{Corollary}
\newtheorem{defn}[thm]{Definition}
\newtheorem*{examp}{Example}
\newtheorem{conj}[thm]{Conjecture}
\newtheorem{rmk}[thm]{Remark}
\newtheorem*{nte}{Note}
\newtheorem*{notat}{Notation}

%\diagramstyle[labelstyle=\scriptstyle]

\lstset{frame=tb,
  language=Oz,
  aboveskip=3mm,
  belowskip=3mm,
  showstringspaces=false,
  columns=flexible,
  basicstyle={\small\ttfamily},
  breaklines=true,
  breakatwhitespace=true,
  tabsize=3
}


\pagestyle{fancy}




\fancyhead{}
\renewcommand{\headrulewidth}{0pt}

\lfoot{\color{black!60}{\sffamily Zhangsheng Lai}}
\cfoot{\color{black!60}{\sffamily Last modified: \today}}
\rfoot{\textsc{\thepage}}



\begin{document}
\flushright{Zhangsheng Lai\\1002554}
\section*{Real Analysis: Homework 4}


\begin{proof}
\hspace{1em}
\begin{enumerate}[(a)]
\item  
\begin{align*}
\mathbb{P}[|X_t-X_s|\geq \epsilon] &= \mathbb{P}[|X_t-X_s|^\alpha\geq \epsilon^\alpha]\\
&\leq \epsilon^{-\alpha}\mathbb{E}[|X_t-X_s|^\alpha],~ \text{ by Markov Inequality}\\
& \leq  \epsilon^{-\alpha}|t-s|^{1+\beta}
\end{align*}
thus as $s \to t$, we have $\mathbb{P}[|X_t-X_s|\geq \epsilon] \to 0$ which shows that $X_s \to X_t$ in probability as $s \to t$.


\item We need to show that for $n \geq N(\omega)$,
\begin{align}
\mathbb{P}\left[\max_{1 \leq k \leq 2^n}\left|X_{\frac{kT}{2^n}}-X_{\frac{(k-1)T}{2^n}}\right|<2^{-\gamma n}\right]=1\label{eq1}
\end{align}
so from (a), we get for all $1 \leq k \leq 2^n$,
\begin{align*}
\mathbb{P}\left[\left|X_{\frac{kT}{2^n}}-X_{\frac{(k-1)T}{2^n}}\right|<2^{-\gamma n}\right] &= 1- \mathbb{P}\left[\left|X_{\frac{kT}{2^n}}-X_{\frac{(k-1)T}{2^n}}\right|\geq 2^{-\gamma n}\right]\\
& \geq 1 - (2^{-\gamma n})^{-\alpha}\left|\frac{T}{2^n}\right|^{1+\beta}\\
& = 1 - |T|^{1+\beta}\cdot 2^{-n(1+\beta-\gamma\alpha)} \to 1 \text{ as } n \to \infty 
\end{align*}
thus we have we result in (\ref{eq1}).

\item
We first see that $D = \bigcup_{n=1}^{\infty}D_n$ where $D_n:=\{(kT/2^n)\mid k=0,1,\ldots, 2^n\}$ is the partition of $[0,T]$ and show, for every $m>N(\omega)$, 
\begin{align}
|X_t - X_s| \leq 2 \sum_{j=n+1}^{m}2^{-\gamma j} ~~\text{ for all $t,s \in D_m$, }0 < t-s < 2^{-N(\omega)} \label{eq2}
\end{align}
For $m=n+1$, we can only have $t = (kT/2^m), s = ((k-1)T/2^m)$ and the result follows from (b). Suppose (\ref{eq2}) is true for $m = n+1, \ldots, M-1$. We take $s<t$ with $s,t \in D_{M}$ and consider the numbers $t^1 = \max \{u \in D_{M-1}: u \leq t\}$ and $s^1=\min\{u \in D_{M-1}:u \geq s\}$ which gives the relationship $s \leq s^1\leq t^1 \leq t$ and  $s-s^1 \leq 2^{-M},  t-t^1\leq 2^{-M}$. Hence from (\ref{eq1}) we have 
\begin{align*}
|X_{s^1}-X_s| &\leq 2^{-\gamma M}\\
|X_{t^1}-X_t| &\leq 2^{-\gamma M}
\end{align*}
and from (\ref{eq2}) with $m=M-1$, 
\begin{align*}
|X_{t^1}-X_{s^1}| \leq 2 \sum_{j=n+1}^{M-1}2^{-\gamma j}
\end{align*}
so
\begin{align*}
|X_t-X_s|&\leq |X_t-X_{t^1}|+|X_{t^1}-X_{s^1}|+|X_{s^1}-X_s|\\
&\leq 2^{-\gamma M} +2 \sum_{j=n+1}^{M-1}2^{-\gamma j} + 2^{-\gamma M}=2 \sum_{j=n+1}^{M}2^{-\gamma j}
\end{align*}
which proves (\ref{eq2}) for $m=M$. Thus by induction, we have shown
\begin{align}
|X_{t}-X_{s}| \leq 2 \sum_{j=n+1}^{\infty}2^{-\gamma j},\quad 0<t-s<2^{-N(\omega)}\label{eq3}
\end{align}



To show that $X$ is uniformly continuous with H\"{o}lder exponent $\gamma$ on the dyadic rationals, for any $s, t \in D$, with $0< t-s<h(\omega) \overset{\Delta}{=}2^{-N(\omega)}$, we choose the $n \geq N(\omega)$ such that $2^{-(n+1)}\leq t-s<2^{-n}$. Using the result from (\ref{eq3}),
\begin{align}
|X_t-X_s| \leq 2 \sum_{j=n+1}^{\infty}2^{-\gamma j}\leq 2^{-\gamma(n+1)}\left(2 \sum_{j=0}^{\infty}2^{-\gamma j}\right) \leq \delta |t-s|^\gamma, \quad 0<t-s<2^{-N(\omega)} \label{eq4}
\end{align}
where $\delta = 2/(1-2^{-\gamma})$ and shows it is uniformly continuous.






\item Set $\widetilde{X}$ to be equal to $X$ on the dyadic rationals.  For $t\in [0,T] \backslash D$, we choose a sequence $\{s_n\}_{n=1}^{\infty} \subseteq D$ with $s_n \to t$, the uniform continuity and Cauchy criterion implies that $\{X_{s_n}(\omega)\}_{n=1}^{\infty}$ has a limit which depends on $t$ but not on the particular sequence $\{s_n\}_{n=1}^{\infty} \subseteq D$ chosen to converge to $t$ and we set $\widetilde{X}_t(\omega) = \lim_{s_n\to t}X_{s_n}(\omega)$. Thus $\widetilde{X}$ is continuous and satisfies (\ref{eq4}), since for $0<t-s<2^{-N(\omega)}$, for any $\epsilon>0$, where $t_n \to t, s_n \to s$, we can choose a sufficient large $n$ such that 
\begin{align*}
|\widetilde{X}_t(\omega) - X_{t_n}(\omega)| &< \epsilon/2\\
|X_{s_n}(\omega) - \widetilde{X}_s(\omega)|&< \epsilon/2\\
|X_{t_n}(\omega)- X_{s_n}(\omega)| &\leq \delta |t-s|^\gamma
\end{align*}
then we have
\begin{align*}
|\widetilde{X}_t(\omega) - \widetilde{X}_s(\omega)| & = |\widetilde{X}_t(\omega) - X_{t_n}(\omega) + X_{t_n}(\omega)- X_{s_n}(\omega) + X_{s_n}(\omega) - \widetilde{X}_s(\omega)|\\
& = |\widetilde{X}_t(\omega) - X_{t_n}(\omega)| + |X_{t_n}(\omega)- X_{s_n}(\omega)| + |X_{s_n}(\omega) - \widetilde{X}_s(\omega)| \leq \delta |t-s|^\gamma + \epsilon
\end{align*} 
and since $\epsilon$ is arbitrary, it shows
\begin{align*}
\mathbb{P}\left[\omega; \sup_{0<t-s<h(\omega)}\left|\widetilde{X}_t(\omega)-\widetilde{X}_s(\omega)\right|\leq \delta|t-s|^\gamma\right]=1
\end{align*}



To see that $\widetilde{X}$ is a modification of $X$, we observe that $\widetilde{X}_t = X_t$ almost surely for $t \in D$ and for $t \in [0,T]  \backslash D$ and $\{s_n\}_{n=1}^{\infty}\subseteq D$ with $s_n \to t$, we have $X_{s_n} \to X_t$ in probability as $X_{s_n} \to \widetilde{X}_t$ almost surely, so $\widetilde{X}_t = X_t$ almost surely.

\item
\begin{align*}
\frac{1}{\sqrt{2\pi}}\int_{-\infty}^{\infty}x^ne^{-\frac{x^2}{2}}\,dx&=\frac{1}{\sqrt{2\pi}}\left[-x^{n-1}e^{-\frac{x^2}{2}}\right]_{-\infty}^{\infty}-\frac{1}{\sqrt{2\pi}}\int_{-\infty}^{\infty}-(n-1)x^{n-2}e^{-\frac{x^2}{2}}\,dx\\
&=(n-1)\frac{1}{\sqrt{2\pi}}\int_{-\infty}^{\infty}x^{n-2}e^{-\frac{x^2}{2}}\,dx\\
&=(n-1)(n-3)\frac{1}{\sqrt{2\pi}}\int_{-\infty}^{\infty}x^{n-4}e^{-\frac{x^2}{2}}\,dx\\
\text{In general, } &=(n-1)(n-3)\ldots(n-(2k-1))\frac{1}{\sqrt{2\pi}}\int_{-\infty}^{\infty}x^{n-2k}e^{-\frac{x^2}{2}}\,dx
\end{align*}
Thus we have
\begin{align*}
\frac{1}{\sqrt{2\pi}}\int_{-\infty}^{\infty}x^ne^{-\frac{x^2}{2}}\,dx :=\begin{cases}
0, & \text{$n$ is odd}\\
(n-1)!! , & \text{$n$ is even}
\end{cases}
\end{align*}
where $n!!$ is the double factorial, the product of all numbers from 1 to $n$ that have the same parity as $n$.


To show that Brownian motion is almost surely locally $\alpha$-H\"{o}lder continuous for all $\alpha<1/2$, we start with a Brownian motion $\{W_t\}_{t\in[0,T]}$. Then $W_t-W_s$ and $\sqrt{t-s}\,W_1$ are both normally distributed with mean 0 and variance $t-s$, for $0 \leq s < t \leq T$ and for each $n=2,3,\ldots$
\begin{align*}
\mathbb{E}\left[|W_t-W_s|^{2n}\right] = |t-s|^n\mathbb{E}\left[|W_1|^{2n}\right]=C_n|t-s|^n
\end{align*}
where $C_n = \mathbb{E}\left[|W_1|^{2n}\right] = \frac{(2n)!}{2^n\,n!}<\infty$ from earlier result in (e). Thus by the main theorem proved above, we can find a continuous modification $\{\widetilde{W}_t\}_{t\in [0,T]}$ of $\{W_t\}_{t\in[0,T]}$ and also it shows that $\{\widetilde{W}_t\}_{t\in [0,T]}$ is almost surely continuous $\alpha$-H\"{o}lder continuous for all $0<\alpha<\frac{\beta}{\alpha} = \frac{n-1}{2n} = \frac{1}{2}-\frac{1}{2n}$ as desired. Thus Brownian motion is almost surely $\alpha$-H\"{o}lder continuous for all $\alpha < 1/2$.
%
%
%$\{W_t\}_{t\in [0,T]}$ is almost surely continuous $\alpha$-H\"{o}lder continuous for all $0<\alpha<\frac{\beta}{\alpha} = \frac{n-1}{2n} = \frac{1}{2}-\frac{1}{2n}$ as desired.
%
%we can find a continuous modification $\{\widetilde{W}_t\}_{t\in [0,T]}$ of $\{W_t\}_{t\in[0,T]}$ and also it shows that $\{\widetilde{W}_t\}_{t\in [0,T]}$ is almost surely continuous 

\end{enumerate}

\end{proof}

\end{document}