%\documentclass[handout]{beamer}
\documentclass[10pt,appendixnumberbeamer]{beamer}
%\usepackage{handoutWithNotes}
%\pgfpagesuselayout{4 on 1 with notes}[a4paper,border shrink=5mm]
%\pgfpagesuselayout{1 on 1 with notes landscape}[a4paper,border shrink=5mm]
\usetheme[
  outer/progressbar=foot,
%outer/numbering=fraction,
]{metropolis}

%\hypersetup{pdfpagemode=FullScreen}

\usepackage{amsmath, amssymb, amsthm, mathtools,media9}
\usepackage[export]{adjustbox}

\usepackage[default]{sourcesanspro}
\usepackage[utf8]{inputenc}
\usepackage[T1]{fontenc}
\metroset{titleformat=smallcaps}

\usepackage{tikz}
\usetikzlibrary{mindmap,shadows}
\usepackage{animate}
\usepackage[compatibility=false]{caption}

\usepackage{algorithm}
\usepackage{algorithmic}
\usepackage{bm}

\usepackage{listings}
\usepackage{tcolorbox}

\newcommand{\lstfont}[1]{\color{#1}\scriptsize\ttfamily}

\lstset { %
    language=[ANSI]C++,
    backgroundcolor=\color{black!5}, % set backgroundcolor
    basicstyle=\scriptsize,% basic font setting
    showstringspaces=false,
    breaklines=true,
}


\usepackage{graphicx,subfigure}
\usetikzlibrary{spy,calc,patterns,arrows,decorations.pathmorphing,backgrounds,positioning,fit,petri,mindmap,trees,intersections}
\usepackage{keystroke}
\usepackage{color}
\usepackage{multicol}
%\usepackage{multimedia}
\usepackage{enumerate}
\usepackage{multirow}
\usepackage{pgfplots,tikz,tcolorbox}
%\usepackage{appendixnumberbeamer}

%\setbeamertemplate{frame footer}{My custom footer}

\pgfplotsset{compat=newest}

%\expandafter\def\expandafter\insertshorttitle\expandafter{%
 % \insertshorttitle\hfill\insertframenumber\,/\,\inserttotalframenumber}
%\usepackage{enumitem}


%\usepackage[T1]{fontenc}
\usepackage{color,hyperref}
\definecolor{darkblue}{rgb}{0.0,0.0,0.5}
%\hypersetup{colorlinks,breaklinks,
%            linkcolor=darkblue,urlcolor=darkblue,
%            anchorcolor=darkblue,citecolor=darkblue}
\usetikzlibrary{patterns}
\usetikzlibrary{decorations.pathmorphing,backgrounds,positioning,fit,petri,mindmap,trees}
\usepgflibrary{arrows,shapes.geometric}
\usepgflibrary{shapes.symbols}
%\hypersetup{colorlinks,breaklinks,linkcolor=darkblue,urlcolor=darkblue,anchorcolor=darkblue,citecolor=darkblue}
\usetikzlibrary{spy,calc,patterns,arrows,decorations.pathmorphing,backgrounds,positioning,fit,petri,mindmap,trees,intersections}
\usepackage{color}
\usepackage{multirow}
\usetikzlibrary{arrows,positioning} 
\tikzset{
    %Define standard arrow tip
    >=stealth',
    %Define style for boxes
    punkt/.style={
           rectangle,
           rounded corners,
           draw=black, very thick,
           text width=8.5em,
           minimum height=2em,
           text centered},
    % Define arrow style
    pil/.style={
           ->,
           thick,
           shorten <=2pt,
           shorten >=2pt,}
}

%\usepackage[hang]{footmisc}
%\setlength\footnotemargin{10pt}

\setbeamertemplate{caption}[numbered]
\usepgfplotslibrary{statistics}

\usetikzlibrary{fadings}
\tikzfading[name=fade out,
  inner color=transparent!0, outer color=transparent!100]


%\setbeameroption{show notes}
%\setbeameroption{show only notes}


\newcommand{\network}[1]{
\begin{figure}[h]
\centering
\begin{tikzpicture}[scale = 0.75,-,draw=black!50, node distance=\layersep]
    \tikzstyle{neuron}=[circle,fill=black!25,minimum size=17pt,inner sep=0pt];
    \tikzstyle{unit}=[neuron, fill=red!50];
    \tikzstyle{spike}=[neuron, fill=blue!50];
 \def \radius {2cm}
% \def \margin {8}
 \def \n {6}
 \foreach \s in {1,...,\n}
  \node[unit] (\s) at ({360/\n * (\s - 1) - 180}:\radius) {$\s$};
 \foreach \s in {1,...,\n}
  \foreach \t in {\s,...,\n}
   \draw (\t) -- (\s);
 \foreach \s in {1,...,\n}
   \draw[very thick] (#1) -- (\s);
   \node[spike] (#1) at ({360/\n * (#1 - 1) - 180}:\radius) {};
\end{tikzpicture}
\end{figure}

}



\title{Equality of Norms in $\mathbb{R}^n$}
\subtitle{Real Analysis Presentation}
%\institute{}
\author{Hung, Zhangsheng}
%\date{\today}
\date{}


\pgfmathdeclarefunction{gauss}{2}{%
  \pgfmathparse{1/(#2*sqrt(2*pi))*exp(-((x-#1)^2)/(2*#2^2))}%
}
%\pgfmathdeclarefunction{gauss1}{2}{%
  %\pgfmathparse{(1/(#2*sqrt(2*pi))*exp(-((-x-#1)^2)/(2*#2^2)))*exp(-x)}%
%}
\pgfmathdeclarefunction{gauss2}{2}{%
  \pgfmathparse{(1/(#2*sqrt(2*pi))*exp(-0.5*abs(x)-#1)*(sqrt(abs(x)))}%
}

%\pgfmathdeclarefunction{gauss2}{2}{%
  %\pgfmathparse{(1/(#2*sqrt(2*pi))*exp(-((x-#1)^2)/(2*#2^2)))*exp(x)}%
%}
\pgfmathdeclarefunction{gauss1}{2}{%
  \pgfmathparse{(1/(#2*sqrt(2*pi))*exp(-0.5*x-#1)*(sqrt(x))}%
}
\def\checkmark{\tikz\fill[scale=0.4](0,.35) -- (.25,0) -- (1,.7) -- (.25,.15) -- cycle;}

\newcommand*{\Perm}[2]{{}^{#1}\!P_{#2}}%
\newcommand*{\Comb}[2]{{}^{#1}C_{#2}}%
\begin{document}


\begin{frame}
\titlepage
\end{frame}


%\begin{frame}{Contents}
%\tableofcontents
%\end{frame}


\begin{frame}{Preliminaries}
\begin{definition}
A norm on $\mathbb{R}^n$ is a function $\|\cdot\|:\mathbb{R}^n \to \mathbb{R}$ with the following properties:\\
For all $x, y \in \mathbb{R}^n$ and $a \in \mathbb{R}$
\begin{enumerate}[1.]
\item $\|x\| \geq 0, \|x\| = 0$ iff $x=\mathbf{0}$
\item $\|ax\| = |a| \|x\|$
\item $\|x+y\| \leq \|x\|+\|y\|$
\end{enumerate}
\end{definition}
\end{frame}

\begin{frame}{Preliminaries}
\begin{block}{1-norm}
For $x \in \mathbb{R}^n$,
\begin{align*}
\|x\|_1 = \sum_{i=1}^{n} |x_i|
\end{align*}
It is easy to see that it satisfy properties 1 and 2, so we shall show the triangle of inequality property
\begin{align*}
\|x+y\|_1 =  \sum_{i=1}^{n} |x_i+y_i| \leq \sum_{i=1}^{n} |x_i|+|y_i| = \sum_{i=1}^{n} |x_i|+ \sum_{i=1}^{n} |y_i| = \|x\|_1+\|y\|_1
\end{align*}
\end{block}
\end{frame}

\begin{frame}{Preliminaries}
\begin{definition}
Two norms $\|\cdot\|_a$ and $\|\cdot\|_b$ on $\mathbb{R}^n$ are equivalent if there exists two constants $c_1, c_2 >0$ such that
\begin{align*}
c_1\|x\|_b \leq \|x\|_a \leq c_2\|x\|_b
\end{align*} 
\end{definition}
\end{frame}

\begin{frame}{Preliminaries}
\begin{theorem}[Extreme Value Theorem]
Let $f : X \to \mathbb{R}$ be continuous. If X is compact\footnote{closed and bounded by Heine-Borel theorem when in $X=\mathbb{R}^n$}, then there exists points $c$ and $d$ such that 
\begin{align*}
f(c) \leq f(x) \leq f(d)
\end{align*}
 for all $x \in X$.
\end{theorem}
\end{frame}


%\begin{frame}{Preliminaries}
%\begin{theorem}[Extreme Value Theorem]
%If a real-valued function $f(x)$ is continuous in the closed and bounded interval $[a,b]$, then f must attain a maximum and a minimum, each at least once. That is, there exist numbers $c$ and $d$ in $[a,b]$ such that:
%\begin{align*}
%f(d) \leq f(x) \leq f(c)
%\end{align*}
%for all $x\in [a,b]$.
%\end{theorem}
%
%\end{frame}


%\begin{frame}{Preliminaries}
%\begin{theorem}[Generalised Extreme Value Theorem]
%If $K$ is a compact\footnote{closed and bounded in $\mathbb{R}^n$ by Heine-Borel theorem} set and $f:K\to \mathbb{R}$ is a continuous function, then $f$ is bounded and there exist $p,q\in K$ such that $f(p)=\sup_{x\in K}f(x)$ and $f(q)=\inf_{x\in K}f(x)$.
%\end{theorem}
%\end{frame}

\begin{frame}{Idea of Proof}
\begin{block}{Steps}
\begin{enumerate}[1.]
\item Transitivity property of norm equivalence
\item Consideration of only $\|x\|_1 = 1$
\item Continuity of any norm $\|\cdot\|_a$ under the $1-$norm
\item Maximum and minimum of $\|\cdot\|_a$ on the $1-$norm unit sphere
\end{enumerate}
\end{block}
\end{frame}


\begin{frame}{Transitivity property of norm equivalence}
Suppose we have two norms $\|\cdot\|_a, \|\cdot\|_{a'}$ that are equivalent to constants $c_1, c_2 >0$ and $c'_1, c'_2 >0$ respectively:
\begin{align*}
c_1 \|x\|_1\leq &\|x\|_a \leq c_2\|x\|_1\\
c'_1 \|x\|_1\leq &\|x\|_{a'} \leq c'_2\|x\|_1
\end{align*}
then
\begin{align*}
\frac{c'_1}{c_2} \|x\|_a \leq c'_1 \|x\|_1\leq &\|x\|_{a'} \leq c'_2\|x\|_1 \leq  \frac{c'_2}{c_1}\|x\|_a
\end{align*}
which shows that if every norm in $\mathbb{R}^n$ is equivalent to the 1-norm, it implies that all the norms in $\mathbb{R}^n$ are equivalent.

\end{frame}

\begin{frame}{Transitivity property of norm equivalence}

\end{frame}

\begin{frame}{Consideration of only $\|x\|_1 = 1$}
We want to show that for every norm $\|\cdot\|_a$ in $\mathbb{R}^n$,
\begin{align*}
c_1\|x\|_1 \leq \|x\|_a \leq c_2\|x\|_1
\end{align*}
this is trivially true for $x=0$ and thus for $x \neq 0$, we can normalize $x$ such that it is unit length in the 1-norm, by scaling $x$ be a multiple of $1/\|x\|_1$, i.e.
\begin{align}
c_1 = c_1 \|u\|_1\leq \|u\|_a\leq c_2 \|u\|_1 = c_2 \label{eq1}
\end{align}
where $u = x/ \|x\|_1$. This means if we can show that (\ref{eq1}) holds for the set $\{x \in \mathbb{R}^n|\|x\|_1=1\}$, it is same as showing for all $x \in \mathbb{R}^n$.
\end{frame}

\begin{frame}{Consideration of only $\|x\|_1 = 1$}

\end{frame}


\begin{frame}{Continuity of any norm $\|\cdot\|_a$ under the $1-$norm}
Showing that every norm $\|\cdot\|_a$ is continuous under the 1- norm allows us to apply the generalised extreme value theorem from earlier. 

{\color{red}Leave to Hung from here}

\end{frame}

\begin{frame}{Continuity of any norm $\|\cdot\|_a$ under the $1-$norm}

\end{frame}


\begin{frame}{Maximum and minimum of $\|\cdot\|_a$ on the $1-$norm unit sphere}

\end{frame}

\begin{frame}{Maximum and minimum of $\|\cdot\|_a$ on the $1-$norm unit sphere}
Here we proof the claim that $S:=\{x \in \mathbb{R}^n| \|x\|_1=1\}$ is a compact set.
\begin{proof}
Let $y$ be any limit point of $S$, then given any $\delta >0$, we have an open ball of radius $\delta$ centered at $y$, denoted by $B_\delta(y)$\footnote{$B_\delta(y):=\{x \mid \|y-x\|_1 < \delta\}$} such that $B_\delta(y) \cap S \neq \varnothing$. Pick $z \in B_\delta(y) \cap S$, then 
\begin{align*}
\delta>\|y-z\|_1 > |\|y\|_1-\|z\|_1| = |\|y\|_1-1|
\end{align*}
since $\delta$ is arbitrary, we have $\|y\|_1=1$ and thus we have shown $S$ is closed since its limit points are all in $S$. $S$ is bounded by construction of the set. 

\end{proof}
\end{frame}


\end{document}

