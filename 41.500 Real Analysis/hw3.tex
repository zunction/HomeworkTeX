\documentclass[a4paper,12pt]{article}
\setlength{\parindent}{0cm}
\usepackage{amsmath, amssymb, amsthm, mathtools,pgfplots,bbm}
\usepackage{graphicx,caption}
\usepackage{verbatim}
\usepackage{venndiagram}
\usepackage[cm]{fullpage}
\usepackage{fancyhdr}
\usepackage{tikz}
\usepackage{listings}
\usepackage{color,enumerate,framed}
\usepackage{color,hyperref}
\definecolor{darkblue}{rgb}{0.0,0.0,0.5}
\hypersetup{colorlinks,breaklinks,
            linkcolor=darkblue,urlcolor=darkblue,
            anchorcolor=darkblue,citecolor=darkblue}

%\usepackage{tgadventor}
%\usepackage[nohug]{diagrams}
\usepackage[T1]{fontenc}
%\usepackage{helvet}
%\renewcommand{\familydefault}{\sfdefault}
%\usepackage{parskip}
%\usepackage{picins} %for \parpic.
%\newtheorem*{notation}{Notation}
%\newtheorem{example}{Example}[section]
%\newtheorem*{problem}{Problem}
\theoremstyle{definition}
%\newtheorem{theorem}{Theorem}
%\newtheorem*{solution}{Solution}
%\newtheorem*{definition}{Definition}
%\newtheorem{lemma}[theorem]{Lemma}
%\newtheorem{corollary}[theorem]{Corollary}
%\newtheorem{proposition}[theorem]{Proposition}
%\newtheorem*{remark}{Remark}
%\setcounter{section}{1}

\newtheorem{thm}{Theorem}[section]
\newtheorem{lemma}[thm]{Lemma}
\newtheorem{prop}[thm]{Proposition}
\newtheorem{cor}[thm]{Corollary}
\newtheorem{defn}[thm]{Definition}
\newtheorem*{examp}{Example}
\newtheorem{conj}[thm]{Conjecture}
\newtheorem{rmk}[thm]{Remark}
\newtheorem*{nte}{Note}
\newtheorem*{notat}{Notation}

%\diagramstyle[labelstyle=\scriptstyle]

\lstset{frame=tb,
  language=Oz,
  aboveskip=3mm,
  belowskip=3mm,
  showstringspaces=false,
  columns=flexible,
  basicstyle={\small\ttfamily},
  breaklines=true,
  breakatwhitespace=true,
  tabsize=3
}


\pagestyle{fancy}




\fancyhead{}
\renewcommand{\headrulewidth}{0pt}

\lfoot{\color{black!60}{\sffamily Zhangsheng Lai}}
\cfoot{\color{black!60}{\sffamily Last modified: \today}}
\rfoot{\textsc{\thepage}}



\begin{document}
\flushright{Zhangsheng Lai\\1002554}
\section*{Real Analysis: Homework 3}

\begin{enumerate}
\item
A function which is in $\mathcal{C}^1(\mathbb{R})$ but not in $\mathcal{C}^2(\mathbb{R})$ means a function that has continuous first derivative but its second derivative is not continuous. Consider $f: \mathbb{R} \to \mathbb{R}$,
\begin{align*}
f(x):=\begin{cases}
x^2 & \text{ if }x \geq 0\\
0 & \text{ if }x <0
\end{cases}\qquad
f'(x):=\begin{cases}
2x & \text{ if }x \geq 0\\
0 & \text{ if }x <0
\end{cases}\qquad
f''(x):=\begin{cases}
2 & \text{ if }x \geq 0\\
0 & \text{ if }x <0
\end{cases}
\end{align*}
we see that the first derivate of $f(x)$ is continuous but the second derivate is not continuous at $x =0$.



\item By Stone-Weierstrass Theorem, for the given function $f(x)$, there exists a sequence of polynomials $P_n(x)$ that converges uniformly to $f(x)$, i.e. $\sup_{x \in [0,1]}|f(x)-P_n(x)| <1/n$ as $n \to 0$. Now, we consider the inner product $\langle f(x), g(x)\rangle:= \int_{0}^{1}f(x)g(x)\,dx$, and if we managed to show that $\langle f(x),f(x)\rangle=0$, we are done. To show that, 
\begin{align*}
\left|\int_{0}^{1}f(x)^2\,dx\right| &= \left|\int_{0}^{1}f(x)^2\,dx-\int_{0}^{1}f(x)P_n(x)\,dx\right| ,\text{ since $\int_{0}^{1}f(x)x^n\,dx=0$ for all $n$.}\\
&\leq \int_{0}^{1} \left|f(x)\right|\left| f(x)-P_n(x)\right|\,dx \\
&\leq \int_{0}^{1} \left|f(x)\right|\,dx \cdot \sup_{x\in [0,1]}|f(x)-P_n(x)|\\
&\leq \frac{1}{n}\int_{0}^{1} \left|f(x)\right|\,dx \text{ for all $n$.}
\end{align*}
Thus we need $\langle f(x), f(x) \rangle=0$, which completes the proof.



%Given $f:[0,1] \to \mathbb{R}$ continuous with $\int_{0}^{1}f(x)x^n\,dx=0$ for all $n \in \mathbb{Z}_{\geq 0}$. By MVT for integrals, for every $n \in \mathbb{Z}_{\geq 0}$ there exists $c_n \in (0,1)$ such that
%\begin{align*}
%0 = \int_{0}^{1}f(x)x^n\,dx = f(c_n) \int_{0}^{1}x^n\,dx = \frac{f(c_n)}{n+1}
%\end{align*}



\item Let $\phi$ be $\lambda$-H\"{o}lder bi-continuous then for $v_1, v_2, u_1, u_2 \in T$, we have
\begin{align*}
\sup_{v \in [0,T]}|\phi(u_2,v)-\phi(u_1,v)| &\leq C_u|u_2-u_1|^\lambda\\
\sup_{u \in [0,T]}|\phi(u,v_2)-\phi(u,v_1)| &\leq C_v|v_2-v_1|^\lambda
\end{align*}
then we also observe that 
\begin{align*}
|\phi(u_1,v_1)-\phi(u_1,v_2)-\phi(u_2,v_1)+\phi(u_2,v_2)| &\leq |\phi(u_1,v_1)-\phi(u_1,v_2)|+|\phi(u_2,v_2)-\phi(u_2,v_1)| \\
|\phi(u_1,v_1)-\phi(u_1,v_2)-\phi(u_2,v_1)+\phi(u_2,v_2)| &\leq |\phi(u_1,v_1)-\phi(u_2,v_1)|+|\phi(u_2,v_2)-\phi(u_1,v_2)|
\end{align*}
which gives us 
\begin{align*}
|\phi(u_1,v_1)-\phi(u_1,v_2)-\phi(u_2,v_1)+\phi(u_2,v_2)| &\leq 2C_v|v_2-v_1|^\lambda \\
|\phi(u_1,v_1)-\phi(u_1,v_2)-\phi(u_2,v_1)+\phi(u_2,v_2)| &\leq 2C_u|u_2-u_1|^\lambda
\end{align*}
multiplying them together, we have 
\begin{align*}
|\phi(u_1,v_1)-\phi(u_1,v_2)-\phi(u_2,v_1)+\phi(u_2,v_2)|^2 \leq 4C_vC_u|v_2-v_1|^\lambda |u_2-u_1|^\lambda
\end{align*}
squaring both sides, we have shown that all $\lambda$-H\"{o}lder bi-continuous are strongly $\lambda/2$-H\"{o}lder bi-continuous.



\item
\begin{enumerate}[(a)]
\item For $0 \leq a < b \leq T$ with $0 \leq \alpha < 1/4$ and $b \leq r_1 \leq T$, we first make the following observations:
\begin{align}
(r_1-b)^{3/4} \leq (r_1-a)^{3/4}\\
\frac{1}{(r_1-a)^{\alpha+1/4}} \leq \frac{1}{(b-a)^{\alpha+1/4}}
\end{align}
then
\begin{align*}
\int_{b}^{T}\frac{1}{(r_1-b)^\alpha(r_1-a)^{\alpha+1}}\,dr_1 &= \int_{b}^{T}\frac{(r_1-b)^{-\alpha-3/4+3/4}}{(r_1-a)^{\alpha+1}}\,dr_1\\
&\leq \int_{b}^{T}\frac{(r_1-b)^{-\alpha-3/4}(r_1-a)^{3/4}}{(r_1-a)^{\alpha+1}}\,dr_1\\
&= \int_{b}^{T}\frac{(r_1-b)^{-\alpha-3/4}}{(r_1-a)^{\alpha+1/4}}\,dr_1\\
&\leq \frac{1}{(b-a)^{\alpha+1/4}}\int_{b}^{T}(r_1-b)^{-\alpha-3/4}\,dr_1\\
&= \frac{1}{(b-a)^{\alpha+1/4}}\left[\frac{(r_1-b)^{1/4-\alpha}}{1/4-\alpha}\right]_{b}^{T}\\
&= C\frac{(T-b)^{1/4-\alpha}}{(b-a)^{\alpha+1/4}} \text{ for some constant $C$.}
\end{align*}

%WTS
%\begin{align*}
%\int_{b}^{T}\frac{1}{(r_1-b)^\alpha(r_1-a)^{\alpha+1}}\,dr_1 \leq \frac{(T-b)^{1/4-\alpha}}{(b-a)^{\alpha+1/4}}
%\end{align*}

\item 
Given $\psi(u,v):= \mathbbm{1}_{[0,v)}(u)\tilde{\psi}(u,v)$ we can understand it as
\begin{align*}
\psi(u,v):=\begin{cases}
\tilde{\psi}(u,v) & u <v\\
0 & u \geq v
\end{cases}
\end{align*}

then to do the double integral of $f(u,v)^2$ over $R = [0,T] \times [0,T]$ is equivalent to integrating over the region $\{(u,v) \in R\mid u < v\}$. Thus 

\begin{align*}
\int_{0}^{T}\int_{0}^{T}f(u,v)^2\,du\,dv = \int_{0}^{T} \int_{0}^{v}\left|\int_{u}^{T}\int_{v}^{T}\frac{\tilde{\psi}(u,v)-\tilde{\psi}(u,r_2)-\tilde{\psi}(r_1,v)+\tilde{\psi}(r_1,r_2)}{(r_1-u)^{1+\alpha}(r_2-v)^{1+\alpha}}dr_2\,dr_1\right|^2du\,dv
%=\int_{0}^{T}\int_{0}^{T}\left[ \int_{u}^{T}\int_{v}^{T}\frac{\psi(u,v)-\psi(u,r_2)-\psi(r_1,v)+\psi(r_1,r_2)}{(r_1-u)^{1+\alpha}(r_2-v)^{1+\alpha}}\,dr_2dr_1\right]^2 \,du\,dv
\end{align*}
We will now massage the term inside the first double integral,
\begin{align*}
&\left|\int_{u}^{T}\int_{v}^{T}\frac{\tilde{\psi}(u,v)-\tilde{\psi}(u,r_2)-\tilde{\psi}(r_1,v)+\tilde{\psi}(r_1,r_2)}{(r_1-u)^{1+\alpha}(r_2-v)^{1+\alpha}}dr_2\,dr_1\right| \\
\leq &\int_{u}^{T}\int_{v}^{T}\left|\frac{\tilde{\psi}(u,v)-\tilde{\psi}(u,r_2)-\tilde{\psi}(r_1,v)+\tilde{\psi}(r_1,r_2)}{(r_1-u)^{1+\alpha}(r_2-v)^{1+\alpha}}\right|dr_2\,dr_1\\
\leq &\int_{u}^{T}\int_{v}^{T}\frac{\left|\tilde{\psi}(u,v)-\tilde{\psi}(u,r_2)-\tilde{\psi}(r_1,v)+\tilde{\psi}(r_1,r_2)\right|}{\left|(r_1-u)^{1+\alpha}(r_2-v)^{1+\alpha}\right|}dr_2\,dr_1\\
\leq &\int_{u}^{T}\int_{v}^{T}\frac{\left|(r_1-u)^\lambda(r_2-v)^\lambda\right|}{\left|(r_1-u)^{1+\alpha}(r_2-v)^{1+\alpha}\right|}dr_2\,dr_1, \quad \text{ by strongly $\lambda$-H\"{o}lder bi-continuous property}
\end{align*}
but I'm stuck.


%\begin{align*}
%|f(u,v)| &\leq \int_{u}^{T}\int_{v}^{T}\left|\frac{\psi(u,v)-\psi(u,r_2)-\psi(r_1,v)+\psi(r_1,r_2)}{(r_1-u)^{1+\alpha}(r_2-v)^{1+\alpha}}\right|\,dr_2dr_1\\
%&\leq \int_{u}^{T}\int_{v}^{T}\left|C\frac{||^\lambda||^\lambda}{(r_1-u)^{1+\alpha}(r_2-v)^{1+\alpha}}\right|\,dr_2dr_1\\
%\end{align*}
\end{enumerate}


\end{enumerate}












\end{document}