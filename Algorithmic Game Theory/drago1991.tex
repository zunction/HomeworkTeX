\documentclass[a4paper,10pt,leqno]{article}
\setlength{\parindent}{0cm}
\usepackage{amsmath, amssymb, amsthm, mathtools,pgfplots}
\usepackage{graphicx,caption}
\usepackage{verbatim}
\usepackage{venndiagram}
\usepackage[cm]{fullpage}
\usepackage{fancyhdr}
\usepackage{tikz}
\usepackage{listings,url,}
\usepackage{color,enumerate,framed}
\usepackage{color,hyperref}
\definecolor{darkblue}{rgb}{0.0,0.0,0.5}
\hypersetup{colorlinks,breaklinks,
            linkcolor=darkblue,urlcolor=darkblue,
            anchorcolor=darkblue,citecolor=darkblue}

\usepackage{sectsty}
\allsectionsfont{\centering}
%\usepackage[normalem]{ulem}
%\allsectionsfont{\sffamily}
%\sectionfont{\centering\ulemheading{\uuline}}

%\usepackage{tgadventor}
%\usepackage[nohug]{diagrams}
\usepackage[T1]{fontenc}
%\usepackage{helvet}
%\renewcommand{\familydefault}{\sfdefault}
\usepackage{parskip}
%\usepackage{picins} %for \parpic.
%\newtheorem*{notation}{Notation}
%\newtheorem{example}{Example}[section]
%\newtheorem*{problem}{Problem}
\theoremstyle{definition}
%\newtheorem{theorem}{Theorem}
%\newtheorem*{solution}{Solution}
%\newtheorem*{definition}{Definition}
%\newtheorem{lemma}[theorem]{Lemma}
%\newtheorem{corollary}[theorem]{Corollary}
%\newtheorem{proposition}[theorem]{Proposition}
%\newtheorem*{remark}{Remark}
%\setcounter{section}{1}

\newtheorem{thm}{Theorem}[section]
\newtheorem{lemma}[thm]{Lemma}
\newtheorem{prop}[thm]{Proposition}
\newtheorem{cor}[thm]{Corollary}
\newtheorem{defn}[thm]{Definition}
\newtheorem*{examp}{Example}
\newtheorem{conj}[thm]{Conjecture}
\newtheorem{rmk}[thm]{Remark}
\newtheorem*{nte}{Note}
\newtheorem*{notat}{Notation}

%\diagramstyle[labelstyle=\scriptstyle]

\lstset{frame=tb,
  language=Oz,
  aboveskip=3mm,
  belowskip=3mm,
  showstringspaces=false,
  columns=flexible,
  basicstyle={\small\ttfamily},
  breaklines=true,
  breakatwhitespace=true,
  tabsize=3
}


\pagestyle{fancy}




\fancyhead{}
\renewcommand{\headrulewidth}{0pt}

\lfoot{\color{black!60}{\sffamily Hung \& Zhangsheng}}
\cfoot{}
\cfoot{\color{black!60}{\sffamily Last modified: \today}}
\rfoot{\color{black!60}{\sffamily\thepage}}



\begin{document}
\begin{flushright}
Nguyen Tan Thai Hung\\
Zhangsheng Lai\\
\end{flushright}
\section*{Review: Competition and cooperation in the
workplace}
\subsection*{Summary}

In this paper the authors analyse the workplace dynamics of cooperation and competition that exists between two risk averse workers with similar attributes. A worker's utility is a function of the worker's effort, $\mu$ and the worker's income $y$, presented as an expectation $\mathbf{E}[U(y)-V(\mu)]$.
%\begin{align*}
%\mathbf{E}[U(y)-V(\mu)]
%\end{align*}
$U$ and $V$ are utility functions that convert income and effort respectively into a common measure. A worker's effort is represented by $\mu_i = \mu_{ii}+\mu_{ij}$, made up of his or her own effort $\mu_{ii}$ and a helping effort $\mu_{ij}$, which denotes effort in helping agent $j$. The paper does not clearly define each worker's task or their task as a whole but uses a more generic measure of the worker's output represented by a production function $Q_i=f^i(\mu_{ii},\mu_{ji})+\epsilon_i$,
%\begin{align*}
%Q_i=f^i(\mu_{ii},\mu_{ji})+\epsilon_i
%\end{align*}
which is a function of own effort and helping effort from other agents and the $\epsilon_i$ is some random disturbance to the production. The paper requires knowledge of the distribution of $\epsilon$ to compute the average utility received by a worker. The random disturbance is assumed to be uncorrelated to each other as it imposes $\mathbf{E}[\epsilon_i\epsilon_j]=0$ and $\mathbf{E}[\epsilon_i]=0$. Instead of proposing a production function explicitly, the paper characterizes $f$ by imposes three restrictions on it; we would potentially like to adopt the following two of them into our discussion: 


%In order to evaluate expectation, we need some probability distribution and here, the distribution of $\epsilon$, is the only one that the paper works with in its analysis and the random noise of each worker is (independent? I think not) and identically distributed. 

\begin{enumerate}[(i)]
\item $f^i(0,\mu_{ji})=0$ which implies $\frac{\partial f^i(0,\mu_{ji})}{\partial \mu_{ji}}=0$
\item For $\mu_{ii}>0$,
\begin{align*}
\left.\frac{\partial f^i}{\partial \mu_{ii}}\right|_{\mu_{ij}=0} < \left.\frac{\partial f^i}{\partial \mu_{ij}}\right|_{\mu_{ij}=0}
\end{align*}
\end{enumerate}
The first is a natural property aligned to real scenarios in workplaces; the amount of work produced by an agent is deemed as zero if the agent's effort is zero, even if there are positive helping efforts from the other agents. The second is a design to induce helping efforts $-$ by making the production per unit of effort higher for helping effort than own effort whenever an agent directs all effort to ownself.

With the desired properties of the components of the utility function established,  the papers considers different scenarios based on the different behaviour of the workers in the sense of their classifying them based on these partial derivatives:
\begin{align*}
\frac{\partial\mu_{ii}}{\partial\mu_{jj}}=\frac{\partial\mu_{jj}}{\partial\mu_{ii}}=a\quad
\frac{\partial\mu_{ij}}{\partial\mu_{ji}}=\frac{\partial\mu_{ji}}{\partial\mu_{ij}}=b
\end{align*}
(i) A \emph{Cournot} behaviour is one where $(a,b)=(0,0)$, the workers efforts are not related to each other in any way. (ii) Partial bargaining behaviour is when $(a,b)=(0,1)$; the amount of helping effort exchanged between the workers are the same but own effort is independent of others independent effort. (iii) Complete bargaining occurs when $(a,b)=(1,1)$.


The next consideration of the paper is the incentive scheme: tournament or quota, which sets up a competitive and non-competitive environment in the workplace respectively. In the contest scenario, the workers are ranked and afforded prizes accordingly; in their two workers scenario they the higher ranked is assigned a prize of $+x$ and the lower is given a prize of $-x$. In the quota scheme, workers are given the prizes when they fulfil an expectation; thus everyone can get the prize $(+x)$ and similarly fail to get the prize $(-x)$. The conclusion derived from this paper is that we cannot expect much helping efforts in a competitive environment 

As every worker aims to maximise their utility that is the expectation introduced earlier, using $\overline{Y}$ as a fixed income-base the expectation is given by $\mathbb{P}_i(\mathbf{\mu})U(\overline{Y}+x)+(1-\mathbb{P}_i(\mathbf{\mu}))U(\overline{Y}-x)-V(\mu_i)$
%\begin{align*}
%\mathbb{P}_i(\mathbf{\mu})U(\overline{Y}+x)+(1-\mathbb{P}_i(\mathbf{\mu}))U(\overline{Y}-x)-V(\mu_i)
%\end{align*}
and Karush-Kuhn-Tucker (KKT) approached is used to find the optimal utility of the player. Conditioning on the sufficient conditions for KKT, the paper then derives the type of behaviours to be exhibited by the workers when they individually try to maximise their utility on both the contest and quota schemes.

The last discussion in the paper compares the difference between the tournament and quota schemes using the term welfare, which then requires some knowledge of the density of $\epsilon$; the paper examines the cases when $\epsilon$ is uniform and the case when it is unimodal symmetric.

TLDR: The paper gives a rigorous explanation of how the compensation scheme of a company plays a part in the helping efforts of its workers. 
 

%\cite{drago1991}

\subsection*{Positivity}
- working with only the distribution of $\epsilon$ makes the model much more neat; if there are too many distributions to manage, the analysis will be complex $-$ too much integrals that might not be tractable.


- subdividing into the contest and quota schemes tells us something important: the way the workers are graded plays an important role in whether the dynamic of the workers lean towards cooperation or competition. This is also seen in schools, where using a bell curve to grade students lead to very individualistic behaviour as compared to assigning grades based on the student's raw score.

-  



\subsection*{Negativity}

-  using KKT conditions to reverse-engineer the worker's behaviour that secures its best interest is an interesting approach but KKT does not guarantee that the workers will display such behaviours; it only guarantee that if the workers display a particular behaviour it will guarantee to maximise their utility. The paper argues in the reverse way and obtains results that are intuitive, however, it is important to note that it is the dynamics of the workers that give the utility, not the opposite which is what the paper is doing.

- the discussion only considers the scenarios where the workers' behaviours are the same, there might be scenarios where workers exhibit different behaviour from each other.

- the definition of the partial and complete bargaining behaviours is too constrained as we set the values to either 0 or 1 in the partial derivatives. This constrain is the reason for the result of propositions 2 and 3, where helping efforts do not occur even in a partial or complete bargaining tournament.  It seems to be this constrain is preventing  helping efforts although the imposing of the restriction on $f$ is to promote helping efforts.






\subsection*{\sffamily My Thoughts}
- the discussions is strictly on two workers and the paper does little to no discussion on extending it to a general $n$ workers. Will KKT conditions argument still hold if it is being extended to $n$ workers?


- in pg 355, the player's maximand will be (7) with (16) with the given $(\overline{Y},x,Q^0)$. Why is the utility function $U$ missing? 

- in pg 357 the argument for equation (26) says that the $g(0)$ reflects the restriction $Q^0=\mathbf{E}[Q_i]$; is this obtained as $f^i(\mu_{ii},0)=\mathbf{E}[Q_i]$?

- in pg 357 why (26) $\to$ (27) when base incomes are identical? The argument used is the the efforts and probability of winning is identical, so the employer sees no need to vary the base income of quote or contest scheme?

%\cite{drago1991}
\bibliography{drago1991} 
\bibliographystyle{apalike}













\end{document}