\documentclass[a4paper,12pt]{article}
\setlength{\parindent}{0cm}
\usepackage{amsmath, amssymb, amsthm, mathtools,pgfplots}
\usepackage{graphicx,caption}
\usepackage{verbatim}
\usepackage{venndiagram}
\usepackage[cm]{fullpage}
\usepackage{fancyhdr}
\usepackage{tikz,multirow}
\usepackage{listings}
\usepackage{color,enumerate,framed}
\usepackage{color,hyperref}
\definecolor{darkblue}{rgb}{0.0,0.0,0.5}
\hypersetup{colorlinks,breaklinks,
            linkcolor=darkblue,urlcolor=darkblue,
            anchorcolor=darkblue,citecolor=darkblue}

%\usepackage{tgadventor}
%\usepackage[nohug]{diagrams}
\usepackage[T1]{fontenc}
%\usepackage{helvet}
%\renewcommand{\familydefault}{\sfdefault}
%\usepackage{parskip}
%\usepackage{picins} %for \parpic.
%\newtheorem*{notation}{Notation}
%\newtheorem{example}{Example}[section]
%\newtheorem*{problem}{Problem}
\theoremstyle{definition}
%\newtheorem{theorem}{Theorem}
%\newtheorem*{solution}{Solution}
%\newtheorem*{definition}{Definition}
%\newtheorem{lemma}[theorem]{Lemma}
%\newtheorem{corollary}[theorem]{Corollary}
%\newtheorem{proposition}[theorem]{Proposition}
%\newtheorem*{remark}{Remark}
%\setcounter{section}{1}

\newtheorem{thm}{Theorem}[section]
\newtheorem{lemma}[thm]{Lemma}
\newtheorem{prop}[thm]{Proposition}
\newtheorem{cor}[thm]{Corollary}
%\newtheorem{defn}[thm]{Definition}
\newtheorem*{defn}{Definition}
\newtheorem*{examp}{Example}
\newtheorem{conj}[thm]{Conjecture}
\newtheorem{rmk}[thm]{Remark}
\newtheorem*{nte}{Note}
\newtheorem*{notat}{Notation}
%\pgfplotset{compat=1.14}
%\diagramstyle[labelstyle=\scriptstyle]

\lstset{frame=tb,
  language=Oz,
  aboveskip=3mm,
  belowskip=3mm,
  showstringspaces=false,
  columns=flexible,
  basicstyle={\small\ttfamily},
  breaklines=true,
  breakatwhitespace=true,
  tabsize=3
}

\usepackage{floatrow}
% Table float box with bottom caption, box width adjusted to content
\newfloatcommand{capbtabbox}{table}[][\FBwidth]

\pagestyle{fancy}




\fancyhead{}
\renewcommand{\headrulewidth}{0pt}

\lfoot{}
\cfoot{}

%\lfoot{\color{black!60}{\sffamily Zhangsheng Lai}}
%\cfoot{\color{black!60}{\sffamily Last modified: \today}}
\rfoot{\textsc{\thepage}}



\begin{document}
\flushright{Nguyen Tan Thai Hung\quad 1001986\\Zhangsheng Lai\quad1002554}
\section*{Algorithmic Game Theory: HW 2}

\begin{enumerate}

\item


\item
\begin{enumerate}[(a)]
\item 
Consider the utility maximizing game below starting with the the initial outcome $(A_1,B_1)$, from which best-response dynamics cycles forever, avoiding the pure Nash of $(A_3,B_2)$.
\begin{figure}[h]
\renewcommand{\arraystretch}{1.5}
    \centering
    \begin{tabular}{cc|c|c|c|}
      & \multicolumn{1}{c}{} & \multicolumn{3}{c}{P$1$}\\
      & \multicolumn{1}{c}{} & \multicolumn{1}{c}{$A_1$}  & \multicolumn{1}{c}{$A_2$} &\multicolumn{1}{c}{$A_3$}\\\cline{3-5}
      \multirow{3}*{P$2$}  & $B_1$ & $4,1$ & $1,2$& $0,0$\\\cline{3-5}
      \cline{3-5}
& $B_2$ & $0,0$ & $0,0$& $5,5$\\\cline{3-5}
      & $B_3$ & $3,3$ & $3,2$ &$0,0$\\\cline{3-5}
    \end{tabular}
 \end{figure}   
 \item Consider the game below
 
 \begin{figure}[h]
\renewcommand{\arraystretch}{1.5}
    \centering
    \begin{tabular}{cc|c|c|c|}
      & \multicolumn{1}{c}{} & \multicolumn{3}{c}{P$1$}\\
      & \multicolumn{1}{c}{} & \multicolumn{1}{c}{$A_1$}  & \multicolumn{1}{c}{$A_2$} &\multicolumn{1}{c}{$A_3$}\\\cline{3-5}
      \multirow{3}*{P$2$}  & $B_1$ & $0,0$ & $0,0$& $5,5$\\\cline{3-5}
      \cline{3-5}
& $B_2$ & $0,0$ & $0,0$& $5,5$\\\cline{3-5}
      & $B_3$ & $5,5$ & $5,5$ &$5,5$\\\cline{3-5}
    \end{tabular}
 \end{figure}   

 
\end{enumerate}

\item 

\item Let $f_\epsilon(x)=(1-\epsilon)^x$ and $g_\epsilon(x)=1+\epsilon x$, then
\begin{align*}
&f_\epsilon(0) = 1 = g_\epsilon(0)\\
&f_\epsilon(1) = 1-\epsilon = g_\epsilon(1)\\
&\begin{rcases}
 f'_\epsilon(x)&=(1-\epsilon)^x\ln (1-\epsilon) \\
  g'_\epsilon(x)&=\epsilon \\
\end{rcases}f'_\epsilon(0) = \ln(1-\epsilon) < 0= g'_\epsilon(0)
\end{align*}
also $f_\epsilon$ is a convex function as $f''_\epsilon(x)=(1-\epsilon)^x\left[\ln (1-\epsilon)\right]^2>0$ for $\epsilon \in(0,1/2]$. This this proves $f_\epsilon(x)\leq g_\epsilon(x)$ since the initial gradient of $f_\epsilon$ is smaller then $g_\epsilon$ 

%Let $f(\epsilon,x)=(1-\epsilon)^x$, then
%\begin{align*}
%\frac{f^{n}(\epsilon,x)}{d\epsilon^n}=(-1)^n\cdot x(x-1)(x-2)\ldots(x-n+1)(1-\epsilon)^{x-n}
%\end{align*}
%Doing Maclaurin series on $f(\epsilon,x)$ for a fixed $x \in [0,1]$ with respect to $\epsilon$, we get:
%\begin{align*}
%(1-\epsilon)^x&=\sum_{n=0}^{\infty}(-1)^n\frac{x(x-1)(x-2)\ldots(x-n+1)}{n!}\epsilon^n\\
%&=1-\epsilon x + \sum_{n=2}^{\infty}(-1)^n\frac{x(x-1)(x-2)\ldots(x-n+1)}{n!}\epsilon^n
%\end{align*}
%To prove the inequality, it suffices to prove that 
%\begin{align*}
%\sum_{n=2}^{\infty}(-1)^n\frac{x(x-1)(x-2)\ldots(x-n+1)}{n!}\epsilon^n \leq 0
%\end{align*}
%which is true since $(-1)^n\cdot x(x-1)(x-2)\ldots(x-n+1) = -x (1-x)(2-x)\ldots (n-1-x)\leq 0$.
\end{enumerate}











\end{document}