\documentclass[a4paper,10pt]{article}
\setlength{\parindent}{0cm}
\usepackage{amsmath, amssymb, amsthm, mathtools,pgfplots}
\usepackage{graphicx,caption}
\usepackage{verbatim}
\usepackage{venndiagram}
\usepackage[cm]{fullpage}
\usepackage{fancyhdr}
\usepackage{tikz}
\usepackage{listings}
\usepackage{color,enumerate,framed}
\usepackage{color,hyperref}
\definecolor{darkblue}{rgb}{0.0,0.0,0.5}
\hypersetup{colorlinks,breaklinks,
            linkcolor=darkblue,urlcolor=darkblue,
            anchorcolor=darkblue,citecolor=darkblue}

%\usepackage{tgadventor}
%\usepackage[nohug]{diagrams}
\usepackage[T1]{fontenc}
%\usepackage{helvet}
%\renewcommand{\familydefault}{\sfdefault}
%\usepackage{parskip}
%\usepackage{picins} %for \parpic.
%\newtheorem*{notation}{Notation}
%\newtheorem{example}{Example}[section]
%\newtheorem*{problem}{Problem}
\theoremstyle{definition}
%\newtheorem{theorem}{Theorem}
%\newtheorem*{solution}{Solution}
%\newtheorem*{definition}{Definition}
%\newtheorem{lemma}[theorem]{Lemma}
%\newtheorem{corollary}[theorem]{Corollary}
%\newtheorem{proposition}[theorem]{Proposition}
%\newtheorem*{remark}{Remark}
%\setcounter{section}{1}

\newtheorem{thm}{Theorem}[section]
\newtheorem{lemma}[thm]{Lemma}
\newtheorem{prop}[thm]{Proposition}
\newtheorem{cor}[thm]{Corollary}
\newtheorem{defn}[thm]{Definition}
\newtheorem*{examp}{Example}
\newtheorem{conj}[thm]{Conjecture}
\newtheorem{rmk}[thm]{Remark}
\newtheorem*{nte}{Note}
\newtheorem*{notat}{Notation}

%\diagramstyle[labelstyle=\scriptstyle]

\lstset{frame=tb,
  language=Oz,
  aboveskip=3mm,
  belowskip=3mm,
  showstringspaces=false,
  columns=flexible,
  basicstyle={\small\ttfamily},
  breaklines=true,
  breakatwhitespace=true,
  tabsize=3
}


\pagestyle{fancy}




\fancyhead{}
\renewcommand{\headrulewidth}{0pt}

\lfoot{\color{black!60}{\sffamily Zhangsheng Lai}}
\cfoot{\color{black!60}{\sffamily Last modified: \today}}
\rfoot{\textsc{\thepage}}



\begin{document}
\flushright{Zhangsheng Lai\\1002554}
\section*{Stochastic Models: Exercise 2}

\begin{enumerate}
\item
\begin{enumerate}[(i)]
\item 
\begin{align*}
\mathbb{P}(S_{n+1}=k)&=\mathbb{P}(S_{n}=k\mid X_{n+1}=0)\mathbb{P}(X_{n+1}=0)+\mathbb{P}(S_{n}=k-1\mid X_{n+1}=1)\mathbb{P}(X_{n+1}=1)\\
&=\mathbb{P}(S_{n}=k\mid X_{n+1}=0)\cdot q+\mathbb{P}(S_{n}=k-1\mid X_{n+1}=1)\cdot p \quad\text{ since $S_n \perp X_{n+1}$}\\
&=q\,\mathbb{P}(S_{n}=k)+p\,\mathbb{P}(S_{n}=k-1)\\
\end{align*}
\item 
\begin{align*}
\mathbb{P}(S_{n+1}=k)&=q\,\mathbb{P}(S_{n}=k)+p\,\mathbb{P}(S_{n}=k-1)\\
\sum_{k=0}^{n+1}s^k\mathbb{P}(S_{n+1}=k)&=\sum_{k=0}^{n+1}s^kq\,\mathbb{P}(S_{n}=k)+\sum_{k=0}^{n+1}s^kp\,\mathbb{P}(S_{n}=k-1)\\
\sum_{k=0}^{n+1}s^k\mathbb{P}(S_{n+1}=k)&=\sum_{k=0}^{n}s^kq\,\mathbb{P}(S_{n}=k)+\sum_{k=0}^{n}s^{k+1}p\,\mathbb{P}(S_{n}=k)\\
&=(q+ps)\sum_{k=0}^{n}s^k\,\mathbb{P}(S_{n}=k)\\
\end{align*}
let $P_{n}(s) = \mathbb{E}[s^{S_n}]$, then for all $n\geq 0$ we have the relation,
\begin{align*}
P_{n+1}(s)=(q+ps)P_{n}(s)
\end{align*}
and we can inductively deduce that 
\begin{align*}
P_{n}(s)=(q+ps)^n
\end{align*}
which shows that $S_n$ has a binomial distribution with parameters $n, p$.
\end{enumerate}
\item
The extinction probability $\pi$ is the smallest fixed point of $P(s)$. 
\begin{align*}
P(s) &= s\\
as^2+bs+c&=s\\
as^2+(b-1)s+c&=0\\
s&=\frac{1-b\pm \sqrt{(b-1)^2-4ac}}{2a}\\
s&=\frac{1-b\pm \sqrt{(a+c)^2-4ac}}{2a}\\
s&=\frac{1-b\pm (a-c)}{2a}
\end{align*}
hence $s=1$ or $s=c/a$. For extinction to occur, we need the expectation to be greater or equal to 1, i.e. $P'(1) \leq 1$,
\begin{align*}
P'(1)=2a+b\leq 1 &\Leftrightarrow a + 1-c \leq 1\\
&\Leftrightarrow a  \leq c\\
\end{align*}
Extinction occurs when $a \leq c$.
\newpage
\item
\begin{enumerate}[(i)]
\item We first observe that $\mathbb{P}(Z_n=0)$ is same as $\mathbb{P}(T \leq n)$  since the latter is the probability that the population becomes extinct before or at the $n$th generation. We can see that
\begin{align*}
P(s)=P_1(s)&=q+ps\\
P_2(s)&=q+p(q+ps)=q + pq +p^2s\\
&\vdots\\
P_n(s)&=q\sum_{i=0}^{n-1}p^i +p^ns\\
\end{align*}
we can then solve for the following,
\begin{align*}
\mathbb{P}(T=n) &= \mathbb{P}(T \leq n)-\mathbb{P}(T \leq n-1)\\
&= \mathbb{P}(Z_{n}=0)-\mathbb{P}(Z_{n-1} = 0)\\
&= P_{n}(0)-P_{n-1}(0)\\
&= q\sum_{k=0}^{n-1}p^k-q\sum_{k=0}^{n-2}p^k = qp^{n-1}
\end{align*}
\item Given the assumption that $Z_0=i$, define
\begin{align*}
T_i:=\inf\{n:Z_n=0\mid Z_0=i\}
\end{align*}
then $\mathbb{P}(T_i\leq n)= (\mathbb{P}(T \leq n))^i$, then
\begin{align*}
\mathbb{P}(T_i= n) &= \mathbb{P}(T_i \leq n)-\mathbb{P}(T_i \leq n-1)\\
&= (\mathbb{P}(T \leq n))^i-(\mathbb{P}(T \leq n-1))^i\\
&= \left(q\sum_{k=0}^{n-1}p^k\right)^i-\left(q\sum_{k=0}^{n-2}p^k\right)^i\\
\end{align*}

%we need all the $i$ progenitors to be extinct by the $n$th generation, i.e. all the progenitors are extinct latest by the $n$th generation.
%\begin{align*}
%\mathbb{P}(T=n|Z_0=i) &= \left[\mathbb{P}(T\leq n|Z_0=1)\right]^i \quad\text{by independence of the $i$ progenitors}\\
%&=  \left(q\sum_{k=0}^{n}p^k\right)^i
%\end{align*}
\end{enumerate}
\item
\begin{enumerate}[(a)]
\item We start by finding the pgf of the branching process
\begin{align*}
P(s)=\sum_{k=0}^{\infty}s^k(1-q)q^k=\frac{1-q}{1-sq}
\end{align*}
where $|s|<1/q$. The extinction probability is obtained from
\begin{align*}
s=\frac{1-q}{1-sq} &\Leftrightarrow qs^2-s+(1-q)=0\\
&\Leftrightarrow s=1 \text{ or } s=\frac{1-q}{q}
\end{align*}
therefore $\frac{1-q}{q}$ is the extinction probability for $q>1/2$ since $\frac{1-q}{q}\geq 1$ for $q \leq 1/2$.
\item For a progenitor Poisson distributed with parameter $\lambda$, the population becomes extinct when all the subpopulation of each progenitor becomes extinct which we know from (a) has a probability of $\frac{1-q}{q}$.
\begin{align*}
%\mathbb{P}(\text{extinct}\,\mid Z_0 \text{ follows a Poisson distribution}) 
\pi \text{ when $Z_0$ is Poisson distributed}
&= \sum_{k=0}^{\infty}\mathbb{P}(\text{all subpopulations become extinct}\mid Z_0=k)\cdot\mathbb{P}(Z_0=k)\\
&=\sum_{k=0}^{\infty}\left(\frac{1-q}{q}\right)^k\frac{\lambda^ke^{-\lambda}}{k!}\\
&=e^{-\lambda}\cdot e^{\frac{\lambda(1-q)}{q}}=\exp\left(\frac{\lambda}{q}-2\lambda\right)
\end{align*}

%Let $S_{Z_0}=\sum_{i=0}^{Z_0}Z_{0,i}$ where $Z_0$ is a Poisson random variable with parameter $\lambda$ and $Z_{0,i}$ denotes a single progenitor with the distribution in (a) with pgf $P(s)=\frac{1-q}{1-sq}$.
\end{enumerate}
\item 
\begin{enumerate}[(a)]
\item The pgf for $n=2k$ is 
\begin{align*}
g\circ f \circ g \circ \ldots  \circ g \circ f(s)
\end{align*}
where there are $k$ $g$ and $f$ functions in the sequence above. Hence the mean number of individuals in the $n$th generation is 
\begin{align*}
\frac{d(g\circ f \circ g \circ \ldots  \circ g \circ f(s))}{ds}\bigg|_{s=1}
\end{align*}
\item The fixed point of the pgf in (a) governs the extinction probability process.
\item For $i=1,2$, 
\begin{align*}
P_i(s)=\sum_{k=0}^{\infty}s^kp_i(1-p_i)^k=\frac{1-p_i}{1-sp_i}
\end{align*}
with $|s|<1/p_i$.
\end{enumerate}
\end{enumerate}
\end{document}