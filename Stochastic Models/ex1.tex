\documentclass[a4paper,10pt]{article}
\setlength{\parindent}{0cm}
\usepackage{amsmath, amssymb, amsthm, mathtools,pgfplots}
\usepackage{graphicx,caption}
\usepackage{verbatim}
\usepackage{venndiagram}
\usepackage[cm]{fullpage}
\usepackage{fancyhdr}
\usepackage{tikz}
\usepackage{listings}
\usepackage{color,enumerate,framed}
\usepackage{color,hyperref}
\definecolor{darkblue}{rgb}{0.0,0.0,0.5}
\hypersetup{colorlinks,breaklinks,
            linkcolor=darkblue,urlcolor=darkblue,
            anchorcolor=darkblue,citecolor=darkblue}

%\usepackage{tgadventor}
%\usepackage[nohug]{diagrams}
\usepackage[T1]{fontenc}
%\usepackage{helvet}
%\renewcommand{\familydefault}{\sfdefault}
%\usepackage{parskip}
%\usepackage{picins} %for \parpic.
%\newtheorem*{notation}{Notation}
%\newtheorem{example}{Example}[section]
%\newtheorem*{problem}{Problem}
\theoremstyle{definition}
%\newtheorem{theorem}{Theorem}
%\newtheorem*{solution}{Solution}
%\newtheorem*{definition}{Definition}
%\newtheorem{lemma}[theorem]{Lemma}
%\newtheorem{corollary}[theorem]{Corollary}
%\newtheorem{proposition}[theorem]{Proposition}
%\newtheorem*{remark}{Remark}
%\setcounter{section}{1}

\newtheorem{thm}{Theorem}[section]
\newtheorem{lemma}[thm]{Lemma}
\newtheorem{prop}[thm]{Proposition}
\newtheorem{cor}[thm]{Corollary}
\newtheorem{defn}[thm]{Definition}
\newtheorem*{examp}{Example}
\newtheorem{conj}[thm]{Conjecture}
\newtheorem{rmk}[thm]{Remark}
\newtheorem*{nte}{Note}
\newtheorem*{notat}{Notation}

%\diagramstyle[labelstyle=\scriptstyle]

\lstset{frame=tb,
  language=Oz,
  aboveskip=3mm,
  belowskip=3mm,
  showstringspaces=false,
  columns=flexible,
  basicstyle={\small\ttfamily},
  breaklines=true,
  breakatwhitespace=true,
  tabsize=3
}


\pagestyle{fancy}




\fancyhead{}
\renewcommand{\headrulewidth}{0pt}

\lfoot{\color{black!60}{\sffamily Zhangsheng Lai}}
\cfoot{\color{black!60}{\sffamily Last modified: \today}}
\rfoot{\textsc{\thepage}}



\begin{document}
\flushright{Zhangsheng Lai\\1002554}
\section*{Stochastic Models: Exercise 1}

\begin{enumerate}
%\item Let $X_i$ denote the random variable for the number of meals required to order to obtain mini toy $i$. Then the expected number of meals ones need to order to have the complete collection of toys will be $\sum_{i}^NX_i$. Each $X_i$ follows a geometric distribution as each event of getting toy $i$ from the purchase of each meal is independent from each other, has two outcomes (toy $i$ or not toy $i$) and has a constant probability of $1/N$. Hence 
%\begin{align*}
%\mathbb{P}(X_i=k)&=\left(\frac{N-1}{N}\right)^{k-1}\left(\frac{1}{N}\right)\\
%EX_i &= \sum_{k=1}^{\infty}k\left(\frac{N-1}{N}\right)^{k-1}\left(\frac{1}{N}\right)=N\\
%\end{align*}
%Therefore the expected number of meals to order to get the complete set of toys is $N^2$.

\item Let $X_k$ denote the number of meals ordered to collect the $k$th different type of mini toy given that $k-1$ different toys have been collected. Then each $X_k$ is a geometric distribution with $p=\frac{N-(k-1)}{N}$ and 
\begin{align*}
E[X_k] =\frac{N}{N-k+1}
\end{align*}
The expected number of meals ones need to order before collecting a complete set of at least one toy is 
\begin{align*}
E\left[\sum_{k=1}^{N}X_k\right]&=\sum_{i=1}^{N}\frac{N}{N-k+1}\\
&=N\sum_{k=1}^{N}\frac{1}{k}
\end{align*}
\item 
First observe that 
\begin{align*}
\sum_{k=1}^{\infty}\mathbb{P}(X\geq k) = \lim_{n\to\infty}\sum_{k=1}^{n}\mathbb{P}(X\geq k)=\lim_{n\to\infty}\sum_{k=0}^{n}\mathbb{P}(X> k)=\sum_{k=0}^{\infty}\mathbb{P}(X>k)
\end{align*}
We shall now show that it equals $E[X]$ by first defining
\begin{align*}
I_n:=
\begin{cases}
1 \qquad \text{if $X \geq n$ for $n \geq 1$}\\
0 \qquad \text{otherwise}
\end{cases}
\end{align*}
then $\mathbb{P}(X\geq n) = E[I_n]$.
\begin{align*}
\sum_{k=1}^{\infty}\mathbb{P}(X\geq k) &= \sum_{k=1}^{\infty}E[I_k] \\
&= E\left[\sum_{k=1}^{\infty}I_k\right], 
\text{ by Fubini's Theorem}\\
&=E[X]
\end{align*}
since the infinite sum of the indicator random variables is equals to $X$.
\vspace{-.05cm}
\begin{align*}
\int_{0}^{\infty}\mathbb{P}(X>x) \,dx&= \int_{0}^{\infty}\int_{x}^{\infty}f_X(t) \,dt\,dx\\
&= \int_{0}^{\infty}\int_{0}^{t}f_X(t) \,dx\,dt\\
&= \int_{0}^{\infty}tf_X(t) \,dt\\
&= \int_{-\infty}^{\infty}tf_X(t) \,dt\\
&=E[X]
\end{align*}
Using the earlier result,
\begin{align*}
E[X^n] &=\int_{0}^{\infty}\mathbb{P}(X^n>x)\,dx\\
&=\int_{0}^{\infty}\mathbb{P}(X>x^{1/n})\,dx, \text{only consider positive root of $x^{1/n}$ since $X$ is non-negative}\\
\end{align*}
we can do a change of variables, by letting $y=x^n$
\begin{align*}
dx&=ny^{n-1}dy\\
\int_{0}^{\infty}\mathbb{P}(X>x^{1/n})\,dx&=\int_{0}^{\infty}ny^{n-1}\mathbb{P}(X>y)\,dy
%&=\int_{0}^{\infty}\int_{x^{1/n}}^{\infty}f_X(t)\,dt\,dx\\
%&=\int_{0}^{\infty}\int_{0}^{t^{n}}f_X(t)\,dx\,dt\\
%&=\int_{0}^{\infty}t^{n}f_X(t)\,dt\\
%\int_{0}^{\infty}nx^{n-1}\mathbb{P}(X>x) \,dx&= \\
%&=E[X^n]
\end{align*}
and we are done.
\item
\hfill
\begin{enumerate}[(a)]
\item Let $Y=F(X)$ and $0\leq Y \leq 1$ since $F_X$ is a cdf.
\begin{align*}
F_Y(y)=\mathbb{P}(Y\leq y)&=\mathbb{P}(F(X)\leq y)\\
&=\mathbb{P}(X\leq F^{-1}(y)), \text{ since $F$ is non-decreasing }\\
&=F(F^{-1}(y))=y
\end{align*}
since $F_Y(y)=y$, $Y$ is the uniform distribution as the cdf of $U$ over $(0,1)$, $F_U( x)=x$.
\item Let $W=F^{-1}(U)$, then
\begin{align*}
F_W(w) = \mathbb{P}(W\leq w)&=\mathbb{P}(F^{-1}(U)\leq w)\\
&=\mathbb{P}(U\leq F(w)), \text{ since $F_X$ is non-decreasing}\\
&=F(w)
\end{align*}
since $F_W(w)=F(w)$ for all $w$, $F^{-1}(U)$ has the same distribution as $F$.
\end{enumerate}

\item 
\begin{align*}
E[X^2]=\int_{0}^{a}x^2\,dF(x)\,dx \leq \int_{0}^{a}ax\,dF(x) =aE[X], \text{ since $0\leq x \leq a$.}
\end{align*}
then
\begin{align*}
Var[X] &= E[X^2]-E[X]^2\\
&\leq aE[X]-E[X]^2\\
&=a^2\left(\frac{E[X]}{a}-\left(\frac{E[X]}{a}\right)^2\right)\\
&=a^2\left[\alpha(1-\alpha)\right], \text{ where $\alpha=\frac{E[X]}{a}$}
\end{align*}
it suffices to show that $\alpha(1-\alpha)\leq 1/4$ and we are done which is true since we know that the function $y=x-x^2$ attains a maximum value of $y=1/4$ at $x=1/2$. Therefore,
\begin{align*}
Var[X] \leq \frac{a}{4}
\end{align*}
\end{enumerate}












\end{document}