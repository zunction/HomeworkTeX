\documentclass[a4paper,10pt]{article}
\setlength{\parindent}{0cm}
\usepackage{amsmath, amssymb, amsthm, mathtools,pgfplots}
\usepackage{graphicx,caption}
\usepackage{verbatim}
\usepackage{venndiagram}
\usepackage[cm]{fullpage}
\usepackage{fancyhdr}
\usepackage{tikz}
\usepackage{listings}
\usepackage{color,enumerate,framed}
\usepackage{color,hyperref}
\definecolor{darkblue}{rgb}{0.0,0.0,0.5}
\hypersetup{colorlinks,breaklinks,
            linkcolor=darkblue,urlcolor=darkblue,
            anchorcolor=darkblue,citecolor=darkblue}

%\usepackage{tgadventor}
%\usepackage[nohug]{diagrams}
\usepackage[T1]{fontenc}
%\usepackage{helvet}
%\renewcommand{\familydefault}{\sfdefault}
%\usepackage{parskip}
%\usepackage{picins} %for \parpic.
%\newtheorem*{notation}{Notation}
%\newtheorem{example}{Example}[section]
%\newtheorem*{problem}{Problem}
\theoremstyle{definition}
%\newtheorem{theorem}{Theorem}
%\newtheorem*{solution}{Solution}
%\newtheorem*{definition}{Definition}
%\newtheorem{lemma}[theorem]{Lemma}
%\newtheorem{corollary}[theorem]{Corollary}
%\newtheorem{proposition}[theorem]{Proposition}
%\newtheorem*{remark}{Remark}
%\setcounter{section}{1}

\newtheorem{thm}{Theorem}[section]
\newtheorem{lemma}[thm]{Lemma}
\newtheorem{prop}[thm]{Proposition}
\newtheorem{cor}[thm]{Corollary}
\newtheorem{defn}[thm]{Definition}
\newtheorem*{examp}{Example}
\newtheorem{conj}[thm]{Conjecture}
\newtheorem{rmk}[thm]{Remark}
\newtheorem*{nte}{Note}
\newtheorem*{notat}{Notation}

%\diagramstyle[labelstyle=\scriptstyle]

\lstset{frame=tb,
  language=Oz,
  aboveskip=3mm,
  belowskip=3mm,
  showstringspaces=false,
  columns=flexible,
  basicstyle={\small\ttfamily},
  breaklines=true,
  breakatwhitespace=true,
  tabsize=3
}


\pagestyle{fancy}




\fancyhead{}
\renewcommand{\headrulewidth}{0pt}

\lfoot{\color{black!60}{\sffamily Zhangsheng Lai}}
\cfoot{}
\cfoot{\color{black!60}{\sffamily Last modified: \today}}
\rfoot{\color{black!60}{\textsc{\thepage}}}



\begin{document}
\flushright{Zhangsheng Lai\\1002554}
\section*{Stochastic Models: Exercise 4}

\begin{enumerate}
\item 
\begin{align*}
m(t) &= \sum_{n=1}^{\infty}F_n(t),\quad \text{where $F_n(t)$ is the $n-$fold convolution.}\\
&=F(t) + \sum_{n=2}^{\infty}F_n(t) ,\quad\text{since $F(t)=F_1(t)$}\\
&=F(t)+\sum_{n=2}^{\infty} F \ast F_{n-1}(t)\\
&=F(t)+\sum_{n=2}^{\infty} \int_{0}^{t} F_{n-1}(t-x)\,dF(x)\\
&=F(t)+\int_{0}^{t} \sum_{n=1}^{\infty} F_{n}(t-x)\,dF(x)\\
&=F(t)+\int_{0}^{t} m(t-x)\,dF(x)
\end{align*}

\item Let $\{N_D(t), t \geq 0\}$ be a given delay renewal process, then
\begin{align*}
P\left[S_{N_D(t)}\leq s\right] &=\sum_{n=0}^{\infty}P\left[S_{n}\leq s, S_{n+1}>t\right]\\
&=\bar{F}(t)+\sum_{n=1}^{\infty}P\left[S_{n}\leq s, S_{n+1}>t\right]\\
&=\bar{F}(t)+\sum_{n=1}^{\infty}\int_{0}^{\infty}P\left[S_{n}\leq s, S_{n+1}>t\mid S_n=y\right]\,dF_n(y)\\
&=\bar{F}(t)+\int_{0}^{s}\bar{F}(t-y)\,d\left(\sum_{n=1}^{\infty}F_n(y)\right)\\
&=\bar{F}(t)+\int_{0}^{s}\bar{F}(t-y)\,dm_D(y),\quad \text{since $F_1(y)=G(y)$}
\end{align*}
where $m_D(y)=\sum_{n=0}^{\infty}G \ast F_{n}(y)$.

\item
%\begin{align*}
%P\left[X_{N(t)+1}>x\right]&=\sum_{n=0}^{\infty}P\left[X_{N(t)+1}>x\right]\\
%&=\bar{F}(x)+\sum_{n=1}^{\infty}P\left[X_{N(t)+1}>x\right]\\
%\end{align*}
\begin{align*}
P\left[X_{N(t)+1}>x\right]&=\mathbf{E}\left[P\left[X_{N(t)+1}>x\mid S_{N(t)}=t-s\right]\right]
\end{align*}
We now consider cases. When $s\geq x$, $P\left[X_{N(t)+1}>x\mid S_{N(t)}=t-s\right]=1 \geq \bar{F}(x)$. For the case of $s < x$,
\begin{align*}
P\left[X_{N(t)+1}>x\mid S_{N(t)}=t-s\right]&=P\left[X>x\mid X > s\right], \quad \text{where $X$ denotes the interarrival time.}\\
&=\frac{\bar{F}(x)}{\bar{F}(s)}\geq \bar{F}(x)
\end{align*}
Taking expectations on both side
\begin{align*}
P\left[X_{N(t)+1}>x\right] \geq \bar{F}(x)
\end{align*}

\item We will define a renewal reward process where the reward is the cost incurred due to the locked turnstile or the busy server. Then the average cost per unit time is $\lim_{t \to \infty}=\mathbf{E}[R]/\mathbf{E}[X]$ where $R$ is the reward obtained in  a cycle and $X$ is the time of one cycle. Here, a cycle starts when the senior manager arrives at an unlocked turnstile. Then the time of one cycle is the time needed for the turnstile to unlock plus the time required for the next senior manager to arrive at the unlocked turnstile which we get
\begin{align*}
\mathbf{E}[X]=t+\frac{1}{\lambda}
\end{align*}
since the arrivals of the senior managers follow a Poisson process with rate $\lambda$, thus their interarrival times are exponentially distributed with mean $1/\lambda$. To find the `reward' incurred, we have to find the cost incurred at the turnstile $(T)$ and server $(S)$. For the fixed time $t$, the number of arrivals is $\mathbf{E}[N(t)]=\lambda t$. Since each arrival incurs a cost of $c$, $\mathbf{E}[T]=\mathbf{E}[cN(t)]=ct\lambda$. To incur a cost at the server, the serving time $Y$ has to be greater than $t$ and also greater than the interarrival time $Z$. Hence
\begin{align*}
\mathbf{E}[S]&=K\cdot P\left[Y\geq t\right]\cdot P\left[Y\geq Z\right]\\
&=K\cdot e^{-\lambda t}\cdot P\left[Y\geq Z\right]\\
&=K\cdot e^{-\lambda t}\cdot \int_{0}^{\infty}P\left[Y\geq Z\mid Z=x\right]f_Z(x)\,dx\\
&=K\cdot e^{-\lambda t}\cdot \int_{0}^{\infty}P\left[Y\geq x\right]f_Z(x)\,dx\\
&=K\cdot e^{-\lambda t}\cdot \int_{0}^{\infty}e^{-\mu x}\lambda e^{-\lambda x}\,dx\\
&=\lambda K\cdot e^{-\lambda t}\cdot \int_{0}^{\infty}e^{-\mu x} e^{-\lambda x}\,dx\\
&=\lambda K\cdot e^{-\lambda t}\cdot \left[-\frac{1}{\lambda+\mu}e^{-x(\lambda+\mu)}\right]_{0}^{\infty}\\
&=\frac{\lambda Ke^{-\lambda t}}{\lambda+\mu}
\end{align*}
Thus we arrive at 
\begin{align*}
\frac{\mathbf{E}[R]}{\mathbf{E}[X]}&=\frac{\mathbf{E}[T]+\mathbf{E}[S]}{\mathbf{E}[X]}\\
&=\frac{ct\lambda+\frac{\lambda Ke^{-\lambda t}}{\lambda+\mu}}{t+\frac{1}{\lambda}}
\end{align*}
\item Given the scenario, a new cycle starts each time the policyholder payment rate reverts to $r_1$. 
\begin{enumerate}[(i)]
\item Since the claims are made with a Poisson process of rate $\lambda$, the interarrival times are exponentially distributed with parameter $\lambda$, thus it is not lattice. Hence
\begin{align*}
P_i=\frac{\mathbf{E}[\text{paying rate } r_i]}{\mathbf{E}[\text{paying rate } r_0]+\mathbf{E}[\text{paying rate } r_1]}
\end{align*}
For the given $s$ and letting $X$ denote the interarrival time, we can have either $X>s$ or $X \leq s$, which we shall use to find the expectations.
\begin{align*}
\mathbf{E}[\text{paying rate } r_0] &=\int_{s}^{\infty}(x-s)\lambda e^{-\lambda x}\,dx\\
&=\int_{s}^{\infty}x\lambda e^{-\lambda x}\,dx-\int_{s}^{\infty}s\lambda e^{-\lambda x}\,dx\\
&=\left[-\frac{x}{\lambda}e^{-\lambda x}-\frac{1}{\lambda^2}e^{-\lambda x}\right]_{s}^{\infty}+ s\left[e^{-\lambda x}\right]_{s}^{\infty}=\frac{1}{\lambda}e^{-\lambda x}\\
\mathbf{E}[\text{paying rate } r_1] &=\int_{0}^{s}x\lambda e^{-\lambda x}\,dx+\int_{s}^{\infty}s\lambda e^{-\lambda x}\,dx\\
&=\lambda\left[-\frac{x}{\lambda}e^{-\lambda x}-\frac{1}{\lambda^2}e^{-\lambda x}\right]_{0}^{s}+se^{-\lambda x}=\frac{1}{\lambda}-\frac{1}{\lambda}e^{-\lambda x}
\end{align*}
Therefore
\begin{align*}
\mathbf{E}[\text{paying at rate }r_0]&=\frac{e^{-\lambda x}/\lambda}{1/\lambda}=e^{-\lambda x} \\
\mathbf{E}[\text{paying at rate }r_1]&=\frac{1/\lambda -e^{-\lambda x}/\lambda}{1/\lambda} =1-e^{-\lambda x}
\end{align*}
\item The long-run average amount paid per unit time is 
\begin{align*}
P_0r_0+P_1r_1&=r_0e^{-\lambda x}+r_1(1-e^{-\lambda x})\\
&=r_1+(r_0-r_1)e^{-\lambda x}
\end{align*}
\end{enumerate}

\item
\begin{enumerate}[(a)]
\item To find the rate of occurrance of $d-$events, we want to find $\lim_{t\to\infty}N(t)/t$ which we know to be $1/\mu$, where $\mu$ is the expected interarrival time of a $d-$event. Since we are given that the events occur according to a Poisson process, the interarrival times, $X_i$ are exponentially distributed with parameter $\lambda$ and a $d-$event occurs when the interarrival time is less than $d$. Let a new cycle start when a $d-$event occurs, we then see that the $d-$events occur with a geometric distribution $Y$ and thus the expected number of renewals (regardless of whether it is a $d-event$ or not) to before a $d-$event occurs is $1/\mathbf{P}[X\leq d]$. The value of $mu$ is given by
\begin{align*}
\mu&=\mathbf{E}\left[\sum_{i=1}^{Y}X_i\right]=\mathbf{E}\left[Y\right]\cdot \mathbf{E}\left[X\right],\quad \text{By Wald's equation}\\
&=\frac{1}{\mathbf{P}[X\leq d]}\cdot \frac{1}{\lambda}=\frac{1}{\lambda\mathbf{P}[X\leq d]}
\end{align*}


%As the expected interarrival time is $1/\lambda$, the expected time between $d-$events is $1/\lambda \mathbf{P}[X\leq d]$ and 
thus we get that the rate is $\lambda(1- e^{-\lambda d})$.
\item Since the arrivals arrive at a rate of $\lambda$, the proportion of $d-$events is $\lambda(1-e^{-\lambda d})/\lambda=1-e^{-\lambda d}$
\end{enumerate}


\item 
\begin{enumerate}[(a)]
\item A cycle is complete when the bus arrives at SUTD and we can see it as on when it is travelling from SUTD to Simei and off when it is travelling in the other direction. Hence the proportion of the driving time spent going from SUTD to Simei is 
\begin{align*}
\frac{\mathbf{E}[\text{SUTD to Simei}]}{\mathbf{E}[\text{SUTD to Simei}]+\mathbf{E}[\text{Simei to SUTD}]}
\end{align*}
taking the distance between the two places to be $d$, the expectations are evaluated to be
\begin{align*}
\mathbf{E}[\text{SUTD to Simei}]&=\int_{40}^{50}\frac{d}{10x}\,dx=d\left[\frac{\ln x}{10}\right]_{40}^{50}=\frac{d}{10}\ln(5/4)\\
\mathbf{E}[\text{Simei to SUTD}]&= \frac{1}{2}\cdot\frac{d}{40}+\frac{1}{2}\cdot\frac{d}{50}=\frac{9d}{400}
\end{align*}
thus the proportion of time is 
\begin{align*}
\frac{d/10\ln(5/4)}{d/10\ln(5/4)+9d/400}=\frac{\ln(5/4)}{\ln(5/4)+9/40}
\end{align*}
\item The travelling speed from SUTD to Simei is uniformly distributed from 40 to 50 as such, the probability of it travelling at 40 is 0 as it is a continuous variable. Thus it suffices to consider the time at which it travels at 40 from Simei to SUTD. The proportion of time travelling at 40 from Simei to SUTD is 
\begin{align*}
\frac{d/40}{d/40+d/50}=\frac{5}{9}
\end{align*}
thus the proportion that the driving time is at a speed of 40 is 
\begin{align*}
\frac{1/8}{\ln(5/4)+9/40}
\end{align*}
\end{enumerate}

\end{enumerate}
\end{document}