\documentclass[a4paper,10pt]{article}
\setlength{\parindent}{0cm}
\usepackage{amsmath, amssymb, amsthm, mathtools,pgfplots}
\usepackage{graphicx,caption}
\usepackage{verbatim}
\usepackage{venndiagram}
\usepackage[cm]{fullpage}
\usepackage{fancyhdr}
\usepackage{tikz}
\usepackage{listings}
\usepackage{color,enumerate,framed}
\usepackage{color,hyperref}
\definecolor{darkblue}{rgb}{0.0,0.0,0.5}
\hypersetup{colorlinks,breaklinks,
            linkcolor=darkblue,urlcolor=darkblue,
            anchorcolor=darkblue,citecolor=darkblue}

%\usepackage{tgadventor}
%\usepackage[nohug]{diagrams}
\usepackage[T1]{fontenc}
%\usepackage{helvet}
%\renewcommand{\familydefault}{\sfdefault}
%\usepackage{parskip}
%\usepackage{picins} %for \parpic.
%\newtheorem*{notation}{Notation}
%\newtheorem{example}{Example}[section]
%\newtheorem*{problem}{Problem}
\theoremstyle{definition}
%\newtheorem{theorem}{Theorem}
%\newtheorem*{solution}{Solution}
%\newtheorem*{definition}{Definition}
%\newtheorem{lemma}[theorem]{Lemma}
%\newtheorem{corollary}[theorem]{Corollary}
%\newtheorem{proposition}[theorem]{Proposition}
%\newtheorem*{remark}{Remark}
%\setcounter{section}{1}

\newtheorem{thm}{Theorem}[section]
\newtheorem{lemma}[thm]{Lemma}
\newtheorem{prop}[thm]{Proposition}
\newtheorem{cor}[thm]{Corollary}
\newtheorem{defn}[thm]{Definition}
\newtheorem*{examp}{Example}
\newtheorem{conj}[thm]{Conjecture}
\newtheorem{rmk}[thm]{Remark}
\newtheorem*{nte}{Note}
\newtheorem*{notat}{Notation}

%\diagramstyle[labelstyle=\scriptstyle]

\lstset{frame=tb,
  language=Oz,
  aboveskip=3mm,
  belowskip=3mm,
  showstringspaces=false,
  columns=flexible,
  basicstyle={\small\ttfamily},
  breaklines=true,
  breakatwhitespace=true,
  tabsize=3
}


\pagestyle{fancy}




\fancyhead{}
\renewcommand{\headrulewidth}{0pt}

\lfoot{\color{black!60}{\sffamily Zhangsheng Lai}}
\cfoot{}
\cfoot{\color{black!60}{\sffamily Last modified: \today}}
\rfoot{\color{black!60}{\textsc{\thepage}}}



\begin{document}
\flushright{Zhangsheng Lai\\1002554}
\section*{Stochastic Models: Exercise 4}

\begin{enumerate}
\item 
\begin{align*}
m(t) &= \sum_{n=1}^{\infty}F_n(t),\quad \text{where $F_n(t)$ is the $n-$fold convolution.}\\
&=F(t) + \sum_{n=2}^{\infty}F_n(t) ,\quad\text{since $F(t)=F_1(t)$}\\
&=F(t)+\sum_{n=2}^{\infty} F \ast F_{n-1}(t)\\
&=F(t)+\sum_{n=2}^{\infty} \int_{0}^{t} F_{n-1}(t-x)\,dF(x)\\
&=F(t)+\int_{0}^{t} \sum_{n=1}^{\infty} F_{n}(t-x)\,dF(x)\\
&=F(t)+\int_{0}^{t} m(t-x)\,dF(x)
\end{align*}

\item Let $\{N_D(t), t \geq 0\}$ be a given delay renewal process, then
\begin{align*}
P\left[S_{N_D(t)}\leq s\right] &=\sum_{n=0}^{\infty}P\left[S_{n}\leq s, S_{n+1}>t\right]\\
&=\bar{F}(t)+\sum_{n=1}^{\infty}P\left[S_{n}\leq s, S_{n+1}>t\right]\\
&=\bar{F}(t)+\sum_{n=1}^{\infty}\int_{0}^{\infty}P\left[S_{n}\leq s, S_{n+1}>t\mid S_n=y\right]\,dF_n(y)\\
&=\bar{F}(t)+\int_{0}^{s}\bar{F}(t-y)\,d\left(\sum_{n=1}^{\infty}F_n(y)\right)\\
&=\bar{F}(t)+\int_{0}^{s}\bar{F}(t-y)\,dm_D(y),\quad \text{since $F_1(y)=G(y)$}
\end{align*}
where $m_D(y)=\sum_{n=0}^{\infty}G \ast F_{n}(y)$.
\item
\begin{align*}
P\left[X_{N(t)+1}>x\right]&=\sum_{n=0}^{\infty}P\left[X_{N(t)+1}>x\right]\\
&=\bar{F}(x)+\sum_{n=1}^{\infty}P\left[X_{N(t)+1}>x\right]\\
\end{align*}


\item

\item Given the scenario, a new cycle starts each time the policyholder payment rate reverts to $r_1$. 
\begin{enumerate}[(i)]
\item Since the claims are made with a Poisson process of rate $\lambda$, the interarrival times are exponentially distributed with parameter $\lambda$, thus it is not lattice. Hence
\begin{align*}
P_i=\frac{\mathbf{E}[\text{paying rate } r_i]}{\mathbf{E}[\text{paying rate } r_0]+\mathbf{E}[\text{paying rate } r_1]}
\end{align*}
For the given $s$ and letting $X$ denote the interarrival time, we can have either $X>s$ or $X \leq s$, which we shall use to find the expectations.
\begin{align*}
\mathbf{E}[\text{paying rate } r_0] &=\int_{s}^{\infty}(x-s)\lambda e^{-\lambda x}\,dx\\
&=\int_{s}^{\infty}x\lambda e^{-\lambda x}\,dx-\int_{s}^{\infty}s\lambda e^{-\lambda x}\,dx\\
&=\left[-\frac{x}{\lambda}e^{-\lambda x}-\frac{1}{\lambda^2}e^{-\lambda x}\right]_{s}^{\infty}+ s\left[e^{-\lambda x}\right]_{s}^{\infty}=\frac{1}{\lambda}e^{-\lambda x}\\
\mathbf{E}[\text{paying rate } r_1] &=\int_{0}^{s}x\lambda e^{-\lambda x}\,dx+\int_{s}^{\infty}s\lambda e^{-\lambda x}\,dx\\
&=\lambda\left[-\frac{x}{\lambda}e^{-\lambda x}-\frac{1}{\lambda^2}e^{-\lambda x}\right]_{0}^{s}+se^{-\lambda x}=\frac{1}{\lambda}-\frac{1}{\lambda}e^{-\lambda x}
\end{align*}
Therefore
\begin{align*}
\mathbf{E}[\text{paying at rate }r_0]&=\frac{e^{-\lambda x}/\lambda}{1/\lambda}=e^{-\lambda x} \\
\mathbf{E}[\text{paying at rate }r_0]&=\frac{1/\lambda -e^{-\lambda x}/\lambda}{1/\lambda} =1-e^{-\lambda x}
\end{align*}
\item The long-run average amount paid per unit time is 
\begin{align*}
P_0r_0+P_1r_1&=r_0e^{-\lambda x}+r_1(1-e^{-\lambda x})\\
&=r_1+(r_0-r_1)e^{-\lambda x}
\end{align*}
\end{enumerate}

\item
\begin{enumerate}[(a)]
\item 
\item
\end{enumerate}


\item 
\begin{enumerate}[(a)]
\item 
\item
\end{enumerate}

\end{enumerate}
\end{document}