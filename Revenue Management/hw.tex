\documentclass[a4paper,10pt]{article}
\setlength{\parindent}{0cm}
\usepackage{amsmath, amssymb, amsthm, mathtools,pgfplots}
\usepackage{graphicx,caption}
\usepackage{verbatim}
\usepackage[cm]{fullpage}
\usepackage{fancyhdr}
\usepackage{tikz}
\usepackage{listings}
\usepackage{color,enumerate,framed}
\usepackage{color,hyperref}
\definecolor{darkblue}{rgb}{0.0,0.0,0.5}
\hypersetup{colorlinks,breaklinks,
            linkcolor=darkblue,urlcolor=darkblue,
            anchorcolor=darkblue,citecolor=darkblue}

%\usepackage{tgadventor}
%\usepackage[nohug]{diagrams}
\usepackage[T1]{fontenc}
%\usepackage{helvet}
%\renewcommand{\familydefault}{\sfdefault}
%\usepackage{parskip}
%\usepackage{picins} %for \parpic.
%\newtheorem*{notation}{Notation}
%\newtheorem{example}{Example}[section]
%\newtheorem*{problem}{Problem}
\theoremstyle{definition}
%\newtheorem{theorem}{Theorem}
%\newtheorem*{solution}{Solution}
%\newtheorem*{definition}{Definition}
%\newtheorem{lemma}[theorem]{Lemma}
%\newtheorem{corollary}[theorem]{Corollary}
%\newtheorem{proposition}[theorem]{Proposition}
%\newtheorem*{remark}{Remark}
%\setcounter{section}{1}

\newtheorem{thm}{Theorem}[section]
\newtheorem{lemma}[thm]{Lemma}
\newtheorem{prop}[thm]{Proposition}
\newtheorem{cor}[thm]{Corollary}
\newtheorem{defn}[thm]{Definition}
\newtheorem*{examp}{Example}
\newtheorem{conj}[thm]{Conjecture}
\newtheorem{rmk}[thm]{Remark}
\newtheorem*{nte}{Note}
\newtheorem*{notat}{Notation}

%\diagramstyle[labelstyle=\scriptstyle]

\lstset{frame=tb,
  language=Oz,
  aboveskip=3mm,
  belowskip=3mm,
  showstringspaces=false,
  columns=flexible,
  basicstyle={\small\ttfamily},
  breaklines=true,
  breakatwhitespace=true,
  tabsize=3
}


\pagestyle{fancy}




\fancyhead{}
\renewcommand{\headrulewidth}{0pt}

\lfoot{\color{black!60}{\sffamily Zhangsheng Lai}}
\cfoot{\color{black!60}{\sffamily Last modified: \today}}
\rfoot{\textsc{\thepage}}



\begin{document}
Zhangsheng Lai\\1002554



\section*{Revenue Management}
We consider the general case where the length of the buffer is $n$. Then the states are given by $\mathcal{X}:=A \cup B$, where $A= \{0,\ldots, n\}$ and $B= \{n+1,\ldots, 2n+1\}$, with $A$ denoting the states where the server is off and $B$ denoting the states where the server is on. The number of customers in the queue is exactly the state number for states in $A$ and the number of customers in the queue for states in $B$ is modulo $n+1$ of the state number. As for the action, we have $\mathcal{A}(x) = \{0,1\}$ where 0 (1) means the server is off (on).


With that, we can get evaluate the reward function by considering cases.

\paragraph{Action: $a=0$}server is switched off
\begin{itemize}
\item $x <n$, $R(x,a,w) = \frac{1}{4}(-x)+\frac{3}{4}(-x-1)$
\item $x =n$, $R(x,a,w) = \frac{1}{4}(-n)+\frac{3}{4}(-n-1000)$
\item $n< x < 2n+1$, $R(x,a,w) = \frac{1}{4}(-(x \mod n+1))+\frac{3}{4}(-(x \mod n+1)-1)$
\item $x =2n+1$, $R(x,a,w) = \frac{1}{4}(-n)+\frac{3}{4}(-n-1000)$
\end{itemize}

\paragraph{Action: $a=1$}server is switched on
\begin{itemize}
\item $x <n$, $R(x,a,w) = \frac{1}{4}(-x)+\frac{3}{4}(-x-1)-10$
\item $x =n$, $R(x,a,w) = \frac{1}{4}(-n)+\frac{3}{4}(-n-1)-10$
\item $n< x < 2n+1$, $R(x,a,w) = \frac{1}{4}(-(x \mod n+1))+\frac{3}{4}(-(x \mod n+1)-1)$
\item $x =2n+1$, $R(x,a,w) = \frac{1}{4}(-n)+\frac{3}{4}(-n-1)$
\end{itemize}



\subsection*{Value Iteration}

\subsection*{Policy Iteration}

\subsection*{Linear Programming}

\end{document}