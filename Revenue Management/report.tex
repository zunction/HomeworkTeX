%!TEX root = ./master.tex



\section{Introduction}
\annotation{Introduction to talk about what we have learnt in class (Single Product DP) and how the model in the paper we have chosen is a generalizing from single to multi-products.}

In the single product dynamic pricing model that we were introduced in class, we looked at a monopolist seller which finite units $x_0$ of a single indivisible product over a finite and continuous horizon $[0,T)$. The unit price $\pi_t$ is decided by the seller at each point of time $t \in [0,T)$ and customers product valuations follow a distribution over $\mathbb{R}_+$. However, practically, sellers often have a wide range of products that the customers can choose from, with the products having similar functionalities; catering to customers of varying purchasing power.

Thus we look to a multiproduct model \cite{Li2009}





\section{Related Work}
\annotation{[Optional] We might like to discuss other papers like \cite{Gallego1997} that might be related to the paper we are looking at, e.g. the 1997 paper Yiwei told us to look at.}


\section{Multinomial Logit Model}
\annotation{The model of course.}


\section{Experiments}
\annotation{If we manage to find time to run any experiments.}


\section{Discussions}
\annotation{This is where we can add our comments and our inputs, how the model can be further improved or how we can find estimates for the solution.}


\section{Conclusions}
\annotation{Closing conclusions, futher areas that can be explored and research opportunities (for Yiwei only haha).}


\bibliographystyle{abbrv}
\bibliography{./bib}

