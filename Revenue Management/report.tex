%!TEX root = ./master.tex



\section{Introduction}
%\annotation{Introduction to talk about what we learnt in class (Single Product DP) and how the model in the paper we have chosen is a generalizing from single to multi-products.}

In the single product dynamic pricing model that we were introduced in class, we looked at a monopolist seller which finite units $x_0$ of a single indivisible product with no replenishment over a finite and continuous horizon $[0,T)$. The unit price $\pi_t$ is decided by the seller at each point of time $t \in [0,T)$ and customers product valuations follow a distribution over $\mathbb{R}_+$. However, sellers often have a wide range of products that the customers can choose from, with the products having similar functionalities; catering to customers of varying purchasing power. 

Thus we look to a multiproduct model \cite{Li2009} where the retailer who encounters a long supply leadtime and short selling season with substitute products when replenishment products are unavailable. For example, when shopping, if the item of your choice is not available and its stock won't be replenish, then you will have to find a substitute product. Thus the retailer can control the price of the products with time so as to maximize his or her revenue. Instead of considering the customers product valuation as a general random variable, we explicitly construct a model that sees customer purchase probability as a function of the quality and the price. Also, a time dependent customer arrival rate is modeled by a nonhomogeneous Poisson process \annotation{(add that it is forcasted in the model)}. Lastly, the paper assumes that the products are equally dissimiliar; the addition or removal of one product affects the choice of all other products in the same portion.





%\section{Related Work}
%\annotation{[Optional] We might like to discuss other papers like \cite{Gallego1997} that might be related to the paper we are looking at, e.g. the 1997 paper Yiwei told us to look at.}


\section{The Multinomial Logit Model}
%\annotation{The model of course.}


\subsection{Summary of the MNL model}
The MNL model is a multiproduct dynamic pricing problem and assumes the products are nominal. It describes dynamic consumer choice preferences over substitute products as prices are varied. The customer makes choices from a range of products to maximise his or her utility. The utility of the $i$ product is defined by the logit demand function $v^i(r^i) = \exp ((q^i-r^i)/\mu)$ which is a positive function of the quality $q^i$, the price $r^i$ of product $i$ for $i=1,\ldots, n$ and $\mu$ is the constant representing the stochastic preference of the choice process. The customer expected demand probability of product $i$ is defined as 
\begin{align}
P^i(r) = \frac{v^i(r^i)}{v^0+\sum_{j=1}^{n}v^j(r^j)},\quad i = 1, \ldots n\label{eq:expecteddemandprob}
\end{align}
and $v^0$ denotes the utility of not making any purchase.


The customer's arrivals are assumed to follow a nonhomogenous Poisson process with a time-dependent rate $\lambda(t)$ and in a small time interval $\delta t$, the probability of one arrival is $\lambda(t)\delta t$. The price of the products at time $t$ is $r_t = (r_t^1,\ldots,r_t^n) \in \mathbb{R}^n$ is decided according to the current inventory level. If some product $i$ is sold out before the end of the selling season, as we do not allow replenishments, the price is set to $r^i_t = \infty$ for all $t$ occuring after $c^i_t=0$. If all the products are sold at at $t < T$, then the selling season ends at $t$. All unsold products are salvaged at $t=0$. 


Let $\mathbf{r}=\{r_t, t \in [T,0]\} = (r^1, \ldots, r^n)$ denote a pricing policy for the entire season, where $r^i$ denotes the trend of the price of product $i$ over the selling season. We denote the probabilty for a customer to arrive at time $t$ to choose product $i \in \mathbf{n} = \{1,\ldots, n\}$ by $P_t^i(r_t)$ and $P_t^0(r_t)$ denotes the no-purchase probability. Using (\ref{eq:expecteddemandprob}) with the no-purchase utility $v^0 = \exp (u_0/\mu)$, we have the demand of product $i$ and of no-purchase
\begin{align}
P_t^i(r_t) &= \frac{\exp((q^i-r^i_t)/\mu)}{\sum_{i \in \mathbf{n}}\exp((q_i-r^i_t)/\mu)+\exp(u_0/\mu)} \label{eq:probi}\\
P_t^0(r_t)&= \frac{\exp(u_0/\mu)}{\sum_{i \in \mathbf{n}}\exp((q_i-r^i_t)/\mu)+\exp(u_0/\mu)}\label{eq:prob0}
\end{align}

and by construction, $P_t^0(r_t) + \sum_{i \in \mathbf{n}}P_t^i(r_t) = 1$.

The revenue rate at time $t$ is 
\begin{align}
\Phi_t(r_t):=\lambda(t) \sum_{i=1}^{n}r_t^iP_t^i(r_t) \label{eq:revenuerate_reward}
\end{align}

By setting the price $r_t$, the retailer controls the probability of customer choice process so the revenue rate is optimized. From (\ref{eq:probi}) and (\ref{eq:prob0}) we obtain $P_t^i(r_t)/P_t^0=\exp ((q^i-r_t^i-u_0)/\mu)$.
\begin{align}
r_t^i=q^i-u_0-\mu \ln P_t^i + \mu \ln P_t^0, \quad i=1,\ldots, n \label{eq:pricetoprob}
\end{align}
which is a mapping from the sales probability space to the price space, and in (\ref{eq:probi}) and (\ref{eq:prob0}) we have a mapping in the opposite direction. However, these two mappings are not inverses of each other as the mapping from the price space to the sales probability space is not injective and hence cannot be a bijection. This can be easily observed for given pricing $r_t = (r_t^1,\ldots,r_t^n)$, choosing $r'_t = (r_t^1\pm\ln k,\ldots,r_t^n\pm \ln k)$ for some positive real constant $k$ will produce the same demand probability of product. Thus the vector of prices obtained from a given sales probability has to be chosen from a set of allowable prices  \cite{Gallego1997}.


Thus, by choosing from the set of allowable prices, we have a one-to-one mapping from the price space to the sales probability space. The revenue rate as in (\ref{eq:revenuerate_reward}) can be rewritten as 
\begin{align}
\Phi_t(P_t) = \lambda(t)\sum_{i=1}^{n}\left[q^i-u_0-\mu \ln P_t^i + \mu \ln P_t^0 \right]P_t^i\label{eq:revenuerate_prob}
\end{align}
where $P_t = (P_t^1,\ldots, P_t^i)$


\subsection{Optimal control of the MNL model}
Let $N_t^i$ be the Poisson counting process of the number of products sold up to time $t$. If a customer arrives and chooses the product $i$ at time $t$ then $dN_t^i = 1$, otherwise $dN_t^i=0$. $N_t = (N_t^1, \ldots, N_t^n)$ is defined to be a multivariate stochastic process controlled by the price vector $r_t$ through the probability $P_t$. Given an initial inventory $c_T$, the expected revenue of the price scheme $\mathbf{r}$ 	for the entire selling season is 
\begin{align}
\mathbf{w}_T^{\mathbf{r}}(c_T) = \left(E_{\mathbf{r}}\left[\int_0^Tr^i_t\,dN^i_t\right]\right)_{i=1,\ldots,n}
\end{align}
Using the relationship between the pricing and the sales probability, the paper rewrites the expected reveue in terms of its corresponding probability scheme $\mathbf{P}=\{P_t, t \in [T,0)\}$ to get
\begin{align}
\mathbf{w}^{\mathbf{P}}_T(c_T) = \left[\int_{0}^{T}\lambda(t) \sum_{i=1}^{n}P_t^i[q^i-u_0-\mu \ln P_t^i + \mu \ln P_t^0]\,dt\right] = \int_{0}^{T}\Phi_t(P_t)\,dt
\end{align}

By writing the expected revenue in terms of sales probabilities instead of pricing, it reduces the space from $\mathbb{R}^n_+$ to $[0,1]^n$. This allows the search of the optimal control and value to be much more managable \annotation{(to remove or to justify it better)} as we are searching for a solution from a probability simplex. The retailer searchs for a pricing policy $\hat{\mathbf{r}}$ that maximizes the expected revenue over the selling season, which is denoted as
\begin{align}
\hat{\mathbf{w}}_T^{\mathbf{P}}(c_T) = \max_{P_t \in [0,1)^n, t \in [T,0)}\int_{0}^{T}\lambda(t)\sum_{i=1}^{n}P_t^i\left(q^i-u_0-\mu \ln P_t^i+\mu \ln P_t^0\right)\, dt
\end{align}
such that $dc_t^i = -dN_t^i$, which simply means any change in the inventory levels of product $i$ is the negative of the change in the Poisson counting process of the number of products $i$ sold up till time $i$. This formulation is observed to be a optimal control problem; our utility is $\mathbf{w}_T^{\mathbf{P}}(c_T)$, with scalar utility rate function $\Phi_t(\cdot)$, the system state vector is $c_t$ and the control vector we are trying to find is $P_t$ for some time $t$. If we used the expected revenue formulation in (\ref{eq:revenuerate_prob}) instead, the control vector we are trying to find will be $r_t$. Now let
\begin{align}
\mathbf{w}_t^{\mathbf{P}}(c_t) = \int_{0}^{t}\lambda(s)\sum_{i=1}^{n}P_s^i\left(q^i-u_0-\mu \ln P_s^i+\mu \ln P_s^0\right)\, ds
\end{align}
and 
\begin{align}
\hat{\mathbf{w}}_t^{\mathbf{P}}(c_t) = \max_{P_s \in [0,1)^n, s \in [t,0)} \mathbf{w}_t^{\mathbf{P}}(c_t)
\end{align}

%\annotation{
%\paragraph{Principle of Optimality}
%For every $(t,x) \in [t_0,t_1) \times \mathbb{R}^n$ and every $\Delta t \in (0, t_1-t]$, the value function $V$ satisfies the relation
%\begin{align*}
%V(t,x) = \inf_{u[t,t+\Delta t]} \left\{\int^{t + \Delta t}_{t} L(s,x(s),u(s))\, ds + V(t+\Delta t, x(t+\Delta t))\right\}
%\end{align*}
%}


The authors then solve the optimal control problem by applying the Hamilton-Jacobi-Bellman (HJB) sufficient conditions. The HJB provide sufficient conditions for optimality, which says that the solution for the equation,
\begin{align}
\frac{\partial \hat{\mathbf{w}}_t^{\mathbf{P}}(c_t)}{\partial t} = \max_{P_t \in [0,1)^n, t \in [T,0)}\sum_{i=1}^{n} \lambda(t)P_t^i \left(q^i-u_0-\mu \ln P_t^i + \mu \ln P_t^0- \Delta^i \hat{\mathbf{w}}_t^{\mathbf{P}}(c_t)\right) \label{eq:HJBsufficient}
\end{align}
gives us the optimal revenue rate, %$\hat{\mathbf{w}}_t^{\mathbf{P}}(c_t)$, 
where $\Delta^i \hat{\mathbf{w}}_t^{\mathbf{P}}(c_t) = \hat{\mathbf{w}}_t^{\mathbf{P}}(c_t) - \hat{\mathbf{w}}_t^{\mathbf{P}}(c_t-e^i)$, is the marginal value of product $i$, with $e^i$ being a $n$-dimensional vector with the $i$-th component being 1 and the rest zeros, and $\hat{\mathbf{w}}_t^{\mathbf{P}}(c_t)$ satisfying the boundary and initial conditions
\begin{align}
\hat{\mathbf{w}}_t^{\mathbf{P}}(0) &= 0 \quad \forall t\\
\hat{\mathbf{w}}_0^{\mathbf{P}}(c_0) &= 0 \quad \forall c_0
\end{align}
<<<<<<< HEAD
The boundary and initial conditions states that if there is zero inventory then at any time the optimal revenue rate is zero, and if the selling season has ended regardless of the amount of inventory left, the optimal revenue rate will also be zero. The HJB conditions are equivalent to maximizing the revenue rate $\Phi_t(P_t)$ at each $t$.
=======
The boundary and initial conditions simply say that if there is zero inventory then at any time the optimal revenue rate is zero, and if the selling season has ended regardless of the amount of inventory left, the optimal revenue rate will also be zero. The HJB conditions are equivalent to maximizing the revenue rate $\Phi_t(c_T)$ \annotation{(or is it $\Phi_t(P_t)$)} at each $t$.

%The HJB provides limited utility for constructing optimal policies and value functions, but
%does provide a relatively inexpensive way to verify optimality if one is able to ?guess? a
%solution. 
>>>>>>> 6bdf37781fdadad3618f86458f70c73ff87326ca

\subsection{Optimal Dynamic Pricing Policy}

To find the optimal dynamic pricing policy, we just need to find the maximizer $P_t^\ast$ of the HJB equation in (\ref{eq:HJBsufficient}). Define
<<<<<<< HEAD

\begin{align*}
\Psi_t^{\mathbf{P}}(c_t) = \sum_{i=1}^{n}\lambda(t) P_t^i\left[q^i - u_0 = \mu \ln P_t^i + \mu \ln P_t^0 - \Delta^i\hat{\mathbf{w}}_t^{\mathbf{P}}(c_t)\right]
\end{align*}

and the fact that $\Psi_t^{\mathbf{P}}(c_t)$ is concave is derived from showing that its Hessian is negative and thus it is semi-negative definite. Thus it has a global maximum over $P_t$ which can be found by solving,
\begin{align}
\frac{\partial \Psi_t^{\mathbf{P}}(c_t)}{\partial P_t^i} = \lambda(t) \left[q^i - u_0 + \mu \ln P_t^i - \mu \ln P_t^0 - \Delta^i \hat{\mathbf{w}}_t^{\mathbf{P}}(c_t)-\mu\right]
\end{align}
as we have $\sum_{i=1}^{n}P_t^i + P_t^0 = 1$, it suffices to find $P_t^i$ for $i>0$. 

In the paper, they solved for the maximizer to be of the form
\begin{align}
(P_t^i)^\ast = \frac{\mu}{m_t(c_t)}\exp \left(\frac{q^i-\Delta^i \hat{\mathbf{w}}_t^{\mathbf{P}^\ast}(c_t)-u_0 - m_t(c_t)}{\mu}\right)\label{eq:maximizer}
\end{align}
where $i \in \mathbb{R}(c_t)$, the set of products with non-negative inventory and 
\begin{align}
m_t(c_t) = \frac{\mu}{1 - \sum_{i=1}^{n}(P_t^i)^\ast} = \frac{\mu}{P_t^{0\ast}}\label{eq:mtct}
\end{align}

By (\ref{eq:maximizer}), (\ref{eq:mtct}) and the probability simplex constrain, we obtain
\begin{align}
\left[\frac{m_t(c_t)}{\mu}-1\right] \exp \left[\frac{m_t(c_t)+u_0}{\mu}\right] = \sum_{i=1}^{n}\exp \left[\frac{q^i-\Delta^i \hat{\mathbf{w}}_t^{\mathbf{P}^\ast}(c_t)}{\mu}\right]\label{eq:uniquesol}
\end{align}
where $m_t(c_t)$ is the unique solution. This uniqueness follows from the concave property of $\Psi_t^{\mathbf{P}}(c_t)$ over all probability simplex $P_t$.

Using the results that we have derived above, the optimal price scheme is
\begin{align}
(r_t^i)^\ast = \Delta \hat{\mathbf{w}}_t^{\mathbf{P}^\ast}(c_t)+m_t(c_t)\label{eq:optpricescheme}
\end{align}

and updating the HJB optimality sufficiency conditions,
\begin{align}
\frac{\partial}{\partial t}\hat{\mathbf{w}}_t^{\mathbf{P}^\ast} = \lambda(t)(m_t(c_t) - \mu)\label{eq:pdw}
\end{align}
By integrating both sides with respect to $t$ from $0$ to $T$,  the maximum expected revenue at the beginning of the selling season in terms of $m_t(c_t)$ is 
\begin{align}
\hat{\mathbf{w}}_t^{\mathbf{P}^\ast} = \int_{0}^{T}\lambda(t)\left[m_t(c_t)-\mu\right]\,dt
\end{align}

From the discussion above, we see that the solution to the optimal pricing policy depends on the term $m_t(c_t)$. However, $m_t(c_t)$ is found both in the base and exponent of (\ref{eq:maximizer}) and thus there is no closed form solution for $m_t(c_t)$ and hence also $(r_t^i)^\ast$



\subsubsection{Our observations}

In our observations, contrary to (\ref{eq:maximizer}), we arrived at a different result
\begin{align}
(P_t^i)^\ast &= \frac{\mu}{m_t(c_t)}\exp \left(\frac{q^i-\Delta^i \hat{\mathbf{w}}_t^{\mathbf{P}^\ast}(c_t)-u_0 - \mu- m_t(c_t)(P_t^i)^\ast}{\mu}\right)\nonumber\\
&= P_t^{0\ast}\exp \left(\frac{q^i-\Delta^i \hat{\mathbf{w}}_t^{\mathbf{P}^\ast}(c_t)-u_0 - \mu - m_t(c_t)(P_t^i)^\ast}{\mu}\right)\label{eq:our_maximizer}
\end{align}
%and if we know the value of $\Delta^i \hat{\mathbf{w}}_t^{\mathbf{P}^\ast}(c_t)$ we can actually solve for the optimal probabilities $(P_t^i)^\ast$ by evaluating $x_i = \exp \left(\frac{q^i-\Delta^i \hat{\mathbf{w}}_t^{\mathbf{P}^\ast}(c_t)-u_0 - \mu}{\mu}\right)$ for all $i=1,\ldots, n$, then normalizing the vector
%\begin{align*}
%\left(1,x_1,\ldots, x_n\right)
%\end{align*}
%to get the vector of probabilities $\left(P_t^{0\ast},(P_t^{1})^\ast,\ldots,(P_t^{n})^\ast \right)$. 
%
However,  $\Delta^i \hat{\mathbf{w}}_t^{\mathbf{P}^\ast}(c_t)$ relies on $P_t^{0\ast}$ and $(P_t^i)^\ast$s, thus we do not have a closed form solution for (\ref{eq:our_maximizer}). Updating the equation in (\ref{eq:uniquesol}), we have the equation below.
\begin{align}
\left[\frac{m_t(c_t)}{\mu}-1\right] \exp \left[1+\frac{u_0}{\mu}\right] &= \sum_{i=1}^{n}\exp \left[\frac{q^i-\Delta^i \hat{\mathbf{w}}_t^{\mathbf{P}^\ast}(c_t)}{\mu}\right]\label{eq:our_uniquesol}
\end{align}








\section{Dynamic Programming For Optimal Dynamic Pricing}

=======
\begin{align}
\Psi_t^{\mathbf{P}}(c_t) = \sum_{i=1}^{n}\lambda(t)P_t^i\left[q^i-u_0-\mu \ln P_t^i + \mu \ln P_t^0- \Delta^i \hat{\mathbf{w}}_t^{\mathbf{P}}(c_t)\right]
\end{align}

By checking that the Hessian of $\Psi_t^{\mathbf{P}}(c_t)$ is negative for all $P_t$, $\Psi_t^{\mathbf{P}}(c_t)$ is concave in $P_t$. Thus solving
\begin{align*}
\frac{\partial \Psi_t^{\mathbf{P}}(c_t)}{\partial P_t^i} = 0 
\end{align*}
for $i=1,\ldots, n$ gives us the global optimum. As $\sum_{i=1}^{n}P_t^i+P_t^0=1$, solving $P_t^i$ for $i>0$ gives us the sales probability for optimal $\Psi_t^{\mathbf{P}}(c_t)$.
>>>>>>> 6bdf37781fdadad3618f86458f70c73ff87326ca

%\subsection{Summary of the Hamilton-Jacobi-Bellman Equation}
%The importance of the Hamilton-Jacobi-Bellman (HJB) equation is that it can be used as part of a strategy for solving the optimal control problem. Optimal control problems consist of some process that evolves over time, called a state, denoted by $x(\cdot)$, with the dynamics given by a stochastic differential equation (SDE). This process is then controllable to some extent by a person who wishes to optimize a performance indicator that depends on the process. The person does so by choosing a function $\alpha(\cdot)$, called a control, that takes valeus in a metric space $\Lambda$ and influences the dynamics of $x(\cdot)$ through the SDE. Different $\alpha(\cdot)$ leads to different states at time $t$, and hence different values for the performance indicator. The optimal control problem is then to answer the following questions:
%\begin{enumerate}
%\item Which control $\alpha(\cdot)$ optimizes the performance indicator?
%\item What value does the performance indicator take for the best controls?
%\end{enumerate}
%
%The strategy for solving the optimal control problem here is by using dynamic programming. However there are other methods like the Pontryagin maximum principle \annotation{find some reference here but do not present}.

%\section{Experiments}
%\annotation{Restoring this section back.}
%

\section{Discussions and Conclusions}
\annotation{This is where we can add our comments and our inputs, how the model can be further improved or how we can find estimates for the solution. Closing conclusions, futher areas that can be explored and research opportunities.}




\bibliographystyle{abbrv}
\bibliography{./bib}

