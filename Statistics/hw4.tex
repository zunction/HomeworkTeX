\documentclass[a4paper,10pt]{article}
\setlength{\parindent}{0cm}
\usepackage{amsmath, amssymb, amsthm, mathtools,pgfplots}
\usepackage{graphicx,caption}
\usepackage{verbatim}
\usepackage{venndiagram}
\usepackage[cm]{fullpage}
\usepackage{bbm}
\usepackage{fancyhdr}
\usepackage{tikz}
\usepackage{listings}
\usepackage{color,enumerate,framed}
\usepackage{color,hyperref}
\definecolor{darkblue}{rgb}{0.0,0.0,0.5}
\hypersetup{colorlinks,breaklinks,
            linkcolor=darkblue,urlcolor=darkblue,
            anchorcolor=darkblue,citecolor=darkblue}
\usepackage[utf8]{inputenc}

% Default fixed font does not support bold face
\DeclareFixedFont{\ttb}{T1}{txtt}{bx}{n}{10} % for bold
\DeclareFixedFont{\ttm}{T1}{txtt}{m}{n}{10}  % for normal

% Custom colors
\usepackage{color}
\definecolor{deepblue}{rgb}{0,0,0.5}
\definecolor{deepred}{rgb}{0.6,0,0}
\definecolor{deepgreen}{rgb}{0,0.5,0}

\usepackage{listings}

% Python style for highlighting
\newcommand\pythonstyle{\lstset{
language=Python,
basicstyle=\ttm,
otherkeywords={self},             % Add keywords here
keywordstyle=\ttb\color{deepblue},
emph={MyClass,__init__},          % Custom highlighting
emphstyle=\ttb\color{deepred},    % Custom highlighting style
stringstyle=\color{deepgreen},
frame=tb,                         % Any extra options here
showstringspaces=false            % 
}}


% Python environment
\lstnewenvironment{python}[1][]
{
\pythonstyle
\lstset{#1}
}
{}

% Python for external files
\newcommand\pythonexternal[2][]{{
\pythonstyle
\lstinputlisting[#1]{#2}}}

% Python for inline
\newcommand\pythoninline[1]{{\pythonstyle\lstinline!#1!}}
%\usepackage{tgadventor}
%\usepackage[nohug]{diagrams}
\usepackage[T1]{fontenc}
%\usepackage{helvet}
%\renewcommand{\familydefault}{\sfdefault}
%\usepackage{parskip}
%\usepackage{picins} %for \parpic.
%\newtheorem*{notation}{Notation}
%\newtheorem{example}{Example}[section]
%\newtheorem*{problem}{Problem}
\theoremstyle{definition}
%\newtheorem{theorem}{Theorem}
%\newtheorem*{solution}{Solution}
%\newtheorem*{definition}{Definition}
%\newtheorem{lemma}[theorem]{Lemma}
%\newtheorem{corollary}[theorem]{Corollary}
%\newtheorem{proposition}[theorem]{Proposition}
%\newtheorem*{remark}{Remark}
%\setcounter{section}{1}

\newtheorem{thm}{Theorem}[section]
\newtheorem{lemma}[thm]{Lemma}
\newtheorem{prop}[thm]{Proposition}
\newtheorem{cor}[thm]{Corollary}
\newtheorem{defn}[thm]{Definition}
\newtheorem*{examp}{Example}
\newtheorem{conj}[thm]{Conjecture}
\newtheorem{rmk}[thm]{Remark}
\newtheorem*{nte}{Note}
\newtheorem*{notat}{Notation}

%\diagramstyle[labelstyle=\scriptstyle]

\lstset{frame=tb,
  language=Oz,
  aboveskip=3mm,
  belowskip=3mm,
  showstringspaces=false,
  columns=flexible,
  basicstyle={\small\ttfamily},
  breaklines=true,
  breakatwhitespace=true,
  tabsize=3
}


\pagestyle{fancy}




\fancyhead{}
\renewcommand{\headrulewidth}{0pt}

\lfoot{\color{black!60}{\sffamily Zhangsheng Lai}}
\cfoot{\color{black!60}{\sffamily Last modified: \today}}
\rfoot{\color{black!60}{\sffamily\thepage}}



\begin{document}
\flushright{Zhangsheng Lai\\1002554}
\section*{Statistics: Homework 4}

\begin{enumerate}
\item 


\begin{enumerate}[(a)]
\item 
Given $p_i$ and $q_i$ denote the probability of choosing box 1 and 2 respectively if the ball color chosen is $i$ where $i=\{B, W ,G\}$, denoting the three different colors.  With the given information of the number of different color balls in the different boxes,
\begin{alignat*}{3}
\mathbb{P}(B|1)=4/10&\quad \mathbb{P}(B|2)=2/10&\quad \mathbb{P}(B|3)=4/10\\
\mathbb{P}(W|1)=3/10&\quad \mathbb{P}(W|2)=6/10&\quad \mathbb{P}(W|3)=1/10\\
\mathbb{P}(G|1)=2/10&\quad \mathbb{P}(G|2)=0&\quad \mathbb{P}(G|3)=8/10\\
\end{alignat*}
The risk function is represented by,
\begin{align*}
R(\theta,\hat{\theta}_{p,q}) &= \mathbb{E}_{\theta}(|\theta^2-\hat{\theta}_{p,q}|^2)\\
&=\mathbb{E}_{\theta}\left(\sum_{i\in\{B,W,G\}}L(\theta,\hat{\theta}_{p,q}(i))\mathbb{P}(i|\theta)\right)
\end{align*}
where $L(\theta,\hat{\theta}_{p,q}(i))=L(\theta,1)p_i+L(\theta,2)q_i+L(\theta,3)(1-p_i-q_i)$.Therefore,
\begin{align*}
R(1,\hat{\theta}_{p,q}) &= [q_B+4(1-p_B-q_B)]\frac{4}{10}+[q_W+4(1-p_W-q_W)]\frac{3}{10}+[q_G+4(1-p_G-q_G)]\frac{2}{10}\\
R(2,\hat{\theta}_{p,q}) &= [9p_B+4q_B+49(1-p_B-q_B)]\frac{2}{10}+[9p_W+4q_W+49(1-p_W-q_W)]\frac{6}{10}%+[9p_B+4q_B+49(1-p_B-q_B)]\frac{2}{10}
\end{align*}
\item Bayes risk is given by
\begin{align*}
r(f,\theta)=\int R(\theta,\hat{\theta}_{p,q})f(\theta)\,d\theta
\end{align*}
but since our scenario is discrete, we instead have
\begin{align*}
r(f,\theta) &= \sum_{\theta=1,2}R(\theta,\hat{\theta}_{p,q})\mathbb{P}(\theta)\\
&= \lambda R(1,\hat{\theta}_{p,q})+ (1-\lambda)R(2,\hat{\theta}_{p,q})
\end{align*}
where $R(1,\hat{\theta}_{p,q})$ and $R(2,\hat{\theta}_{p,q})$ are the values are from (a).
\item Given $\lambda = 1/2$, we have
\begin{align*}
r(f,\theta)=\frac{1}{2} \left(R(1,\hat{\theta}_{p,q})+ R(2,\hat{\theta}_{p,q})\right) &=\frac{1}{20}\left(428-96p_B-102q_B-252p_W-279q_W-8p_G-6q_G\right)
\end{align*}
thus to the infimum of Bayes risk is when $q_B = q_W = p_G =1$.
\end{enumerate}
\end{enumerate}

\end{document}