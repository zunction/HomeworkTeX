\documentclass[a4paper,10pt]{article}
\setlength{\parindent}{0cm}
\usepackage{amsmath, amssymb, amsthm, mathtools,pgfplots}
\usepackage{graphicx,caption}
\usepackage{verbatim}
\usepackage{venndiagram}
\usepackage[cm]{fullpage}
\usepackage{fancyhdr}
\usepackage{tikz}
\usepackage{listings}
\usepackage{color,enumerate,framed}
\usepackage{color,hyperref}
\definecolor{darkblue}{rgb}{0.0,0.0,0.5}
\hypersetup{colorlinks,breaklinks,
            linkcolor=darkblue,urlcolor=darkblue,
            anchorcolor=darkblue,citecolor=darkblue}

%\usepackage{tgadventor}
%\usepackage[nohug]{diagrams}
\usepackage[T1]{fontenc}
%\usepackage{helvet}
%\renewcommand{\familydefault}{\sfdefault}
%\usepackage{parskip}
%\usepackage{picins} %for \parpic.
%\newtheorem*{notation}{Notation}
%\newtheorem{example}{Example}[section]
%\newtheorem*{problem}{Problem}
\theoremstyle{definition}
%\newtheorem{theorem}{Theorem}
%\newtheorem*{solution}{Solution}
%\newtheorem*{definition}{Definition}
%\newtheorem{lemma}[theorem]{Lemma}
%\newtheorem{corollary}[theorem]{Corollary}
%\newtheorem{proposition}[theorem]{Proposition}
%\newtheorem*{remark}{Remark}
%\setcounter{section}{1}

\newtheorem{thm}{Theorem}[section]
\newtheorem{lemma}[thm]{Lemma}
\newtheorem{prop}[thm]{Proposition}
\newtheorem{cor}[thm]{Corollary}
\newtheorem{defn}[thm]{Definition}
\newtheorem*{examp}{Example}
\newtheorem{conj}[thm]{Conjecture}
\newtheorem{rmk}[thm]{Remark}
\newtheorem*{nte}{Note}
\newtheorem*{notat}{Notation}

%\diagramstyle[labelstyle=\scriptstyle]

\lstset{frame=tb,
  language=Oz,
  aboveskip=3mm,
  belowskip=3mm,
  showstringspaces=false,
  columns=flexible,
  basicstyle={\small\ttfamily},
  breaklines=true,
  breakatwhitespace=true,
  tabsize=3
}


\pagestyle{fancy}




\fancyhead{}
\renewcommand{\headrulewidth}{0pt}

\lfoot{\color{black!60}{\sffamily Zhangsheng Lai}}
\cfoot{\color{black!60}{\sffamily Last modified: \today}}
\rfoot{\textsc{\thepage}}



\begin{document}
\flushright{Zhangsheng Lai\\1002554}
\section*{Algebraic Geometry: Homework 1}

\begin{enumerate}[1.]
\item $R$ be a ring and $S$ a multiplicative subset of $R$ with $1 \in S$ and $0 \notin S$
\begin{enumerate}[(i)]
\item It is reflexive, since for any $t \in S$, $t(rs-rs)=0$, thus $(r,s) \sim (r,s)$. Suppose $(r,s) \sim (r',s')$, so there exist $t \in S$ such that $t(rs'-r's)=0$ which also means $t(r's-rs')=0$ and we have symmetry. Lastly, let $(r,s)\sim(t,u)$ and $(t,u)\sim(v,w)$, then there exist $a, b \in S$ such that 
\begin{align*}
a(ru-ts)&=0\\
b(tw-vu)&=0
\end{align*}
Then $abwru-abwts=0$, $bastw-basvu=0$ and summing them gives $abu(rw-vs)=0$ with $abu \in S$ which shows transitivity.
\item Same as part (i).
\item Define the action of $R_s$ on $M_s$, $\phi:R_s \times M_s \to M_s$, $((r,s),(m,s'))\mapsto (rm,ss')$. We first show that the action is well-defined. Let $(r_1,s_1)\sim (r_2,s_2) \in R_s$ and $(m_1,t_1)\sim (m_1,t_1) \in M_s$. Then $\phi((r_i,s_i),(m_i,t_i)) = (r_im_i,s_it_i)$ for $i=1,2$. Since $a(r_{1}s_2-r_2s_{1})=0$ and $b(m_{1}t_2-m_2t_{1})=0$ with some $a, b \in S$,
\begin{align*}
ab(r_1s_2m_1t_2-r_2s_1m_1t_2)=0\\
ab(r_2s_1m_{1}t_2-r_2s_1m_2t_{1})=0\\
\end{align*}
thus we have $ab(r_1m_1s_2t_2 -r_2m_2s_1t_1 )$, so $(r_1m_1,s_1t_1)\sim(r_2m_2,s_2t_2)$. $M_s$ is a $R_s$-module, since for $m_i \in M$, $r_i \in R$ and $s_i, t_i \in S$,
\begin{align*}
\bullet &(r,s) ((m_1,t_1)+(m_2,t_2)) = (r,s)((m_1t_2+m_2t_1,t_1t_2))=(rm_1t_2+rm_2t_1,st_1t_2)=(rm_1,st_1)+(rm_2,st_2)\\
\bullet &((r_1,s_1)+(r_2,s_2))(m,t) = (mr_1s_2+mr_2s_1,s_1s_2t) = (r_1m,s_1t) + (r_2m,s_2t)\\
\bullet &((r_1,s_1)(r_2,s_2))(m,t) = (r_1r_2m,s_1s_2t) = (r_1,s_1)(r_2m,s_2t)= ((r_1,s_1)(r_2,s_2))(m,t)\\
\bullet &(1_R,1_S)(m,t) = (m,t)
\end{align*}
\end{enumerate}
\item Given morphisms of $R$-modules, $\phi:M \to N$ and $\psi:N\to P$, it is an \emph{exact sequence} if the image of $\phi$ is equal to the kernel of $\psi$ in $N$.
\begin{enumerate}[(i)]
\item If $M=0_M$, then $\phi(0_M) = \{0_N\} = \ker(\psi)$, thus $\psi$ is injective.
\item If $P=0_P$, then $\phi(M) = \ker(\psi) = N$, thus $\phi$ is surjective.
\item For a prime ideal $\mathfrak{p}$ with $S = R - \mathfrak{p}$, and $R_S = R_\mathfrak{p}$, $M_S = M_\mathfrak{p}$. Let $p(r,s) \in \mathfrak{p}R_{\mathfrak{p}}$ with $p \in \mathfrak{p}$ and $(r,s) \in R_{\mathfrak{p}}$. It is an ideal since, $p(r,s)(r',s')=p(rr',ss') \in \mathfrak{p}R_{\mathfrak{p}}$. 


$\mathfrak{p}$ does not contain any unit of $R$, else $\mathfrak{p} = R$ and $S = \varnothing$. Thus for $(p,s) \in \mathfrak{p}R_{\mathfrak{p}}$, it is not a unit in $R_{\mathfrak{p}}$ and thus it is a proper ideal. It is maximal since for any $(r,s) \in R_{\mathfrak{p}}-\mathfrak{p}R_{\mathfrak{p}}$, it is a unit of $R_{\mathfrak{p}}$ since $r \in S$ and its inverse is $(s,r)$. Thus $\mathfrak{p}R_{\mathfrak{p}}+((r,s)) = R_{\mathfrak{p}}$ for any $(r,s) \in R_{\mathfrak{p}}-\mathfrak{p}R_{\mathfrak{p}}$

\item Given the natural maps $\phi_\mathfrak{p}:M_{\mathfrak{p}} \to N_{\mathfrak{p}}$ and $\psi_\mathfrak{p}:N_{\mathfrak{p}} \to P_{\mathfrak{p}}$ given by $(m,s) \mapsto (\phi(m),s)$ and $(n,s) \mapsto (\psi(n),s)$. Let $(m_1,s_1) \sim (m_2,s_2)$ thus $t(m_1s_2-m_2s_1)=0_M$ for some $t \in S$. Then $\phi(t(m_1s_2-m_2s_1))=t(\phi(m_1)s_2-\phi(m_2)s_1)=\phi(0_M)=0_N$. Thus $(\phi(m_1),s_1)\sim(\phi(m_2),s_2)$. Similar argument for $\psi_{\mathfrak{p}}$. 
\item Let $(\phi(m),s) \in \phi_{\mathfrak{p}}(M_{\mathfrak{p}})$, then $(\psi_{\mathfrak{p}}\circ\phi_{\mathfrak{p}}(m),s)$
\end{enumerate}
\item
\begin{enumerate}[(i)]
\item
\end{enumerate}
\item
\begin{enumerate}[(i)]
\item
\end{enumerate}
\item
\begin{enumerate}[(i)]
\item Let $S$ be any subset of $k[\underline{x}]$.



Let $S \subseteq k[\underline{x}]$, then $V(S)=\{p \in A^n|F(p)=0,~ \forall p \in S\}$. Thus $I(V(S))=\{F \in k[\underline{x}]|F(\underline{a})=0,~\forall \underline{a}\in V(S)\}$.

$I(V(S)) \supseteq S$
\item
\end{enumerate}
\item
\begin{enumerate}[(i)]
\item
\end{enumerate}
\item
\begin{enumerate}[(i)]
\item
\end{enumerate}


\end{enumerate}

\end{document}