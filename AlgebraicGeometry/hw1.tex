\documentclass[a4paper,10pt]{article}
\setlength{\parindent}{0cm}
\usepackage{amsmath, amssymb, amsthm, mathtools,pgfplots}
\usepackage{graphicx,caption}
\usepackage{verbatim}
\usepackage{venndiagram}
\usepackage[cm]{fullpage}
\usepackage{fancyhdr}
\usepackage{tikz}
\usepackage{listings}
\usepackage{color,enumerate,framed}
\usepackage{color,hyperref}
\definecolor{darkblue}{rgb}{0.0,0.0,0.5}
\hypersetup{colorlinks,breaklinks,
            linkcolor=darkblue,urlcolor=darkblue,
            anchorcolor=darkblue,citecolor=darkblue}

%\usepackage{tgadventor}
%\usepackage[nohug]{diagrams}
\usepackage[T1]{fontenc}
%\usepackage{helvet}
%\renewcommand{\familydefault}{\sfdefault}
%\usepackage{parskip}
%\usepackage{picins} %for \parpic.
%\newtheorem*{notation}{Notation}
%\newtheorem{example}{Example}[section]
%\newtheorem*{problem}{Problem}
\theoremstyle{definition}
%\newtheorem{theorem}{Theorem}
%\newtheorem*{solution}{Solution}
%\newtheorem*{definition}{Definition}
%\newtheorem{lemma}[theorem]{Lemma}
%\newtheorem{corollary}[theorem]{Corollary}
%\newtheorem{proposition}[theorem]{Proposition}
%\newtheorem*{remark}{Remark}
%\setcounter{section}{1}

\newtheorem{thm}{Theorem}[section]
\newtheorem{lemma}[thm]{Lemma}
\newtheorem{prop}[thm]{Proposition}
\newtheorem{cor}[thm]{Corollary}
\newtheorem{defn}[thm]{Definition}
\newtheorem*{examp}{Example}
\newtheorem{conj}[thm]{Conjecture}
\newtheorem{rmk}[thm]{Remark}
\newtheorem*{nte}{Note}
\newtheorem*{notat}{Notation}

%\diagramstyle[labelstyle=\scriptstyle]

\lstset{frame=tb,
  language=Oz,
  aboveskip=3mm,
  belowskip=3mm,
  showstringspaces=false,
  columns=flexible,
  basicstyle={\small\ttfamily},
  breaklines=true,
  breakatwhitespace=true,
  tabsize=3
}


\pagestyle{fancy}




\fancyhead{}
\renewcommand{\headrulewidth}{0pt}

\lfoot{\color{black!60}{\sffamily Zhangsheng Lai}}
\cfoot{\color{black!60}{\sffamily Last modified: \today}}
\rfoot{\textsc{\thepage}}



\begin{document}
\flushright{Zhangsheng Lai\\1002554}
\section*{Algebraic Geometry: Homework 1}

\begin{enumerate}[1.]
\item $R$ be a ring and $S$ a multiplicative subset of $R$ with $1 \in S$ and $0 \notin S$
\begin{enumerate}[(i)]
\item It is reflexive, since for any $t \in S$, $t(rs-rs)=0$, thus $(r,s) \sim (r,s)$. Suppose $(r,s) \sim (r',s')$, so there exist $t \in S$ such that $t(rs'-r's)=0$ which also means $t(r's-rs')=0$ and we have symmetry. Lastly, let $(r,s)\sim(t,u)$ and $(t,u)\sim(v,w)$, then there exist $a, b \in S$ such that 
\begin{align*}
a(ru-ts)&=0\\
b(tw-vu)&=0
\end{align*}
Then $abwru-abwts=0$, $bastw-basvu=0$ and summing them gives $abu(rw-vs)=0$ with $abu \in S$ which shows transitivity.
\item Same as part (i).
\item Define the action of $R_s$ on $M_s$, $\phi:R_s \times M_s \to M_s$, $((r,s),(m,s'))\mapsto (rm,ss')$. We first show that the action is well-defined. Let $(r_1,s_1)\sim (r_2,s_2) \in R_s$ and $(m_1,t_1)\sim (m_1,t_1) \in M_s$. Then $\phi((r_i,s_i),(m_i,t_i)) = (r_im_i,s_it_i)$ for $i=1,2$. Since $a(r_{1}s_2-r_2s_{1})=0$ and $b(m_{1}t_2-m_2t_{1})=0$ with some $a, b \in S$,
\begin{align*}
ab(r_1s_2m_1t_2-r_2s_1m_1t_2)=0\\
ab(r_2s_1m_{1}t_2-r_2s_1m_2t_{1})=0\\
\end{align*}
thus we have $ab(r_1m_1s_2t_2 -r_2m_2s_1t_1 )$, so $(r_1m_1,s_1t_1)\sim(r_2m_2,s_2t_2)$. $M_s$ is a $R_s$-module, since for $m_i \in M$, $r_i \in R$ and $s_i, t_i \in S$,
\begin{align*}
\bullet &(r,s) ((m_1,t_1)+(m_2,t_2)) = (r,s)((m_1t_2+m_2t_1,t_1t_2))=(rm_1t_2+rm_2t_1,st_1t_2)=(rm_1,st_1)+(rm_2,st_2)\\
\bullet &((r_1,s_1)+(r_2,s_2))(m,t) = (mr_1s_2+mr_2s_1,s_1s_2t) = (r_1m,s_1t) + (r_2m,s_2t)\\
\bullet &((r_1,s_1)(r_2,s_2))(m,t) = (r_1r_2m,s_1s_2t) = (r_1,s_1)(r_2m,s_2t)= ((r_1,s_1)(r_2,s_2))(m,t)\\
\bullet &(1_R,1_S)(m,t) = (m,t)
\end{align*}
\end{enumerate}
\item Given morphisms of $R$-modules, $\phi:M \to N$ and $\psi:N\to P$, it is an \emph{exact sequence} if the image of $\phi$ is equal to the kernel of $\psi$ in $N$.
\begin{enumerate}[(i)]
\item If $M=0_M$, then $\phi(0_M) = \{0_N\} = \ker(\psi)$, thus $\psi$ is injective.
\item If $P=0_P$, then $\phi(M) = \ker(\psi) = N$, thus $\phi$ is surjective.
\item For a prime ideal $\mathfrak{p}$ with $S = R - \mathfrak{p}$, and $R_S = R_\mathfrak{p}$, $M_S = M_\mathfrak{p}$. Let $p(r,s) \in \mathfrak{p}R_{\mathfrak{p}}$ with $p \in \mathfrak{p}$ and $(r,s) \in R_{\mathfrak{p}}$. It is an ideal since, $p(r,s)(r',s')=p(rr',ss') \in \mathfrak{p}R_{\mathfrak{p}}$. 


$\mathfrak{p}$ does not contain any unit of $R$, else $\mathfrak{p} = R$ and $S = \varnothing$. Thus for $(p,s) \in \mathfrak{p}R_{\mathfrak{p}}$, it is not a unit in $R_{\mathfrak{p}}$ and thus it is a proper ideal. It is maximal since for any $(r,s) \in R_{\mathfrak{p}}-\mathfrak{p}R_{\mathfrak{p}}$, it is a unit of $R_{\mathfrak{p}}$ since $r \in S$ and its inverse is $(s,r)$. Thus $\mathfrak{p}R_{\mathfrak{p}}+((r,s)) = R_{\mathfrak{p}}$ for any $(r,s) \in R_{\mathfrak{p}}-\mathfrak{p}R_{\mathfrak{p}}$

\item Given the natural maps $\phi_\mathfrak{p}:M_{\mathfrak{p}} \to N_{\mathfrak{p}}$ and $\psi_\mathfrak{p}:N_{\mathfrak{p}} \to P_{\mathfrak{p}}$ given by $(m,s) \mapsto (\phi(m),s)$ and $(n,s) \mapsto (\psi(n),s)$. Let $(m_1,s_1) \sim (m_2,s_2)$ thus $t(m_1s_2-m_2s_1)=0_M$ for some $t \in S$. Then $\phi(t(m_1s_2-m_2s_1))=t(\phi(m_1)s_2-\phi(m_2)s_1)=\phi(0_M)=0_N$. Thus $(\phi(m_1),s_1)\sim(\phi(m_2),s_2)$. Similar argument for $\psi_{\mathfrak{p}}$. 
\item Let $(\phi(m),s) \in \phi_{\mathfrak{p}}(M_{\mathfrak{p}})$, then $(\psi_{\mathfrak{p}}\circ\phi_{\mathfrak{p}}(m),s)$
\end{enumerate}

\item
$R_{\mathfrak{p}}$


\item

\item
\begin{enumerate}[(i)]
\item Let $S$ be any subset of $k[x_1,\ldots,x_n]$. For every $\mathbf{a} \in V(S)$, $F(\mathbf{a}) = 0$ for every $F \in S$. Thus $F \in I(V(S))$ for every $F \in S$ and $S \subseteq I(V(S))$.



\item $V(I(V(S))) = V(S)$
\end{enumerate}
\item
\begin{enumerate}[(i)]
\item Let $X \subseteq A^n(k)$, then $I(X):=\{F \in k[x_1,\ldots,x_n]|F(\mathbf{a})=0 \text{ for all } \mathbf{a} \in X\}$. For each $F^n \in I(X)$ where $n>0$ integer, suppose $F \notin I(X)$, i.e. there exists $\mathbf{a} \in X $ such that $F(\mathbf{a}) \neq 0$. But $(F(\mathbf{a}))^n = F^n(\mathbf{a})=0$ which implies $F(\mathbf{a}) \in k$ is a zero divisor, a contradiction since fields do not have zero divisors. Thus $F^n \in I(X)$ and shows $I(X) \supseteq \text{Rad}(I(X))$. The other containment is obvious since for $F \in I(X), F \in\text{Rad}(I(X))$ by choosing $n=1$.

\item Let $X \subseteq A^n(k)$, then for any $F \in I(X)$, $F(\mathbf{a})=0$ for all $\mathbf{a} \in X$, thus $\mathbf{a} \in V(I(X))$, thus $V(I(X)) \supseteq X$.



\item Since $I(X) = I$ is a radical ideal, by Hilbert's Nullstellensatz, $I(V(I)) = \text{Rad}(I) = I$.
\end{enumerate}
\item
\begin{enumerate}[(i)]
\item Let $J$ be an ideal of $R$ and $\pi(J):=\{j + I | j \in J\}$. Thus for $r + I 
\in R/I, j+I \in \pi(J)$, $(j+I)(r+I) = jr +I$ and $(r+I)(j+I) = rj +I$ are both in $\pi(J)$ since $rj, jr \in J$. It is easy to see that $\pi(J)$ is closed under addition, and $\pi(J)$ is an ideal of $R/I$.

\item Let $J'$ be an ideal of $R/I$ and $\pi^{-1}(J'):=\{j \in R | j+I \in J'\}$. Let $j,j' \in \pi^{-1}(J')$, then $j-j' \in \pi^{-1}(J')$ as $(j-j') + I \in J'$. Also for $r \in R, j \in \pi^{-1}(J')$, $rj, jr \in \pi^{-1}(J')$ as $rj +I, jr +I \in J'$. Thus $\pi^{-1}(J')$ is an ideal in $R$. $J \supseteq I$ as $0_{R/I} \in J'$ and thus $\pi^{-1}(0_{R/I}) \supseteq I$.

\item To show the bijection, we have to show that $\pi \circ \pi^{-1} = 1_{R/I}$ and $\pi^{-1} \circ \pi = 1_{R}$. For $J' \subseteq R/I$ ideal, $\pi^{-1}(J'):=\{j \in R | j+I \in J'\}$, thus $\pi(j) = j+I \in J'$ for $j \in \pi^{-1}(J')$ and so $\pi \circ \pi^{-1}(J') =  J'$. Now let $J \subseteq R$ ideal, $\pi(J):=\{j+I|j\in J\}$ and so $\pi^{-1}(j+I)=j \in J$ so $\pi^{-1}\circ \pi = 1_R$.

Since we have a one to one correspondence between $\{\text{Ideals }J \supseteq I\}$ and $\{\text{Ideals }J' \subseteq R/I\}$, if ...

\item Let $J'$ be a radical ideal, i.e. $J' = \text{Rad}(J'):=\{r +I\in R/I | r^n+I \in J' \text{ for some integer } n > 0\}$. Then $J = \pi^{-1}(J'):=\{j \in R | j+I \in J'\}$ ideal. Take $j^n \in J$ for some integer $n>0$ and $\pi(j^n) = j^n+ I \in J'$, which also implies $j +I \in J'$ since $J'$ is a radical ideal. Thus $j = \pi^{-1}(j+I) \in J$, i.e. $J$ is a radical ideal. Conversely, let $J\subseteq R$ radical ideal. Then for $j^n+I \in J'$, $\pi^{-1}(j^n+I) = j^n \in J$ and so $j \in J$. Hence, $\pi(j) = j+I \in J'$ which proves $J'$ is a radical ideal.
\\
Let $J'$ be a prime ideal, then for $ab+I\in J'$, either $a+I$ or $b+I$ is in $J'$. Let $cd \in \pi^{-1}(J')$ ideal, then $\pi(cd) = cd +I \in J'$, thus $c +I$ or $d+I$ is in $J'$. Thus $\pi^{-1}(c +I) = c \in J$ or $\pi^{-1}(d +I) = d \in J$, thus $\pi^{-1}(J')$ is also a prime ideal. Conversely, let $J\subseteq R$ prime ideal. Take $ab+I\in J'$, then $ab = \pi^{-1}(ab+I) \in J$. Thus $a \in J$ or $b \in J$ and we have $a+I \in J'$ or $b+I \in J'$. Thus $J'$ is a prime ideal.
\\The proof for maximal ideals follows from the result from (iii). Suppose $J'$ is maximal, thus for $J'\subseteq K' \subseteq R/I$, $K' = J'$ or $K' = R/I$. So if $J = \pi^{-1}(J')$ is not maximal, there exists a $K \neq J, R$ such that $J \subseteq K \subseteq R$, then $\pi(K)$ is an ideal in $R/I$ that will contradict the maximality of $J'$. The converse direction follows the same argument.
\end{enumerate}


\end{enumerate}

\end{document}