\documentclass[a4paper,10pt]{article}
\setlength{\parindent}{0cm}
\usepackage{amsmath, amssymb, amsthm, mathtools,pgfplots}
\usepackage{graphicx,caption}
\usepackage{verbatim}
\usepackage{venndiagram}
\usepackage[cm]{fullpage}
\usepackage{fancyhdr}
\usepackage{tikz}
\usepackage{listings}
\usepackage{color,enumerate,framed}
\usepackage{color,hyperref}
\definecolor{darkblue}{rgb}{0.0,0.0,0.5}
\hypersetup{colorlinks,breaklinks,
            linkcolor=darkblue,urlcolor=darkblue,
            anchorcolor=darkblue,citecolor=darkblue}

%\usepackage{tgadventor}
%\usepackage[nohug]{diagrams}
\usepackage[T1]{fontenc}
%\usepackage{helvet}
%\renewcommand{\familydefault}{\sfdefault}
%\usepackage{parskip}
%\usepackage{picins} %for \parpic.
%\newtheorem*{notation}{Notation}
%\newtheorem{example}{Example}[section]
%\newtheorem*{problem}{Problem}
\theoremstyle{definition}
%\newtheorem{theorem}{Theorem}
%\newtheorem*{solution}{Solution}
%\newtheorem*{definition}{Definition}
%\newtheorem{lemma}[theorem]{Lemma}
%\newtheorem{corollary}[theorem]{Corollary}
%\newtheorem{proposition}[theorem]{Proposition}
%\newtheorem*{remark}{Remark}
%\setcounter{section}{1}

\newtheorem{thm}{Theorem}[section]
\newtheorem{lemma}[thm]{Lemma}
\newtheorem{prop}[thm]{Proposition}
\newtheorem{cor}[thm]{Corollary}
\newtheorem{defn}[thm]{Definition}
\newtheorem*{examp}{Example}
\newtheorem{conj}[thm]{Conjecture}
\newtheorem{rmk}[thm]{Remark}
\newtheorem*{nte}{Note}
\newtheorem*{notat}{Notation}

%\diagramstyle[labelstyle=\scriptstyle]

\lstset{frame=tb,
  language=Oz,
  aboveskip=3mm,
  belowskip=3mm,
  showstringspaces=false,
  columns=flexible,
  basicstyle={\small\ttfamily},
  breaklines=true,
  breakatwhitespace=true,
  tabsize=3
}


\pagestyle{fancy}




\fancyhead{}
\renewcommand{\headrulewidth}{0pt}

\lfoot{\color{black!60}{\sffamily Zhangsheng Lai}}
\cfoot{\color{black!60}{\sffamily Last modified: \today}}
\rfoot{\color{black!60}{\sffamily\thepage}}



\begin{document}
\flushright{Zhangsheng Lai\\1002554}
\section*{Problems to Ponder}

\begin{enumerate}
\item Let $(\Omega, \mathcal{F}, \mathbb{P})$ be a given probability space with $A_n \uparrow A$ and $B_n \downarrow B$ where $A_n, B_n \in \mathcal{F}$ for $n \geq 1$. WLOG, consider $A_n \uparrow A$, then since $A = \bigcup_nA_n$, $A \in \mathcal{F}$. By considering the complements of $B_n^c = H_n$, $H_n \uparrow H = \bigcup_n B_n^c \in \mathcal{F}$ implies $\bigcap_nB_n \in \mathcal{F}$. Use countably additive property to prove $\lim_{n\to\infty}\mathbb{P}(A_n)$. 

\item We are given $\Omega$ to be infinite, $\mathcal{F} = \{A \subset \Omega: A \text{ is finite or } A^c \text{ is finite}\}$ and $\mu: \mathcal{F}\to [0,\infty)$ by $\mu(A)=0$ if $A$ is finite and $\mu(A) = 1$ if $A^c$ is finite. We shall show that $\mathcal{F}$ is an algebra,
\begin{enumerate}[(a)]
\item $\varnothing, \Omega \in \mathcal{F}$ since $\varnothing = \Omega^c$ is finite. 
\item For $A \in \mathcal{F}$ either $A$ or $A^c$ is finite, thus $A^c \in \mathcal{F}$.
\item For $A, B \in \mathcal{F}$, if both are finite, $A \cup B \in \mathcal{F}$. If both $A^c, B^c$ are finite, $(A \cup B)^c = A^c \cap B^c \in \mathcal{F}$. If $A, B^c$ is finite, $(A \cup B)^c \subset B^c$, thus $A \cup B$ is finite.
\end{enumerate}
Therefore, $\mathcal{F}$ is an algebra. To show that $\mu$ is finitely additive on $\mathcal{F}$, we have to show that for $A_k \in \mathcal{F}$, $k \geq 1$ and pairwise disjoint, $\mu(\cup_{k=1}^nA_k) = \sum_{i=1}^n\mu(A_k)$. This trivially holds when the $A_k$'s are finite. When we have $A_k$ infinite, we can have at most one since if $A_k, A_j$ are infinite, since they are disjoint, $A_j \subset A_k^c$ which is a contradiction. Thus for any collection of disjoint $A_k$'s there is at most one of them which is infinite and thus it satisfies the finite additivity property.

\item Let $\mu$ be the Lebesgue probability measure on the $\sigma$-algebra $\mathcal{F}$ of $[0,1]$ and $\mu_\ast$ the outer measure on all subsets of $[0,1]$ defined to be $\mu_\ast(A)=\inf \{\sum_{k=1}^{\infty}|I_k|; \text{ each }I_k \text{ is an interval and }\{I_k\} \text{ is a countable cover for } A\}$. We also define $\mathcal{B}$ to be the Borel $\sigma$-algebra on $[0,1]$ and $N$ is called a null set if $\mu_\ast(N)=0$ where $N \subset [0,1]$. We want to show that $\overline{\mathcal{B}}:=\{B \cup N: B \in \mathcal{B}\}=\mathcal{F}$.
\end{enumerate}

\end{document}