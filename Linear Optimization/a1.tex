\documentclass[a4paper,10pt]{article}
\setlength{\parindent}{0cm}
\usepackage{amsmath, amssymb, amsthm, mathtools,pgfplots}
\usepackage{graphicx,caption}
\usepackage{verbatim}
\usepackage{venndiagram}
\usepackage[cm]{fullpage}
\usepackage{fancyhdr}
\usepackage{tikz}
\usepackage{listings}
\usepackage{color,enumerate,framed}
\usepackage{color,hyperref}
\definecolor{darkblue}{rgb}{0.0,0.0,0.5}
\hypersetup{colorlinks,breaklinks,
            linkcolor=darkblue,urlcolor=darkblue,
            anchorcolor=darkblue,citecolor=darkblue}
\pgfplotsset{compat=1.12}
%\usepackage{tgadventor}
%\usepackage[nohug]{diagrams}
\usepackage[T1]{fontenc}
%\usepackage{helvet}
%\renewcommand{\familydefault}{\sfdefault}
%\usepackage{parskip}
%\usepackage{picins} %for \parpic.
%\newtheorem*{notation}{Notation}
%\newtheorem{example}{Example}[section]
%\newtheorem*{problem}{Problem}
\theoremstyle{definition}
%\newtheorem{theorem}{Theorem}
%\newtheorem*{solution}{Solution}
%\newtheorem*{definition}{Definition}
%\newtheorem{lemma}[theorem]{Lemma}
%\newtheorem{corollary}[theorem]{Corollary}
%\newtheorem{proposition}[theorem]{Proposition}
%\newtheorem*{remark}{Remark}
%\setcounter{section}{1}

\newtheorem{thm}{Theorem}[section]
\newtheorem{lemma}[thm]{Lemma}
\newtheorem{prop}[thm]{Proposition}
\newtheorem{cor}[thm]{Corollary}
\newtheorem{defn}[thm]{Definition}
\newtheorem*{examp}{Example}
\newtheorem{conj}[thm]{Conjecture}
\newtheorem{rmk}[thm]{Remark}
\newtheorem*{nte}{Note}
\newtheorem*{notat}{Notation}

%\diagramstyle[labelstyle=\scriptstyle]

\lstset{frame=tb,
  language=Oz,
  aboveskip=3mm,
  belowskip=3mm,
  showstringspaces=false,
  columns=flexible,
  basicstyle={\small\ttfamily},
  breaklines=true,
  breakatwhitespace=true,
  tabsize=3
}


\pagestyle{fancy}




\fancyhead{}
\renewcommand{\headrulewidth}{0pt}

\lfoot{\color{black!60}{\sffamily Zhangsheng Lai}}
\cfoot{\color{black!60}{\sffamily Last modified: \today}}
\rfoot{\color{black!60}{\sffamily\thepage}}



%Model Formulation: 1. BHM  1.17     2. BHM  1.18      3. BHM  1.20      4. BHM  1.25
%
%Simplex Algorithm: 5. BHM  2.7     6. BHM  2.8     7. BHM  2.9      8. BHM  2.12     9. BHM 2.13
%
%Sensitivity Analysis: 10. BHM  3.15 (typo in final tableau: 15 should be 1.5)     
%                         11. BHM  3.22 (typos in problem statement: $5000  $500, thousands  hundreds;
%                                                             and in diagram: Minimize  Maximize)


\begin{document}
\flushright{Zhangsheng Lai\\1002554}
\section*{Linear Optimization: Assignment 1}
\begin{align*}
  \begin{array}{r@{}r@{}r@{}l}
    \text{max} \quad z=x_1 &{} + 12x_2 \\[\jot]
    \text{s.t.}\qquad 3x_1 &{} + \phantom{12}x_2 &{} + 12x_3 &{} \leq 5 \\                          x_1 &         &{} +   \phantom{12}x_3 &{} \leq 16 \\                        15x_1 &{} + \phantom{12}x_2 &           &{} = 14 \\         
 \multicolumn{4}{c}{x_j \geq 0, \quad j=1,2,3.}
  \end{array}
\end{align*}

\begin{enumerate}
\item[1.17]
\begin{enumerate}[(a)]
\item
\begin{align*}
  \begin{array}{r@{}r@{}r@{}r@{}l}
    \text{min} \quad c_1x_1 &{} + c_2x_2&{}+ c_3x_3&{}+ c_4x_4&{} \\[\jot]
    \text{s.t.}\qquad x_1 &{} +\phantom{c_2}x_2 &{} + \phantom{c_3}x_3 &{}+\phantom{c_3} x_4&{}  \geq K\\ [\jot]
    x_1 &{} +\phantom{c_2}x_2 &{} + \phantom{c_3}x_3 &{}+\phantom{c_3} x_4&{}  \leq M\\ [\jot]
P_1x_1 &{}+P_2x_2 &{}+P_3x_3 &{}+P_4x_4 &{}\leq PM\\        [\jot]          
N_1x_1 &{}+N_2x_2 &{}+N_3x_3 &{}+N_4x_4 &{}\leq NM\\    [\jot]              
 \multicolumn{4}{c}{x_j \geq 0, \quad j=1,\ldots,4}
  \end{array}
\end{align*}
\item
\begin{align*}
  \begin{array}{r@{}r@{}r@{}r@{}l}
    \text{min} \quad c_1x_1 &{} + c_2x_2&{}+ c_3x_3&{}+ c_4x_4&{} \\[\jot]
    \text{s.t.}\qquad x_1 &{} +\phantom{c_2}x_2 &{} + \phantom{c_3}x_3 &{}+\phantom{c_3} x_4&{}  \geq K\\ [\jot]
    x_1 &{} +\phantom{c_2}x_2 &{} + \phantom{c_3}x_3 &{}+\phantom{c_3} x_4&{}  \leq M\\ [\jot]
(P_1-P)x_1 &{}+(P_2-P)x_2 &{}+(P_3-P)x_3 &{}+(P_4-P)x_4 &{}\leq 0\\        [\jot]          
(N_1x-N)_1 &{}+(N_2-N)x_2 &{}+(N_3-N)x_3 &{}+(N_4-N)x_4 &{}\leq 0\\    [\jot]              
 \multicolumn{4}{c}{x_j \geq 0, \quad j=1,2,3,4}
  \end{array}
\end{align*}
\end{enumerate}
\item[1.18]
\begin{enumerate}[(a)]
\item
\begin{align*}
  \begin{array}{r@{}r@{}r@{}l}
    \text{min} \quad \sum_{i=1}^{4}c_ix_{1,i} &{} + \sum_{i=1}^{4}c_ix_{2,i}&{}+ \sum_{i=1}^{4}c_ix_{3,i}&{} \\[\jot]
    \text{s.t.}\qquad \sum_{i=1}^{4}x_{1,i} &{} &{} &{}   \geq K_A\\ [\jot]
  &{}  \ \sum_{i=1}^{4}x_{2,i}  &{} &{}   \geq K_B\\ [\jot]
 &{} &{}    \sum_{i=1}^{4}x_{3,i}  &{}   \geq K_C\\ [\jot]
 \sum_{i=1}^{4}x_{1,i} &{} &{} &{}   \leq M_1\\ [\jot]
  &{}  \ \sum_{i=1}^{4}x_{2,i}  &{} &{}   \leq M_2\\ [\jot]
 &{} &{}    \sum_{i=1}^{4}x_{3,i}  &{}   \leq M_1+M_2\\ [\jot]
 \sum_{i=1}^{4}P_ix_{1,i} &{} &{} &{}   \geq P_SM_1\\ [\jot]
  &{}  \sum_{i=1}^{4}P_ix_{2,i}  &{} &{}   \geq P_BM_2\\ [\jot]
 &{} &{}   \sum_{i=1}^{4}P_i(x_{1,i}+x_{2,i}+x_{3,i})  &{}   \geq P_S(M_1+M_2)\\ [\jot]
 \sum_{i=1}^{4}N_ix_{1,i} &{} &{} &{}   \geq N_SM_1\\ [\jot]
  &{}   \sum_{i=1}^{4}N_ix_{2,i}  &{} &{}   \geq N_BM_2\\ [\jot]
 &{} &{}    \sum_{i=1}^{4}N_i(x_{1,i}+x_{2,i}+x_{3,i})  &{}   \geq N_S(M_1+M_2)\\ [\jot]               
 \multicolumn{4}{c}{x_{i,j} \geq 0, \quad i=1,2,3, j=1,2,3,4}
  \end{array}
\end{align*}

\item The $c_i$'s, $P_i$'s and $N_i$'s will unique for each plant thus we will have $c_{p,i}$'s, $P_{p,i}$ and $N_{p,i}$ for $p \in \{A, B, C\}$.
\end{enumerate}
\item[1.20] We shall let $t$ denote the $t+6 \mod 12$ month of the year, i.e. $t=0$ is June and $t=10$ is April. Let $x_t = x_t^+-x_t^-$ denote the change in production from month $t$ to month $t+1$ and $d_t$ denote the sales forecast for month $t$.  Letting the units to be in thousands below:

\begin{align*}
  \begin{array}{r@{}r@{}r@{}r@{}r@{}r@{}r@{}r@{}r@{}l}
    \text{min} \quad&{} 0.5\sum_{t=0}^{11}x_t^+ &{} + 0.25\sum_{t=0}^{11}x_t^-&{}&{}&{} \\[\jot]
    \text{s.t.}\qquad4&{}+\phantom{\sum_{i=7}^{8}}x_0&{} +\phantom{\sum_{i=7}^{8}}2 -\phantom{\sum_{i=7}^{8}}d_1 &{}&{}&{}&{}\leq 10\\ [\jot]      
4&{}+\phantom{\sum_{i=7}^{8}}x_0&{} +\phantom{\sum_{i=7}^{8}}2-\phantom{\sum_{i=7}^{8}}d_1&{}+4&{}+\sum_{t=0}^{1}x_t&{} -d_2  &{}  \leq 10\\ [\jot]
2(4)&{}+\sum_{t=0}^{1}(2-t)x_t&{} +\phantom{\sum_{i=7}^{8}}2-\sum_{i=1}^{2}d_t&{}+4&{}+\sum_{t=0}^{2}x_t&{} -d_3&{}    \leq 10\\ [\jot]
3(4)&{}+\sum_{t=0}^{2}(3-t)x_t&{} +\phantom{\sum_{i=7}^{8}}2-\sum_{i=1}^{3}d_t&{}+4&{}+\sum_{t=0}^{3}x_t&{} -d_4&{}    \leq 10\\ [\jot]
\vdots&{}\vdots&{}\vdots&{}&{}\vdots&{}&{}&{}&{}&{}\\
11(4)&{}+\sum_{t=0}^{10}(11-t)x_t&{} +\phantom{\sum_{i=7}^{8}}2-\sum_{i=1}^{11}d_t&{}+4&{}+\sum_{t=0}^{11}x_t&{} -d_{12}&{}    \leq 10\\ [\jot]
 \multicolumn{4}{c}{x_i^+, x_i^- \geq 0, \quad i=1,\ldots,12}
  \end{array}
\end{align*}
\item[1.25]
We first list down the different ways such that a 100-inch roll can be cut into combinations of 24-, 40-, and 32- inch widths.
\begin{table}[h]
\centering
\begin{tabular}{c|c|c|c|c}
\hline
Combination&24 &40 & 32 & trim waste\\
\hline
1&4 &0 &0 & 4\\
 2&0 &2 & 0& 20\\
 3& 0 & 0& 3& 4\\
4&   1 &1 &1 &4 \\
 5&   2 &1 & 0& 12\\
  6&   2 &0 &1 & 20\\
   7&    1  &0 &2 & 12\\
       \hline
\end{tabular}
\end{table}
Let $x_i$ denote the number of combination $i$ used.
\begin{align*}
  \begin{array}{r@{}r@{}r@{}r@{}r@{}r@{}r@{}r@{}r@{}r@{}l}
    \text{min} \quad 4x_1 &{}+ 20x_2&{}+ 4x_3&{}+ 4x_4&{}+ 12x_5&{}+ 20x_6&{}+ 12x_7&{}+24x_8 &{} +40x_9 &{} +32x_{10}\\[\jot]
    \text{s.t.}\qquad  4x_1 &{} &{} &{} +x_4 &{} +2x_5&{}+2x_6&{}+x_7  &{} +x_8&{}&{}&{}= 75\\[\jot]
    &{} 2x_2&{} &{} +x_4 &{} +x_5&{}&{} &{}&{}+x_9&{}&{}= 50\\[\jot]
    &{} &{} 3x_3&{} +x_4 &{} &{}+x_6&{}+2x_7  &{} &{}&{}+x_{10}&{}= 110\\[\jot]
 \multicolumn{4}{c}{x_j \geq 0, \quad j=1,\ldots,10}
  \end{array}
\end{align*}

\item[2.7]
Letting the units be in thousands below: 
\begin{align*}
  \begin{array}{r@{}r@{}l}
    \text{max} \quad 2x_1 &{}+ 1.8x_2 \\[\jot]
    \text{s.t.}\qquad x_1 &{}+ x_2&{} \leq 10 \\
    2x_1 &{}+ x_2&{} \leq 9 \\
     \multicolumn{3}{c}{x_1, x_2 \geq 0}
  \end{array}
\end{align*}
\item[2.8]
Let the units be in pounds below, and $x_1, x_2$ and $x_3$ denoting amount of ingredient $A, B$ and $C$ used respectively.
\begin{align*}
  \begin{array}{r@{}r@{}r@{}l}
    \text{min} \quad 4x_1 &{}+ 3x_2&{}+2x_3 \\[\jot]
    \text{s.t.}\qquad    x_1 &{}+ x_2&{}+x_3&{}=2000 \\[\jot]
     \multicolumn{4}{c}{x_1 \geq 200, x_2 \geq 400 , 0 \leq x_3 \leq 800}
  \end{array}
\end{align*}
The bounded variable index method can be used to solve this problem.

\item[2.9]
Phase I:
\begin{align*}
  \begin{array}{r@{}r@{}r@{}r@{}r@{}r@{}r@{}r@{}r@{}l}
    \text{max} \quad x_0 &{}&{}\\[\jot]
    \text{s.t.}\qquad (-x_0) &{}&{}&{}&{}&{}&{}+x_6&{}+x_7&{}+x_8&{}=0\\[\jot]
    &{}+2x_1 &{}+3x_2&{}-x_3&{}+x_4 &{} &{} +x_6 &{} &{}&{}=9  \\[\jot]
    &{}&{}2x_2&{}+x_3&{} &{}-x_5 &{} &{}+x_7&{}&{} = 4 \\[\jot]
   &{}+x_1&{}&{}+x_3&{} &{}&{} &{}&{}+x_8&{}= 6 \\ [\jot]
     \multicolumn{2}{c}{x_i \geq 0,\, \forall i}
  \end{array}
\end{align*}
\begin{table}[h]
\centering
\begin{tabular}{c|ccc|c}
($-w$) & 0 & 0 & 0 & \\
\hline
$x_6$ & 1 & 0 & 0 & 9\\
$x_7$ & 1 & 0 & 0 & 4\\
$x_8$ & 1 & 0 & 0 & 6\\
\end{tabular}
\end{table}
\item[2.12]
\begin{enumerate}[(a)]
\item
\begin{align*}
  \begin{array}{r@{}r@{}r@{}r@{}r@{}r@{}r@{}r@{}l}
    \text{min} \quad x_6 &{} + x_7 &{} +x_8 \\[\jot]
    \text{s.t.}\qquad x_1 &{} -6x_2 &{} + \phantom{2}x_3 &{}- \phantom{3}x_4 &{}&{}+x_6&{}&{}&{}= 5 \\  
                           x_1 &{} -2x_2         &{} +   2x_3 &{}-3x_4&{} - x_5&{}&{}+x_7&{}&{} = 3 \\            
                     -3x_1 &{} &{}+2x_3&{}-4x_4    &{} &{}&{}&{}+x_8&{}= 1 \\         
 \multicolumn{8}{c}{x_j \geq 0, \quad j=1,\dots,8.}
  \end{array}
\end{align*}

\item
\begin{align*}
  \begin{array}{r@{}r@{}r@{}r@{}r@{}r@{}r@{}l}
    \text{max} \quad -x_6 &{} - x_7  \\[\jot]
    \text{s.t.}\qquad -3x_1 &{} -2x_2 &{} +\phantom{2}x_3 &{}-x_4 &{}&{}+x_6&{}&{}= 3 \\  
                           x_1 &{} +x_2   &{} -   2x_3 &{}&{} - x_5&{}&{}+x_7&{}= 1 \\            
 \multicolumn{8}{c}{x_j \geq 0, \quad j=1,\dots,7.}
  \end{array}\\
    \begin{array}{r@{}r@{}r@{}r@{}r@{}r@{}r@{}l}
    \text{max} \quad -4 &{} -2x_1 &{} - x_2&{}-x_3&{}-x_4&{}-x_5  \\[\jot]
    \text{s.t.}\qquad -3x_1 &{} -2x_2 &{} +\phantom{2}x_3 &{}-x_4 &{}&{}+x_6&{}&{}= 3 \\  
                           x_1 &{} +x_2   &{} -   2x_3 &{}&{} - x_5&{}&{}+x_7&{}= 1 \\            
 \multicolumn{8}{c}{x_j \geq 0, \quad j=1,\dots,7.}
  \end{array}
\end{align*}
thus since all the coefficients of the objective function is $\leq 0$ the system contains no feasible solution.
\end{enumerate}


\item[2.13]
\begin{enumerate}[(a)]
\item
\item
\item
\item
\end{enumerate}
\end{enumerate}
\end{document}