\documentclass[a4paper,12pt]{article}
\setlength{\parindent}{0cm}
\usepackage{amsmath, amssymb, amsthm, mathtools,pgfplots}
\usepackage{graphicx,caption}
\usepackage{verbatim}
\usepackage{venndiagram}
\usepackage[cm]{fullpage}
\usepackage{fancyhdr}
\usepackage{tikz}
\usepackage{listings}
\usepackage{color,enumerate,framed}
\usepackage{color,hyperref}
\definecolor{darkblue}{rgb}{0.0,0.0,0.5}
\hypersetup{colorlinks,breaklinks,
            linkcolor=darkblue,urlcolor=darkblue,
            anchorcolor=darkblue,citecolor=darkblue}

%\usepackage{tgadventor}
%\usepackage[nohug]{diagrams}
\usepackage[T1]{fontenc}
%\usepackage{helvet}
%\renewcommand{\familydefault}{\sfdefault}
%\usepackage{parskip}
%\usepackage{picins} %for \parpic.
%\newtheorem*{notation}{Notation}
%\newtheorem{example}{Example}[section]
%\newtheorem*{problem}{Problem}
\theoremstyle{definition}
%\newtheorem{theorem}{Theorem}
%\newtheorem*{solution}{Solution}
%\newtheorem*{definition}{Definition}
%\newtheorem{lemma}[theorem]{Lemma}
%\newtheorem{corollary}[theorem]{Corollary}
%\newtheorem{proposition}[theorem]{Proposition}
%\newtheorem*{remark}{Remark}
%\setcounter{section}{1}

\newtheorem{thm}{Theorem}[section]
\newtheorem{lemma}[thm]{Lemma}
\newtheorem{prop}[thm]{Proposition}
\newtheorem{cor}[thm]{Corollary}
\newtheorem{defn}[thm]{Definition}
\newtheorem*{examp}{Example}
\newtheorem{conj}[thm]{Conjecture}
\newtheorem{rmk}[thm]{Remark}
\newtheorem*{nte}{Note}
\newtheorem*{notat}{Notation}

%\diagramstyle[labelstyle=\scriptstyle]

\lstset{frame=tb,
  language=Oz,
  aboveskip=3mm,
  belowskip=3mm,
  showstringspaces=false,
  columns=flexible,
  basicstyle={\small\ttfamily},
  breaklines=true,
  breakatwhitespace=true,
  tabsize=3
}


\pagestyle{fancy}




\fancyhead{}
\renewcommand{\headrulewidth}{0pt}

\lfoot{\color{black!60}{\sffamily Zhangsheng Lai}}
\cfoot{\color{black!60}{\sffamily Last modified: \today}}
\rfoot{\textsc{\thepage}}



\begin{document}
\flushright{Zhangsheng Lai\\1002554}
\section*{Real Analysis: Homework 2}

\begin{enumerate}

\item 
\begin{enumerate}[(a)]
\item Let $f(x,y) = \cosh x \cosh y$, with $\vec{x} = (0,0)$, $\vec{v}=(x,y)$,
\begin{align*}
F(h): = f(\vec{x}+h\vec{v}) = f(h\vec{v}) = \cosh hx\cosh hy
\end{align*}
%We first compute the following,
%\begin{alignat*}{2}
%\frac{\partial f(h\vec{v})}{\partial e_1^3} &= \sinh hx \cosh hy &\quad  \frac{\partial f(h\vec{v})}{\partial e_2^3} &= \cosh hx \sinh hy \\
%\frac{\partial f(h\vec{v})}{\partial e_1e_2e_1}=\frac{\partial f(h\vec{v})}{\partial e_1^2e_2} &= \cosh hx \sinh hy &\quad  \frac{\partial f(h\vec{v})}{\partial e_2e_1e_2} &= \frac{\partial f(h\vec{v})}{\partial e_2^2e_1}=\sinh hx \cosh hy \\
%\frac{\partial f(h\vec{v})}{\partial e_1e_2^2}&= \sinh hx \cosh hy &\quad  \frac{\partial f(h\vec{v})}{\partial e_2e_1^2} &= \cosh hx \sinh hy 
%\end{alignat*}
then
\begin{align*}
F'(h)=\left\langle\nabla f(h\vec{v}),\vec{v}\right\rangle  &= x \sinh hx \cosh hy + y \cosh hx \sinh hy\\
F''(h)=\nabla^2 f(h\vec{v})(\vec{v},\vec{v})&=\begin{bmatrix}x & y\end{bmatrix}\begin{bmatrix}\cosh hx \cosh hy & \sinh hx \sinh hy \\ \sinh hx \sinh hy &\cosh hx \cosh hy \end{bmatrix}\begin{bmatrix}x \\ y\end{bmatrix}\\
%F'''(h)=\nabla^3 f(h\vec{v})(\vec{v},\vec{v},\vec{v}) &= \sum_{i,j,k=1,2}\frac{f(h\vec{v})}{\partial e_i\partial e_j\partial e_k}v_iv_jv_k \\
%&= (x^3 + 3xy^2)(\sinh hx \cosh hy) + (y^3+3x^2y)(\cosh hx \cosh hy)
\end{align*}
and 
\begin{alignat*}{2}
F(0) &= 1 \quad F'(0) &= 0 \quad  F''(0) &= x^2 + y^2 %\quad F'''(0) = y^3+3x^2y
\end{alignat*}
%Thus the polynomial of third degree that best approximate $f(x,y)$ is $1+\frac{1}{2}(x^2+y^2)+ \frac{1}{6}(y^3+3x^2y)$.
Thus the polynomial of second degree that best approximate $f(x,y)$ is $1+\frac{1}{2}(x^2+y^2)$.

%\begin{align*}
%\left.\nabla f \right|_{(0,0)} &= \left.\begin{bmatrix}\sinh x \cosh y \\ \cosh x \sinh y \end{bmatrix}\right|_{(0,0)} = \begin{bmatrix}0 \\ 0\end{bmatrix}\\
%\left.\nabla^2 f \right|_{(0,0)} &= \left.\begin{bmatrix}\cosh x \cosh y & \sinh x \sinh y \\ \sinh x \sinh y &\cosh x \cosh y \end{bmatrix}\right|_{(0,0)} = \begin{bmatrix}1 & 0 \\ 0 & 1\end{bmatrix}
%\end{align*}


\item Let $g(x,y) = \sin (x^2+y^2)$, with $\vec{x} = (0,0)$, $\vec{v}=(x,y)$,
\begin{align*}
G(h): = g(\vec{x}+h\vec{v}) = g(h\vec{v}) = \sin ((hx)^2+(hy)^2)
\end{align*}
then
\begin{align*}
G'(h)&=\left\langle\nabla g(h\vec{v}),\vec{v}\right\rangle  = x (2hx\cos((hx)^2+(hy)^2)) + y (2hy\cos((hx)^2+(hy)^2))\\
G''(h)&=\nabla^2 g(h\vec{v})(\vec{v},\vec{v})\\
&=x^2(2\cos((hx)^2+(hy)^2)-4(xh)^2\sin((hx)^2+(hy)^2))\\
&-2xy(4xyh^2\sin((hx)^2+(hy)^2))\\
&+y^2(2\cos((hx)^2+(hy)^2)-4(yh)^2\sin((hx)^2+(hy)^2))\\
%G'''(h)&=\nabla^3 g(h\vec{v})(\vec{v},\vec{v},\vec{v}) = \sum_{i,j,k=1,2}\frac{g(h\vec{v})}{\partial e_i\partial e_j\partial e_k}v_iv_jv_k\\
%&=x^3(-8(hx)^3 \cos((hx)^2+(hy)^2) - 12 (hx) \sin((hx)^2+(hy)^2))\\
%&y^3(-8(hy)^3 \cos((hx)^2+(hy)^2) - 12 (hy) \sin((hx)^2+(hy)^2))\\
%&+3x^2y(-8(hx)^2(hy)\cos((hx)^2+(hy)^2)-4(hy)\sin((hx)^2+(hy)^2))\\
%&+3xy^2(-8(hx)(hy)^2\cos((hx)^2+(hy)^2)-4(hx)\sin((hx)^2+(hy)^2))
\end{align*}
and 
\begin{alignat*}{2}
G(0) &= 0 \quad G'(0) &= 0 \quad  G''(0) &= 2x^2 + 2y^2 %\quad G'''(0) = 0
\end{alignat*}
%Thus the polynomial of third degree that best approximate $g(x,y)$ is $x^2+y^2$.	
Thus the polynomial of second degree that best approximate $g(x,y)$ is $x^2+y^2$.	
\end{enumerate}



\item
\begin{enumerate}[(a)]
\item We observe that $f(x,y)$ is continuous at all points $(x,y) \neq (0,0)$ since the denominator is nonzero. Thus we need to show that $\lim_{(x,y) \to (0,0)}f(x,y)=0$ to show $f(x,y)$ is everywhere continuous.
\begin{align*}
\lim_{(x,y) \to (0,0)}\frac{xy(x^2-y^2)}{x^2+y^2} = \lim_{(x,y) \to (0,0)}xy\frac{x^2}{x^2+y^2} - \lim_{(x,y) \to (0,0)}xy\frac{y^2}{x^2+y^2}
\end{align*}
then since,
\begin{align*}
0 \leq \frac{x^2}{x^2+y^2} \leq 1~,&\qquad -|xy| \leq xy \leq |xy|\\
-|xy| \leq &xy\frac{x^2}{x^2+y^2} \leq|xy|\\
\end{align*}
and $\lim_{(x,y) \to (0,0)}\pm|xy|=0$, which by Squeeze theorem gives us $\lim_{(x,y) \to (0,0)}xy\frac{x^2}{x^2+y^2}=0$. We use the same argument and get the limit to be zero for the second term which shows the desired.
%\begin{align*}
%\lim_{(x,y) \to (0,0)}\frac{xy(x^2-y^2)}{x^2+y^2} = \lim_{(x,y) \to (0,0)}xy\cdot \lim_{(x,y) \to (0,0)}\left(\frac{1}{1+(y/x)^2}+\frac{1}{1+(x/y)^2}\right)
%\end{align*}
%Let $z = x/y$, then $h(z) = 1/(1+z^2)$ is a continuous function for all $x,y \in \mathbb{R}$. Thus if $\lim_{(x,y) \to (0,0)}z = z_0$, $\lim_{(x,y) \to (0,0)}h(z) = h(z_0)$. Using L' Hopital's Rule, we get $z_0=1$ and thus the limit of the second term above is 2. Since the limit of the first term above is 0, we show that as $(x,y)\to (0,0)$, $f(x,y)\to 0$. Thus $f(x,y)$ is everywhere continuous.
We have $\nabla f = (f_x, f_y)$, which are given below,
\begin{align*}
f_x(x,y) = \frac{x^4y+4x^2y^3-y^5}{(x^2+y^2)^2},\qquad f_y(x,y) = \frac{x^5-4x^3y^2-xy^4}{(x^2+y^2)^2}
\end{align*}
Again it suffices to show that it is continuous at $(0,0)$. We see that 
\begin{align*}
f_x = y \left(1-\frac{2y^4}{(x^2+y^2)^2}\right), \qquad f_y = -x \left(1-\frac{2x^4}{(x^2+y^2)^2}\right)
\end{align*}
and $-1\leq 1-\frac{2y^4}{(x^2+y^2)^2},1-\frac{2x^4}{(x^2+y^2)^2}\leq 1$. Thus 
\begin{align*}
-y\leq &y\left(1-\frac{2y^4}{(x^2+y^2)^2}\right)\leq y\\
-x \leq &-x\left(1-\frac{2x^4}{(x^2+y^2)^2}\right)\leq x
\end{align*}
which by Squeeze theorem, we get $\lim_{(x,y)\to (0,0)}f_x = 0 = \lim_{(x,y)\to (0,0)}f_y$.

\item The computation gives,
\begin{align*}
\frac{\partial^2f}{\partial x \partial y} = \frac{x^6+9x^4y^2-9x^2y^4-y^6}{(x^2+y^2)^3} = \frac{\partial^2f}{\partial y \partial x}
\end{align*}
To show it is continuous everywhere we have to show that $\lim_{(x,y)\to(0,0)}\frac{x^6+9x^4y^2-9x^2y^4-y^6}{(x^2+y^2)^3}$ exists and is well defined along different paths.
\begin{align*}
\text{Along $y=x$, }\lim_{(x,y)\to(0,0)}\frac{x^6+9x^4y^2-9x^2y^4-y^6}{(x^2+y^2)^3} &= 0\\
\text{Along $y=2x$, } \lim_{(x,y)\to(0,0)}\frac{x^6+9x^4y^2-9x^2y^4-y^6}{(x^2+y^2)^3}&= \lim_{(x,y)\to(0,0)}\frac{x^6+36x^6-144x^6-64x^5}{(x^2+4x^2)^3} \\
&=-171/125
\end{align*}
\end{enumerate}

\item We recall the geometric series,
\begin{align*}
\frac{1}{1-x} &= \sum_{k=0}^{\infty}x^k~, \quad\text{ where $|x|<1$}\\
\text{substituting $x$ with $-x$ }, \frac{1}{1+x} &= \sum_{k=0}^{\infty}(-1)^kx^k~, \quad\text{ where $|x|<1$}\\
\text{substituting $x$ with $x^2$ }, \frac{1}{1+x^2} &= \sum_{k=0}^{\infty}(-1)^kx^{2k}~, \quad\text{ where $|x|<1$}
\end{align*}
we can then do integration term wise on the right hand side while integrating $\frac{1}{1+x^2}$,
\begin{align*}
\pi/4=\tan^{-1}(1) = \int_{0}^{1}\frac{1}{1+t^2}\,dt &= \sum_{k=0}^{\infty}(-1)^k\int_{0}^{1}t^{2k}\,dt\\
&= \sum_{k=0}^{\infty}(-1)^k\frac{1}{2k+1}\\
\end{align*}
thus $\pi = 4\sum_{k=0}^{\infty}(-1)^k\frac{1}{2k+1} = 4\left(1-\frac{1}{3}+\frac{1}{5}-\frac{1}{7}+\ldots\right)$.


\item
\begin{enumerate}[(a)]
\item Let $f(t) = e^{-\frac{(x-t)^2}{2}} = e^{-x^2/2}\cdot e^{xt-\frac{t^2}{2}}$. Then $f(t) = \sum_{k=0}^{\infty}\frac{f^{(k)}(0)}{k!}t^k$, where
\begin{align*}
f^{(k)}(0) &= \left.\frac{d^k}{dt^k}e^{-\frac{(x-t)^2)}{2}}\right|_{t=0}\\
&= \left.(-1)^k\frac{d^k}{du^k}e^{-\frac{u^2}{2}}\right|_{u=x}\quad \text{ letting $u=x-t$, so $\frac{d}{du}=-\frac{d}{dt}$}\\
&= (-1)^k\frac{d^k}{dx^k}e^{-\frac{x^2}{2}}
\end{align*}
then
\begin{align*}
e^{-x^2/2}\cdot e^{xt-\frac{t^2}{2}} &= \sum_{k=0}^{\infty}\frac{(-1)^kt^k}{k!}\frac{d^k}{dx^k}e^{-\frac{x^2}{2}}\\
e^{xt-\frac{t^2}{2}} &= e^{x^2/2}\sum_{k=0}^{\infty}\frac{(-1)^kt^k}{k!}\frac{d^k}{dx^k}e^{-\frac{x^2}{2}}\\
&= \sum_{k=0}^{\infty}t^kH_k(x)
\end{align*}
It converges because it is a Taylor series of an exponential which has radius of convergence, $R < \infty$.


%We will show the result by doing Taylor's expansion at $0$, which requires us to compute the higher derivatives of $f(t) = e^{tx-\frac{t^2}{2}}$.
%\begin{align*}
%f'(t) &= (x-t) f(t) = P_1(t)f(t)\\
%f''(t) &= \left[(x-t)^2-1\right] f(t) = P_2(t)f(t)\\
%f'''(t) &= \left[(x-t)^3-3(x-t)\right] f(t)= P_3(t)f(t)\\
%&\vdots\\
%f^{(n)}(t) &= P_n(t)f(t)\\
%\end{align*}
%then we would now like to show that $P_i(t) = P'_{i-1}(t)+(x-t)P_{i-1}(t)$, $P_0 = 1$. It is clear that it is true for $i=1$. Suppose it is true for $i=k$, $f^{(k)}(t) = P_k(t)f(t)$, then
%\begin{align*}
%f^{(k+1)}(t) &= P'_k(t)f(t) + P_k(t)f'(t)\\
%&=P'_k(t)f(t)+(x-t)P_k(t)f(t)
%\end{align*}
%which completes the proof by induction.
%We know that $e^x = \sum_{k=0}^{\infty}\frac{x^k}{k!}$, then 
%\begin{align*}
%H_n(x) &:= \frac{(-1)^n}{n!}e^{\frac{x^2}{2}}\frac{d^n}{dx^n}e^{-\frac{x^2}{2}}\\
%e^{tx-\frac{t^2}{2}} &= \sum_{k=0}^{\infty}\frac{(tx-\frac{t^2}{2})^k}{k!} \\
%&=\sum_{k=0}^{\infty}t^k \frac{(-1)^k}{k!}e^{\frac{x^2}{2}}\frac{d^k}{dx^k}e^{-\frac{x^2}{2}}\\
%\end{align*}

\item
\begin{enumerate}[(i)]
\item 
\begin{align*}
H'_n(x) &= xH_n(x)+\frac{(-1)^n}{n!}e^{\frac{x^2}{2}}\frac{d^{n+1}}{dx^{n+1}}e^{-\frac{x^2}{2}}\\
&= xH_n(x)+\frac{(-1)^n}{n!}e^{\frac{x^2}{2}}\frac{d^{n}}{dx^{n}}(-xe^{-\frac{x^2}{2}})\\
&= xH_n(x)+\frac{(-1)^n}{n!}e^{\frac{x^2}{2}}\left[-x\frac{d^n}{dx^n}e^{-\frac{x^2}{2}}-n\frac{d^{n-1}}{dx^{n-1}}e^{-\frac{x^2}{2}}\right],~\text{by General Leibniz Rule}\\
&=\frac{(-1)^{n-1}}{(n-1)!}e^{\frac{x^2}{2}}\frac{d^{n-1}}{dx^{n-1}}e^{-\frac{x^2}{2}}=H_{n-1}(x)
\end{align*}
%We first show a small result, for $f(x) = e^{-\frac{x^2}{2}}$, $f^{n}(x) = P_n(x)f(x)$ where
%\begin{align*}
%P_i(x) = P'_{i-1}(x) -xP_{i-1}(x), \quad \text{with $P_0=1$.}
%\end{align*}
%\begin{align*}
%f'(x) &=-x f(x) = P_1(x)f(x)\\
%f''(x) &= \left[-(1+x)^2\right] f(x) = P_2(x)f(x)\\
%f'''(x) &= \left[x^3-x\right] f(t)= P_3(t)f(t)\\
%&\vdots\\
%f^{(n)}(t) &= P_n(t)f(t)\\
%\end{align*}
%It is clear that it is true for $i=1$. Suppose it is true for $i=k$, $f^{(k)}(t) = P_k(t)f(t)$, then
%\begin{align*}
%f^{(k+1)}(x) &= P'_k(x)f(x) + P_k(x)f'(x)\\
%&=P'_k(x)f(x)-xP_k(x)f(x)
%\end{align*}
%which completes the proof by induction.


\item 
\begin{align*}
(n+1)H_n(x) & = (n+1)\frac{(-1)^{n+1}}{(n+1)!}e^{\frac{x^2}{2}}\frac{d^{n+1}}{dx^{n+1}}e^{-\frac{x^2}{2}}\\
& = \frac{(-1)^{n+1}}{n!}e^{\frac{x^2}{2}}\frac{d^{n}}{dx^{n}}-xe^{-\frac{x^2}{2}}\\
& = \frac{(-1)^{n+1}}{n!}e^{\frac{x^2}{2}}\left[-x\frac{d^n}{dx^n}e^{\frac{-x^2}{2}}-n\frac{d^{n-1}}{dx^{n-1}}e^{\frac{-x^2}{2}}\right]\\
&=xH_n(x) - H_{n-1}(x)
\end{align*}


\item Let $y=-x$, then
\begin{align*}
H_n(y) &= \frac{(-1)^n}{n!}e^{\frac{y^2}{2}}\frac{d^n}{dy^n}e^{-\frac{y^2}{2}}\\
&= \frac{(-1)^n}{n!}e^{\frac{x^2}{2}}(-1)^n\frac{d^n}{dx^n}e^{-\frac{x^2}{2}},\quad \text{ since $\frac{d}{dy}=-\frac{d}{dx}$}\\
&=(-1)^nH_n(x)
\end{align*}
\end{enumerate}

\item
\begin{enumerate}[(i)]
\item 
\begin{align*}
\frac{1}{\sqrt{2\pi}}\int_{\mathbb{R}}e^{sx-\frac{s^2}{2}}\,e^{tx-\frac{t^2}{2}}\,e^{-\frac{x^2}{2}}\,dx &=e^{st}\frac{1}{\sqrt{2\pi}}\int_{\mathbb{R}}e^{-\frac{(x-(s+t))^2}{2}}\,dx = e^{st}
\end{align*}
\item Applying the partial derivatives $\left.\frac{\partial^{n+m}}{\partial s^n\partial t^m}\right|_{s,t=0}$ to (i), we get
\begin{align*}
0=\left.\frac{\partial^{n+m}}{\partial s^n\partial t^m}\right|_{s,t=0}e^{st} &= \left.\frac{\partial^{n+m}}{\partial s^n\partial t^m}\right|_{s,t=0}\frac{1}{\sqrt{2\pi}}\int_{\mathbb{R}}e^{sx-\frac{s^2}{2}}e^{tx-\frac{t^2}{2}}e^{-\frac{x^2}{2}}\,dx\\&= \frac{1}{\sqrt{2\pi}}\int_{\mathbb{R}}\left.\frac{\partial^{n}}{\partial s^n}\right|_{s=0}e^{sx-\frac{s^2}{2}}\left.\frac{\partial^{m}}{\partial t^m}\right|_{t=0}e^{tx-\frac{t^2}{2}}e^{-\frac{x^2}{2}}\,dx\\
\end{align*}
 and since from 4(a),
\begin{align*}
\left.\frac{\partial^{n}}{\partial s^n}\right|_{s=0}e^{sx-\frac{s^2}{2}}=H_n(x)
\end{align*}
we are done.
\end{enumerate}
\end{enumerate}
\end{enumerate}












\end{document}