\documentclass[a4paper,12pt]{article}
\setlength{\parindent}{0cm}
\usepackage{amsmath, amssymb, amsthm, mathtools,pgfplots}
\usepackage{graphicx,caption}
\usepackage{verbatim}
\usepackage{venndiagram}
\usepackage[cm]{fullpage}
\usepackage{fancyhdr}
\usepackage{tikz}
\usepackage{listings}
\usepackage{color,enumerate,framed}
\usepackage{color,hyperref}
\definecolor{darkblue}{rgb}{0.0,0.0,0.5}
\hypersetup{colorlinks,breaklinks,
            linkcolor=darkblue,urlcolor=darkblue,
            anchorcolor=darkblue,citecolor=darkblue}

%\usepackage{tgadventor}
%\usepackage[nohug]{diagrams}
\usepackage[T1]{fontenc}
%\usepackage{helvet}
%\renewcommand{\familydefault}{\sfdefault}
%\usepackage{parskip}
%\usepackage{picins} %for \parpic.
%\newtheorem*{notation}{Notation}
%\newtheorem{example}{Example}[section]
%\newtheorem*{problem}{Problem}
\theoremstyle{definition}
%\newtheorem{theorem}{Theorem}
%\newtheorem*{solution}{Solution}
%\newtheorem*{definition}{Definition}
%\newtheorem{lemma}[theorem]{Lemma}
%\newtheorem{corollary}[theorem]{Corollary}
%\newtheorem{proposition}[theorem]{Proposition}
%\newtheorem*{remark}{Remark}
%\setcounter{section}{1}

\newtheorem{thm}{Theorem}[section]
\newtheorem{lemma}[thm]{Lemma}
\newtheorem{prop}[thm]{Proposition}
\newtheorem{cor}[thm]{Corollary}
\newtheorem{defn}[thm]{Definition}
\newtheorem*{examp}{Example}
\newtheorem{conj}[thm]{Conjecture}
\newtheorem{rmk}[thm]{Remark}
\newtheorem*{nte}{Note}
\newtheorem*{notat}{Notation}

%\diagramstyle[labelstyle=\scriptstyle]

\lstset{frame=tb,
  language=Oz,
  aboveskip=3mm,
  belowskip=3mm,
  showstringspaces=false,
  columns=flexible,
  basicstyle={\small\ttfamily},
  breaklines=true,
  breakatwhitespace=true,
  tabsize=3
}


\pagestyle{fancy}




\fancyhead{}
\renewcommand{\headrulewidth}{0pt}

\lfoot{\color{black!60}{\sffamily Zhangsheng Lai}}
\cfoot{\color{black!60}{\sffamily Last modified: \today}}
\rfoot{\textsc{\thepage}}



\begin{document}
\flushright{Zhangsheng Lai\\1002554}
\section*{Real Analysis: Homework 2}

\begin{enumerate}

\item 
\begin{enumerate}[(a)]
\item Let $f(x,y) = \cosh x \cosh y$, with $\vec{x} = (0,0)$, $\vec{v}=(x,y)$,
\begin{align*}
F(h): = f(\vec{x}+h\vec{v}) = f(h\vec{v}) = \cosh hx\cosh hy
\end{align*}
%We first compute the following,
%\begin{alignat*}{2}
%\frac{\partial f(h\vec{v})}{\partial e_1^3} &= \sinh hx \cosh hy &\quad  \frac{\partial f(h\vec{v})}{\partial e_2^3} &= \cosh hx \sinh hy \\
%\frac{\partial f(h\vec{v})}{\partial e_1e_2e_1}=\frac{\partial f(h\vec{v})}{\partial e_1^2e_2} &= \cosh hx \sinh hy &\quad  \frac{\partial f(h\vec{v})}{\partial e_2e_1e_2} &= \frac{\partial f(h\vec{v})}{\partial e_2^2e_1}=\sinh hx \cosh hy \\
%\frac{\partial f(h\vec{v})}{\partial e_1e_2^2}&= \sinh hx \cosh hy &\quad  \frac{\partial f(h\vec{v})}{\partial e_2e_1^2} &= \cosh hx \sinh hy 
%\end{alignat*}
then
\begin{align*}
F'(h)=\left\langle\nabla f(h\vec{v}),\vec{v}\right\rangle  &= x \sinh hx \cosh hy + y \cosh hx \sinh hy\\
F''(h)=\nabla^2 f(h\vec{v})(\vec{v},\vec{v})&=\begin{bmatrix}x & y\end{bmatrix}\begin{bmatrix}\cosh hx \cosh hy & \sinh hx \sinh hy \\ \sinh hx \sinh hy &\cosh hx \cosh hy \end{bmatrix}\begin{bmatrix}x \\ y\end{bmatrix}\\
F'''(h)=\nabla^3 f(h\vec{v})(\vec{v},\vec{v},\vec{v}) &= \sum_{i,j,k=1,2}\frac{f(h\vec{v})}{\partial e_i\partial e_j\partial e_k}v_iv_jv_k \\
&= (x^3 + 3xy^2)(\sinh hx \cosh hy) + (y^3+3x^2y)(\cosh hx \cosh hy)
\end{align*}
and 
\begin{alignat*}{2}
F(0) &= 0 \quad F'(0) &= 0 \quad  F''(0) &= x^2 + y^2 \quad F'''(0) = 
\end{alignat*}
Thus the polynomial of second degree that best approximate $f(x,y)$ is $\frac{1}{2}(x^2+y^2)$.

%\begin{align*}
%\left.\nabla f \right|_{(0,0)} &= \left.\begin{bmatrix}\sinh x \cosh y \\ \cosh x \sinh y \end{bmatrix}\right|_{(0,0)} = \begin{bmatrix}0 \\ 0\end{bmatrix}\\
%\left.\nabla^2 f \right|_{(0,0)} &= \left.\begin{bmatrix}\cosh x \cosh y & \sinh x \sinh y \\ \sinh x \sinh y &\cosh x \cosh y \end{bmatrix}\right|_{(0,0)} = \begin{bmatrix}1 & 0 \\ 0 & 1\end{bmatrix}
%\end{align*}


\item Let $g(x,y) = \sin (x^2+y^2)$, with $\vec{x} = (0,0)$, $\vec{v}=(x,y)$,
\begin{align*}
F(h): = g(\vec{x}+h\vec{v}) = g(h\vec{v}) = \sin ((hx)^2+(hy)^2)
\end{align*}
then
\begin{align*}
F'(h)&=\left\langle\nabla g(h\vec{v}),\vec{v}\right\rangle  = x (2hx\cos((hx)^2+(hy)^2)) + y (2hy\cos((hx)^2+(hy)^2))\\
F''(h)&=\nabla^2 g(h\vec{v})(\vec{v},\vec{v})\\
&=x^2(2\cos((hx)^2+(hy)^2)-4(xh)^2\sin((hx)^2+(hy)^2))\\
&-2xy(4xyh^2\sin((hx)^2+(hy)^2))\\
&+y^2(2\cos((hx)^2+(hy)^2)-4(yh)^2\sin((hx)^2+(hy)^2))
%\begin{bmatrix}x & y\end{bmatrix}\begin{bmatrix}2\cos((hx)^2+(hy)^2)-4x^2\sin((hx)^2+(hy)^2) & -4xy\sin((hx)^2+(hy)^2) \\ -4xy\sin((hx)^2+(hy)^2) &2\cos((hx)^2+(hy)^2)-4y^2\sin((hx)^2+(hy)^2) \end{bmatrix}\begin{bmatrix}x \\ y\end{bmatrix}\\
\end{align*}
and 
\begin{alignat*}{2}
F(0) &= 0 \quad F'(0) &= 0 \quad  F''(0) &= 2x^2 + 2y^2
\end{alignat*}
Thus the polynomial of second degree that best approximate $g(x,y)$ is $x^2+y^2$.	
\end{enumerate}



\item
\begin{enumerate}[(a)]
\item


\item 
\begin{align*}
\frac{\partial^2f}{\partial x \partial y}
\end{align*}
\end{enumerate}

\item We recall the geometric series,
\begin{align*}
\frac{1}{1-x} &= \sum_{k=0}^{\infty}x^k~, \quad\text{ where $|x|<1$}\\
\text{substituting $x$ with $-x$ }, \frac{1}{1+x} &= \sum_{k=0}^{\infty}(-1)^kx^k~, \quad\text{ where $|x|<1$}\\
\text{substituting $x$ with $x^2$ }, \frac{1}{1+x^2} &= \sum_{k=0}^{\infty}(-1)^kx^{2k}~, \quad\text{ where $|x|<1$}
\end{align*}
we can then do integration term wise on the right hand side while integrating $\frac{1}{1+x^2}$,
\begin{align*}
\pi/4=\tan^{-1}(1) = \int_{0}^{1}\frac{1}{1+t^2}\,dt &= \sum_{k=0}^{\infty}(-1)^k\int_{0}^{1}t^{2k}\,dt\\
&= \sum_{k=0}^{\infty}(-1)^k\frac{1}{2k+1}\\
\end{align*}
thus $\pi = 4\sum_{k=0}^{\infty}(-1)^k\frac{1}{2k+1} = 4\left(1-\frac{1}{3}+\frac{1}{5}-\frac{1}{7}+\ldots\right)$.


\item
\begin{enumerate}[(a)]
\item We know that $e^x = \sum_{k=0}^{\infty}\frac{x^k}{k!}$, then 
\begin{align*}
H_n(x) &:= \frac{(-1)^n}{n!}e^{\frac{x^2}{2}}\frac{d^n}{dx^n}e^{-\frac{x^2}{2}}\\
e^{tx-\frac{t^2}{2}} &= \sum_{k=0}^{\infty}\frac{(tx-\frac{t^2}{2})^k}{k!} \\
&=\sum_{k=0}^{\infty}t^k \frac{(-1)^k}{k!}e^{\frac{x^2}{2}}\frac{d^k}{dx^k}e^{-\frac{x^2}{2}}\\
\end{align*}

\item
\begin{enumerate}[(i)]
\item \begin{align*}
H_n'(x = )
\end{align*}
\end{enumerate}

\item
\end{enumerate}
\end{enumerate}












\end{document}